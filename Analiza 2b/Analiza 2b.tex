\documentclass[a4paper, 10pt]{article}
\usepackage[T1]{fontenc}
\usepackage[utf8]{inputenc}
\usepackage[slovene]{babel}
\usepackage{lmodern}
\usepackage{amsmath}
\usepackage{leftidx}
\usepackage{amssymb}
\usepackage{amsfonts}
\usepackage{graphicx}
\usepackage{wrapfig}
\usepackage{amsthm}
\usepackage{mathrsfs}
\usepackage{mathtools}
\usepackage{url}
\usepackage{subfigure}
\usepackage{multirow}
\usepackage{lipsum}
\usepackage{wrapfig}
\usepackage{tikz}
\usepackage[format=plain, font=small, labelfont=bf, textfont=it, justification=centerlast]{caption}
\usepackage{booktabs}
\usepackage{siunitx}

\newtheorem{trditev}{Trditev}
\newtheorem{izr}{Izrek}

\newcounter{defcount}
\newcounter{opombe}
\newcounter{zgledcount}

\newenvironment{opomba}{\begin{flushleft}\stepcounter{opombe}\textbf{Opomba \arabic{opombe}:}}{\hfill\end{flushleft}}
\setlength{\parindent}{0mm}

\newenvironment{zgled}{\begin{flushleft}\stepcounter{zgledcount}\textbf{Zgled \arabic{zgledcount}:}}{\hfill\end{flushleft}}
\setlength{\parindent}{0mm}

\newenvironment{definicija}{\begin{flushleft}\stepcounter{defcount}\textbf{Definicija \arabic{defcount}:}}{\hfill\end{flushleft}}
\setlength{\parindent}{0mm}

\newcommand{\abs}[1]{\ensuremath{\lvert #1 \rvert}}
\newcommand{\mth}[1]{\ensuremath{\mathbb{#1}}}
\newcommand{\R}{\mth{R}}
\newcommand{\Z}{\mth{Z}}
\newcommand{\N}{\mth{N}}
\newcommand{\No}{\mth{N}_0}
\newcommand{\C}{\mth{C}}
\newcommand{\Qu}{\mth{Q}_u}
\newcommand{\pojem}[1]{\emph{#1}}
\newcommand{\con}{\ensuremath{\mathscr{C}}}
\newcommand{\padex}[2]{\ensuremath{{#1}^{\underline{#2}}}}
\newcommand{\rastx}[2]{\ensuremath{{#1}^{\bar{#2}}}}
\newcommand{\map}[3]{\ensuremath{{#1}: {#2} \rightarrow {#3}}}
\newcommand{\pra}[3]{{#1}{\ast}({#2}) = {#3}}

\title{Analiza 2b Episode II: The mnogoterosti strike back \\ Predavanja}
\date{18.2.2019}
\author{Nek študent FMF}
%===============================================================================
\begin{document}
\maketitle
\newpage

\begin{abstract}
\noindent Ta kreacija, ki jo trenutno berete, dragi bralec, je skupek prepisanih zapiskov študijskega predmeta Analiza 2b s perspektive nekega študenta 2. letnika matematike na FMF.
Uporaba tega dokumenta za kakršnikoli namene je na lastno odgovornost.
\end{abstract}

\newpage

\section{Podmnogoterosti v $\R^n$}

\subsection{Uvod}

\begin{opomba}
Pri tem predmetu ne bomo posebej definirali kaj so mnogoterosti, zanimale nas bodo samo podmnogoterosti, ki so gladke, torej so vsaj $\con^1$.
\end{opomba}

\begin{definicija}
Naj bo $\Omega$ odprta podmnožica v $\R^n$ in naj bo $a \in \Omega$. Naj bo $F: \Omega \rightarrow \R^m $ preslikava, ki je vsaj $\con^1$. \pojem{Rang} preslikave $F$ v točki $a$ je rang odvoda preslikave $F$ v točki $a$ oziroma:
\[
rang(F)(a) = rang (DF)(a)
\]
$F$ ima rang $k$ na $\Omega$, če ima rang $k$ v vsaki točki $a \in \Omega$.
\end{definicija}

\begin{definicija}
Naj bo $M \subseteq \R^n$ neprazna množica. $M$ je (gladka) \pojem{podmnogoterost} razsežnosti/dimenzije $k$ v $\R^n$, če za vsak $x_0 \in M$ obstajajo taka odprta okolica $U$ točke $x_0$ in $\con^1$ funkcije $F_1, \ldots, F_{n-k}$ ($0 \leq k \leq n$) na $U$, da ima preslikava $F = (F_1, \ldots, F_{n-k}): U \rightarrow \R^{n-k}$ maksimalen možen rang (torej $n-k$) in da velja:
\[
M \cap U = \{x \in U \mid F(x) = 0\}
\]
\begin{itemize}

\item $F_1, \ldots, F_{n-k}$ so lokalne definicijske funkcije $M$ v okolici točke $x_0$
\item $n-k$ je \pojem{kodimenzija} podmnogoterosti $M$
\item Če je $k = n - 1$ ali $n - k = 1$ je $M$ \pojem{hiperploskev}
\item Če je $k = 1$ je $M$ krivulja, če je $k = 2$ pa je $M$ ploskev
\item Vsaka odprta množica v $\R^n$ je podmnogoterost dimenzije $n$.

\end{itemize}

\end{definicija}

\begin{zgled}
\\
$n = 3, k = 2, n-k = 1$
\[
A = \{ (x, y, z) \in \R^3 \mid x^2 + y^2 + z^2 = 1 \} 
\]
Definiramo funkcijo $F$:
\[
F(x, y, z) =  x^2 + y^2 + z^2 - 1 = 0 
\]
\[
(x, y, z) \mapsto F(x, y, z) 
\]
tedaj velja:
\[
A = \{ (x, y, z) \in \R^3 \mid F(x, y, z) = 0\} 
\]
\[
(DF)(x, y, z) = 0 \iff x = y = z = 0 
\]
Iz definicije množice $A$ pa vemo, da $(0, 0, 0) \notin A$, torej je $F$ (globalna) definicijska funkcija za $A$.
\end{zgled}

\begin{zgled}
\\
$n = 3, k = 1, n-k = 2$
\[
N = \{ (x, y, z) \in \R^3 \mid x^2 + y^2 + z^2 = 1; x + y + z = 0\}
\]
Za N definirajmo definicijski funkciji $F$ in $G$:
\[
F(x, y, z) =  x^2 + y^2 + z^2 - 1 = 0 \text{ in } G(x, y, z) =  x + y + z = 0 
\]
Definirajmo novo funkcijo $\Phi$, ki ju obe poveže:
\[
\Phi(x, y, z):\R^3 \rightarrow \R^2
\]
\[
(x, y, z) \mapsto (F(x, y, z), G(x, y, z))
\]
Poglejmo si odvod $D\Phi$:
\[
D\Phi(x, y, z) = 
\begin{bmatrix}
$2x$ & $2y$ & $2z$ \\
$1$ & $1$ & $1$
\end{bmatrix}
\]
\[
rang(D\Phi)(x, y, z) \leq 1:
\]
Velja:
\[
rang(D\Phi)(x, y, z) \neq 0
\]
, zaradi neničelne druge vrste matrike.
\[
rang(D\Phi)(x, y, z) \leq 1 \iff (2x, 2y, 2z) = \lambda \cdot (1, 1, 1)
\]
za nek $\lambda \in \R$. To pa velja natanko tedaj, ko
\[
x = y = z = \alpha
\]
Ker $x + y + z = 0 \Rightarrow \alpha = 0$.
Poleg tega pa velja:
\[
x^2 + y^2 + z^2 = 1 \Rightarrow 3 \cdot \alpha^2 = 1
\]
Pridemo v protislovje, saj $3 \cdot 0 \neq 1$
Torej ima ta preslikava rang $2$ v okolici $N$ (saj je $2$ edina preostala opcija).
\end{zgled}

\begin{zgled}
Poglejmo si še primer v množici $\R^{n^2} = \R^{n \times n}$, $n \times n$ realnih matrik.
\[
X \in \R^{n \times n}, F(X) = det(X) \in \con^\infty(\R^{n \times n})
\]
\[
det(X) \neq 0
\]
Množica
\[
\{ X \in \R^{n \times n}, det(X) > 0 \lor det(X) < 0 \} = F^{-1}(\R\setminus\{0\})
\]
je odprta v $\R^{n \times n}$. Poleg tega je tudi vredno omeniti, da je $\R\setminus\{0\}$ odprta v $\R$.

Poglejmo si splošno linearno grupo:
\[
GL_n(\R) = \{ X \in \R^{n \times n} \mid det(X) \neq 0\} = M
\]
Ta grupa je podmnogoterost dimenzije $n^2$.

Poglejmo še posebno linearno grupo:
\[
SL_n(\R) = \{ X \in \R^{n \times n} \mid det(X) = 1\}
\]
Ali je to podmnogoterost dimenzije $n^2 - 1$?

Definirajmo definicijsko funkcijo $F(X)$:
\[
F(X) - 1 = 0; F: \R^{n^2} \rightarrow \R
\]
Naj bo $X \in SL_n(\R)$ matrika oblike
\[
\begin{bmatrix}
x_{11} & x_{12} & \cdots & x_{1n} \\
x_{21} & \ddots & & \vdots \\
\vdots & & \ddots & \vdots \\
x_{n1} & \cdots & \cdots & x_{nn} 
\end{bmatrix}
\]
Kaj je $\frac{\partial F}{\partial x_{ij}}$?

Poglejmo primer ko je $n=2$:
\[
\begin{bmatrix}
x_{11} & x_{12} \\
x_{21} & x_{22}
\end{bmatrix}
\]
\[
F(X) = x_{11} \cdot x_{22} - x_{21} \cdot x_{12}
\]
\[
\frac{\partial F}{\partial x_{22}} = x_{11}
\]
\[
\frac{\partial F}{\partial x_{11}} = x_{22}
\]
\[
\frac{\partial F}{\partial x_{12}} = -x_{21}
\]
\[
\frac{\partial F}{\partial x_{21}} = -x_{12}
\]
Velja:
\[
rang(F) = 0 \lor X = 0
\]
Razpišimo zdaj formulo za splošni $X$ s pomočjo razvoja po prvi vrstici:
\[
F(X) = x_{11} \cdot \tilde{X}_{11} - x_{12} \cdot \tilde{X}_{12} + \ldots
\]
Torej
\[
\frac{\partial F}{\partial x_{ij}} =(-1)^{i+j} \cdot \tilde{X}_{ij}
\]
Velja:
\[
rang(F)(A) = 0 \Rightarrow det(A) = 0 \Rightarrow A \notin SL_n(\R)
\]
Dobimo protislovje iz katerega sledi:
\[
rang(F) = 1 \text{ v okolici } SL_n(\R)
\]
\end{zgled}

\begin{trditev}
Neprazna podmnožica $M \subseteq \R^n$ je podmnogoterost dimenzije $k$ ($0 < k < n$) natanko tedaj, ko za vsako točko $x_0 \in M$ obstaja okolica $U$ in permutacija koordinat $\sigma : (x_1, \ldots, x_n) \mapsto (x_{\sigma(1)}, \ldots, x_{\sigma(n)})$, ter odprta množica $D \subseteq \R^k$ , ki je okolica točke $(x^{0}_{\sigma(1)}, \ldots, x^{0}_{\sigma(k)})$, ter $\con^1$ preslikava $\varphi : D \subseteq \R^k \rightarrow \Omega \subseteq \R^{n-k} $, da je $U = D \times \Omega$ in $M \cap U = \{x \in U \mid (x_{\sigma(1)}, \ldots, x_{\sigma(k)}, \varphi(x_{\sigma(1)}, \ldots, x_{\sigma(k)})) = (x_{\sigma(1)}, \ldots, x_{\sigma(k)}, \varphi_1(x_{\sigma(1)}, \ldots, x_{\sigma(k)}), \varphi_2(x_{\sigma(1)}, \ldots, x_{\sigma(k)}), \ldots, \varphi_{n-k}(x_{\sigma(1)}, \ldots, x_{\sigma(k)}))\}$.
\begin{opomba}
Lokalno je $M$ graf preslikave iz $D \subseteq \R^k$ v $\Omega \subseteq \R^{n-k}$
\end{opomba}
\end{trditev}

\begin{proof}

Najprej se bomo lotili dokazovanja z leve proti desni:
V tem primeru dokazujemo posledico posledice izreka o implicitni preslikavi:
\[
M \cap \tilde{U} = \{ x \in \tilde{U} \mid F(x) = 0 \} ; F = (F_1, \ldots, f_{n-k}); rang(F) = n-k \iff rang(DF) = n-k
\]
To (baje) zadošča.

Dokažimo zdaj še v drugo smer:

Naj bo $\varphi = (\varphi_1, \ldots, \varphi_{n-k})$ in $M \cap U = \{ (x, \varphi(x)) \mid x = (x_1, \ldots, x_k) \in D \subseteq \R^k, \varphi : D \rightarrow \Omega \subseteq \R^{n-k}\}$
\[
F_1(x_1, \ldots, x_n) = x_{k+1} - \varphi_1(x_1, \ldots, x_k)
\]
\[
F_2(x_1, \ldots, x_n) = x_{k+2} - \varphi_2(x_1, \ldots, x_k)
\]
\[
\vdots
\]
\[
F_{n-k}(x_1, \ldots, x_n) = = x_n - \varphi_{n-k}(x_1, \ldots, x_k)
\]
Iz tega sledi:
\[
M \cap U = \{ x \in U \mid F_1 = \ldots = F_{n-k} = 0 \}
\]
\[
(DF)(x) =
\begin{bmatrix}
-D\varphi(x_1, \ldots, x_k) & I
\end{bmatrix}
\]
$rang(DF) = n-k$
\end{proof}

$F = 0$: implicitno podajanje podmnogoterosti
kot graf: eksplicitno podajanje podmnogoterosti

\begin{trditev}
Naj bo $M \subseteq \R^n$ neprazna množica. $M$ je podmnogoterost dimenzije $k$ $(0 < k < n)$ natanko tedaj, ko za vsak $x_0 \in M$ obstaja taka odprta okolica $U$ točke $x_0$ v $\R^n$ in difeomorfizem $\Phi: U \subseteq \R^n \rightarrow V \subseteq \R^n$, da velja, da je $\Phi(M \cap U) = V \cap (\R^k \times \{0\}^{n-k})$ Pravimo tudi, da smo M lokalno izravnali.
\end{trditev}

\begin{proof}

Najprej dokažimo iz leve proti desni:
Naj bo $M$ podmnogoterost v $\R^n$. Potem je lokanlo graf nad $D \subseteq \R^k$ in $M \cap U = \{ (x, \varphi(x)) \mid x \in D\} \in \R^k\times\R^{n-k}$. Naj bo
\[
\Phi(x, y) = (x, y - \varphi(x)) = (x_1, \ldots, x_k, y_1 - \varphi_1(x), \ldots, y_{n-k} - \varphi_{n-k}(x))
\]
potem je inverz:
\[
\Phi^{-1}(x, z) = (x, z + \varphi(x))
\]
Poglejmo odvod:
\[
D\Phi(x, y) = 
\begin{bmatrix}
I & 0 \\
-D\varphi & I
\end{bmatrix}
, det(D\Phi) \neq 0
\]
velja:
\[
\Phi(M \cap U) = \{ (x, 0) \mid x \in D \subseteq \R^k \}
\]
To (baje) zadošča.
Dokažimo zdaj še v drugo smer:
\[
\Phi : U \rightarrow V \text{ je difeomorfizem.}
\]
\[
\Phi = (\Phi_1, \ldots, \Phi_n); ~rang(D\Phi) = n \text{ v vsaki točki.}
\]
Velja:
\[
\Phi(U \cap M) = V \cap (\R^k \times \{0\}^{n-k}) \iff 
\Phi_{k+1} = \ldots = \Phi_n = 0 \text{ na } M \cap U
\]
Imamo torej $n-k$ funkcij, kjer je $rang(\Phi_{k+1}, \ldots, \Phi_n) = n-k$, saj je $\Phi$ difeomorfizem.
\end{proof}

\begin{trditev}
Parametrično podajanje podmnogoterosti:

Naj bo $M \neq \emptyset \subseteq \R^n$. $M$ je gladka podmnogoterost dimenzije $k (0 \leq k \leq n)$, če za vsako točko $x_0 \in M$ obstaja taka okolica $U$ točke $x_0$, ter odprta množica $\Omega \subseteq \R^k$ in injekcija $\Psi: \Omega \subseteq \R^k \xrightarrow{\con^1} \R^n$ ranga $k$, da velja:
\[
\Psi_{\ast}(\Omega) = M \cap U
\]

\end{trditev}

\begin{proof}
Najprej bomo dokazali implikacijo v desno $(\Rightarrow)$:

$M$ je lokalno graf, torej $M \cap U = \{ (x',\varphi(x')) \mid x' \in D \subseteq \R^k; \varphi: D \xrightarrow{\con^1} V\}$.
Določimo preslikavo $\Psi(x') = (x', \varphi(x')); \Psi: D \rightarrow M \cap U$ in poglejmo njen rang:
\[
D\Psi = \begin{bmatrix}
I \\
D\varphi
\end{bmatrix}
, rang(D\Psi) = rang(\Psi) = k
\]
S tem se konča dokaz desne implikacije.

Poglejmo zdaj še dokaz implikacije v levo $(\Leftarrow)$:

Naj bo $\Psi: \Omega \subseteq \R^k \xrightarrow{\con^1} \R^n$ preslikava ranga $k$ in naj bo $\Psi(t_0) = x_0$ ter $\Psi_{\ast}(\Omega) = M \cap U$. $\Psi(t) = (\Psi_1(t), \ldots, \Psi_n(t))$ in $rang(D\Psi) = k$.

Poglejmo si $D\Psi$:
\[
D\Psi = \begin{bmatrix}
D\Psi_1 \\
\vdots \\
D\Psi_n
\end{bmatrix}
k \times n \text{~matrika}
\]

Naredimo permutacijo koordinat v $\R^n$, da v okolici točke $t_0$ velja:

\[
D\Psi = \begin{bmatrix}
D\Psi_1 \\
\vdots \\
D\Psi_k \\
D\Psi_{k + 1} \\
\vdots \\
D\Psi_n
\end{bmatrix}(t)
, \text{~da je matrika} \begin{bmatrix}
D\Psi_1 \\
\vdots \\
D\Psi_k
\end{bmatrix}(t)
\text{~obrnljiva.}
\]

Zapišemo $\Psi(t) = (\tilde{\Psi}_1(t), \tilde{\Psi}_2(t)) = ((\Psi_1, \ldots, \Psi_k), (\Psi_{k + 1}, \ldots, \Psi_n))(t)$
\[
t \mapsto \tilde{\Psi}_1(t) \text{~v okolici~} t_0
\]

Odvod je obrnljiv, torej je v okolici točke $t_0$ difeomorfizem na sliko ($Im(\tilde{\Psi}_1)$).

$\tilde{\Psi}_1(t_0)=\leftidx{^1}{x_0}$ ima $\con^1$ inverz. Torej:
\[
\tilde{\Psi}_1(t) = \leftidx{^1}{x} \Rightarrow t = \tilde{\Psi}^{-1}_1(\leftidx{^1}{x})
\]
$\Psi$ v okolici $t_0$ je torej:
\[
\Psi(t) = (\tilde{\Psi}_1(t), \tilde{\Psi}_2(t)) = (\leftidx{^1}{x}, \tilde{\Psi}_2(\tilde{\Psi}^{-1}_1(\leftidx{^1}{x})) = (\leftidx{^1}{x}, \varphi(\leftidx{^1}{x}))
\]

$\varphi$ je $\con^1$ preslikava. Lokalno gledano je $M$ graf preslikave $\varphi: D \subseteq \R^k \rightarrow \R^{n - k}$

\end{proof}

\subsection{Tangentni prostori na podmnogoterosti}

Če gledamo $\gamma: (\alpha, \beta) = I^{interval} \subseteq \R \rightarrow \R^n; \gamma = (\gamma_1, \ldots, \gamma_n)$ je to $\con^1$ preslikava. $\gamma$ je >>pot<< v $\R^n$
\[
\dot{\gamma}(t) = (\dot{\gamma}_1(t), \ldots, \dot{\gamma}_n(t));\text{~pri čemer je~} \dot{\gamma}(t) = \frac{\partial\gamma}{\partial t} \text{~odvod po času}
\]

\begin{definicija}

Naj bo $M \subseteq \R^n$ podmnogoterost in naj bo $x_0 \in M$. \pojem{Tangentni prostor} na $M$ v točko $x_0$ je:
\[
T_{x_0}(M) = \{\dot{\gamma}(t_0)\mid \gamma : I \xrightarrow{\con^1} M \subseteq \R^n\ \text{~je pot in~} \gamma(t_0) = x_0\}
\]
\end{definicija}

\begin{trditev}

Naj bo $M$ podmnogoterost dimenzije $k$ v $\R^n$ in naj bo $x_0 \in M$ ter naj bodo $F = (F_1, \ldots, F_{n - k})$ definicijske funkcije $M$ v okolici $x_0$ (Pomnimo: $M \cap U = \{x_0 \in U \mid F(x_0) = 0\}, rang(F) = n - k $)
Tedaj $T_{x_0}(M) = Ker(DF(x_0))$

\end{trditev}

\begin{proof}

Lokalno je $M \cap U = \{ (x, \varphi(x))\mid \varphi: D\subseteq \R^k \xrightarrow{\con^1} \R^{n - k}\}$, torej je $F((x, \varphi(x))) = 0$ za vsak $x\in D$.

Odvajamo:
\[
DF_x((x, \varphi(x))) \cdot I + DF_y((x, \varphi(x))) \cdot D\varphi = 0
\]
\[
\underbrace{\begin{bmatrix}
DF_x & DF_y
\end{bmatrix}}_{=DF}
\cdot
\begin{bmatrix}
I \\
D\varphi
\end{bmatrix} 
= 0 \text{,~torej~} DF \cdot
\begin{bmatrix}
I \\
D\varphi
\end{bmatrix} = 0 ~(\text{v točki~} (x, \varphi(x)))
\]
\[
DF(x_0, y_0) \cdot
\begin{bmatrix}
I \\
D\varphi(x_0)
\end{bmatrix} = 0 
\Rightarrow
\underbrace{Im\bigg(\begin{bmatrix}
I \\
D\varphi(x_0)
\end{bmatrix}\bigg)}_{T_{(x_0, y_0)}(M)} \subseteq Ker(DF(x_0, y_0))
\]
Sledi:
\[
Dim(Im\bigg(\begin{bmatrix}
I \\
D\varphi(x_0)
\end{bmatrix}\bigg)) = k = Dim(Ker(DF(x_0, y_0))) \Rightarrow \text{sta enaka}
\]
\end{proof}

$M$ podana implicitno:
\[
T_{x_0}(M) = Ker(DF(x_0))
\]

\begin{trditev}
Naj bo $\Psi: \Omega \subseteq \R^k \rightarrow \R^n$ lokalna parametrizacija $M$ v okolici $x_0 \in M$, ranga $k$ in naj velja $\Psi_{\ast}(\Omega) = M \cap U$. Tedaj je $T_{x_0}(M) = Im(D\Psi(t_0))$, kjer je $\Psi(t_0) = x_0$.
\end{trditev}

\begin{proof}

Lahko predpostavimo, da je $\Psi = (\tilde{\Psi}_1, \tilde{\Psi}_2)$, kjer je $D\tilde{\Psi}_1(t_0)$ v okolici $t_0$ obrnljiva (Linearna sprememba koordinat). Tedaj je $M \cap U$ graf preslikave $\varphi = \tilde{\Psi}_2(\tilde{\Psi}^{-1}_1)$.
Torej je:
\[
T_{(x_0, y_0)}(M) = T_{(\leftidx{^1}{x_0})}(M) = Im\bigg( \begin{bmatrix}
I \\
D(\tilde{\Psi}_2 \circ \tilde{\Psi}^{-1}_1)(\leftidx{^1}{x_0})
\end{bmatrix}\bigg) = Im\bigg( \begin{bmatrix}
I \\
D\tilde{\Psi}_2(t_0) \cdot (D\tilde{\Psi}^{-1}_1)(\leftidx{^1}{x_0})
\end{bmatrix}\bigg) 
\]
\[
= Im\bigg( \begin{bmatrix}
I \\
D\tilde{\Psi}_2(t_0) \cdot (D\tilde{\Psi}^{-1}_1)(t_0)
\end{bmatrix}\bigg) = Im\bigg( \begin{bmatrix}
(D\tilde{\Psi}_1)(t_0) \\
(D\tilde{\Psi}_2)(t_0)
\end{bmatrix}\bigg) \cdot (D\tilde{\Psi}^{-1}_1)(t_0)
\]
\[
= Im\bigg(\begin{bmatrix}
(D\tilde{\Psi}_1)(t_0) \\
(D\tilde{\Psi}_2)(t_0)
\end{bmatrix}\bigg) = Im(D\Phi(t_0))	
\]
\end{proof}

\begin{zgled}
\[
M = \{ (x, y, z) \in \R^3 \mid x^2 + y^2 + z^2 = 1\}; F(x, y, z) = x^2 + y^2 + z^2 - 1
\]
\[
DF(x, y, z) = \begin{bmatrix}
2x & 2y & 2z
\end{bmatrix}
\]
\[
T_{(x, y, z)}(M) = \{(X, Y, Z) \mid \langle (X, Y, Z), (x, y, z) \rangle = 0 \} = \{ (X, Y, Z) \mid x \cdot X + y \cdot Y + z \cdot Z = 0\}
\]
Za $x = y = \frac{1}{\sqrt{6}}$ in $z = - \frac{2}{\sqrt{6}}$:
\[
T_{(\frac{1}{\sqrt{6}}, \frac{1}{\sqrt{6}}, -\frac{2}{\sqrt{6}})}(M) = \{ (X, Y, Z) \mid \frac{(X + Y - 2Z)}{\sqrt{6}} = 0 \} = \{ (X, Y, \frac{X + Y}{2}) \mid X, Y \in \R^2 \}
\]
\end{zgled}

\end{document}
