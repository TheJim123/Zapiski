\documentclass[a4paper, 10pt]{article}
\usepackage[T1]{fontenc}
\usepackage[utf8]{inputenc}
\usepackage[slovene]{babel}
\usepackage{lmodern}
\usepackage{amsmath}
\usepackage{leftidx}
\usepackage{amssymb}
\usepackage{amsfonts}
\usepackage{graphicx}
\usepackage{wrapfig}
\usepackage{amsthm}
\usepackage{mathrsfs}
\usepackage{mathtools}
\usepackage{url}
\usepackage{subfigure}
\usepackage{multirow}
\usepackage{lipsum}
\usepackage{wrapfig}
\usepackage{tikz}
\usepackage[format=plain, font=small, labelfont=bf, textfont=it, justification=centerlast]{caption}
\usepackage{booktabs}
\usepackage{siunitx}
\usepackage{enumerate}
\usepackage{ulem}
\usepackage{cancel}
\usepackage{algorithm2e}

\newtheorem{lema}{Lema}
\newtheorem{trditev}{Trditev}
\newtheorem{izr}{Izrek}
\newtheorem{nal}{Naloga}
\newtheorem{posl}{Posledica}

\newcounter{defcount}
\newcounter{opombe}
\newcounter{zgledcount}

\newenvironment{opomba}{\begin{flushleft}\stepcounter{opombe}\textbf{Opomba \arabic{opombe}:}}{\hfill\end{flushleft}}
\setlength{\parindent}{0mm}

\newenvironment{zgled}{\begin{flushleft}\stepcounter{zgledcount}\textbf{Zgled \arabic{zgledcount}:}}{\hfill\end{flushleft}}
\setlength{\parindent}{0mm}

\newenvironment{definicija}{\begin{flushleft}\stepcounter{defcount}\textbf{Definicija \arabic{defcount}:}}{\hfill\end{flushleft}}
\setlength{\parindent}{0mm}

\newenvironment{resitev}{\begin{flushleft}\textit{Rešitev:}}{\hfill\end{flushleft}}
\setlength{\parindent}{0mm}

\newcommand{\ul}[1]{\underline{\ensuremath{#1}}}
\newcommand{\abs}[1]{\ensuremath{\lvert #1 \rvert}}
\newcommand{\mth}[1]{\ensuremath{\mathbb{#1}}}
\newcommand{\R}{\mth{R}}
\newcommand{\Z}{\mth{Z}}
\newcommand{\N}{\mth{N}}
\newcommand{\No}{\mth{N}_0}
\newcommand{\C}{\mth{C}}
\newcommand{\Cc}{\ensuremath{\mathcal{C}}}
\newcommand{\Dd}{\ensuremath{\mathcal{D}}}
\newcommand{\Mu}{\mathcal{M}}
\newcommand{\sigalg}{$\sigma$-algebra~}
\newcommand{\Qu}{\mth{Q}_u}

\newcommand{\pojem}[1]{\emph{#1}}
\newcommand{\Ob}[1]{\mathcal{O}b({#1})}
\newcommand{\Mor}[2][ ]{\mathcal{M}or_{#1}({#2})}
\newcommand{\con}{\ensuremath{\mathscr{C}}}
\newcommand{\padex}[2]{\ensuremath{{#1}^{\underline{#2}}}}
\newcommand{\rastx}[2]{\ensuremath{{#1}^{\bar{#2}}}}
\newcommand{\map}[3]{\ensuremath{{#1}: {#2} \rightarrow {#3}}}
\newcommand{\pra}[3]{{#1}{\ast}({#2}) = {#3}}

\newcommand{\Pot}[1]{\mathcal{P}({#1})}
\newcommand{\set}[1]{\ensuremath{\{1, 2, \ldots , #1\}}}
\newcommand{\seto}[1]{\ensuremath{\{0, 1, \ldots , #1\}}}

\newcommand{\mn}[1]{\ensuremath{\left[#1\right]}}

\title{Izbrana poglavja iz topologije \\ Zapiski predavanj}
\date{2023/24}
%===============================================================================
\begin{document}
	\maketitle
	\newpage
	\begin{abstract}
		\noindent Dokument vsebuje zapiske predavanj predmeta Izbrana poglavja iz topologije v okviru študija prvega letnika magistrskega študija matematike na FNM.
	\end{abstract}
	\tableofcontents
	\newpage
	\section{Metrizabilnost topoloških prostorov}
	Za začetek se spomnimo nekaj znanih pojmov s področja topologije. Naj bo $X$ neprazna množica: \begin{itemize}
		\item Pravimo, da je topološki prostor $(X, \mathcal{T})$ \pojem{metrizabilen}, če obstaja metrika $\map{d}{X}{\R}$, taka, da je $\mathcal{T}_d = \mathcal{T}$, kjer je $\mathcal{T}_d$ topologija na $X$, ki ima za bazo $\mathcal{B}_d = \{K_d(x, r);~x\in X ~\&~r>0\}$.
		\item Naj bosta $\mathcal{T}_1$ in $\mathcal{T}_2$ topologiji na $X$ in $\mathcal{B}_1$ ter $\mathcal{B}_2$ bazi za $\mathcal{T}_1$ in $\mathcal{T}_2$. Tedaj je $\mathcal{T}_1 \subseteq \mathcal{T}_2$, če je $\mathcal{B}_1 \subseteq \mathcal{T}_2$
		\item Naj bosta $d_1$ in $d_2$ metriki na $X$. Pravimo, da sta $d_1$ in $d_2$ \pojem{ekvivalentni}, če je $\mathcal{T}_{d_1} = \mathcal{T}_{d_2}$
	\end{itemize}
	
	\begin{lema}
		Naj bo $X_d$ matrični prostor in naj bo $\map{\rho}{X\times X}{\R}$ definirana s predpisom $$\rho(x, y) = \frac{d(x, y)}{1 + d(x, y)} $$. Tedaj velja: \begin{enumerate}
			\item $\rho$ je metrika na $X$ za katero je $\rho(x, y) \in [0, 1)\forall x, y\in X$
			\item $\mathcal{T}_d = \mathcal{T}_\rho$
		\end{enumerate}
	\end{lema}
	\begin{proof}
		\begin{enumerate}
			\item Glede na definicijo $\rho$ je očitno, da je $\rho(x, y) < 1$. Preverimo torej, da je $\rho$ metrika na $X$. \begin{enumerate}[i)]
				\item Naj bosta $x, y\in X$ poljubna elementa. Vemo, da je $d(x, y) \geq 0$ ker je $d$ metrika na $X$). Potem je pa tudi $\rho(x, y) \geq 0$. Posledično je torej $\rho(x, y)\in [0, 1) \forall x, y \in X$.
				\item Velja: $\rho(x, y) = 0 \iff d(x, y) = 0 \iff x = y$.
				\item Velja: $\forall x, y \in X: \rho(x, y) = \frac{d(x, y)}{1 + d(x, y)} = \frac{d(y, x)}{1 + d(y, x)} = \rho(y, x)$.
				\item  Naj bodo $x, y, z \in X$ poljubni. Preverimo, ali velja $\rho(x, y) \leq \rho(x, z)+ \rho(z, y)$: \begin{equation*}
					\rho(x, y) \leq \rho(x, z) + \rho(z, y) \iff \frac{d(x, y)}{1 + d(x, y)} \leq \frac{d(x, z)}{1 + d(x, z)} + \frac{d(z, y)}{1 + d(z, y)}
				\end{equation*}
				Z množenjem se znebimo ulomkov in tako dobimo neenakost:
				\begin{align*}
					d(x, y)(1 + d(x, z))(1 + d(z, y)) &\leq d(x, z)(1 + d(x, y))(1 + d(z, y)) ~+ \\ &+ d(z, y)(1 + d(x, z))(1 + d(x, y)) 	
				\end{align*}
				Da bo zapis krajši označimo $d(x, y) = d_{x, y}, d(x, z) = d_{x, z}$ in $d(z, y) = d_{z, y}$. Ko poračunamo faktorje na obeh straneh neenačaja dobimo:
				\begin{align*}
					d_{x, y} &+ d_{x, y}d_{x, z} + d_{x, y}d_{z, y} + d_{x, y}d_{x, z}d_{z, y} \leq \\
					&\leq d_{x, z}d_{x, y} + d_{x, z} + d_{x, z}d_{z, y} + d_{x, y}d_{x, z}d_{z, y} + \\
					&+ d_{z, y} + d_{z, y}d_{x, z} + d_{z, y}d_{x, y} + d_{x, y}d_{x, z}d_{z, y}
				\end{align*}
				Pokrajšamo skupne člene in tako dobimo:
				\begin{equation*}
					d_{x, y} \leq d_{x, z} + d_{x, z}d_{z, y} + d_{z, y} + d_{z, y}d_{x, z} + d_{x, y}d_{x, z}d_{z, y}
				\end{equation*}
				Ta pogoj pa je izpolnjen, ker je $d(x, y)\leq d(x, z) + d(z, y)$.
			\end{enumerate}
			Sledi, da je $\rho$ res metrika na $X$.
			\item Naj bo $x\in X$ in $r>0$. Označimo $K_d(x, r) = \{y\in X;~ d(x, y) < r\}$. Ker je $\rho(x, y) = \frac{d(x, y)}{1 + d(x, y)}$, je potem $\rho(x, y)(1 + d(x, y)) = d(x, y)$ oz. $d(x, y) = \frac{\rho(x, y)}{1 - \rho(x, y)}$. Ko to vstavimo v definicijo $K_d(x, r)$ dobimo: $$K_d(x, r) = \{y\in X;~\frac{\rho(x, y)}{1 - \rho(x, y)}< r\} = \{y\in X;~\rho(x, y)< r - r\rho(x, y)\}$$ Pogoj $\rho(x, y) < r - r\rho(x, y)$ je pa ekvivalenten pogoju $\rho(x, y)(1 + r) < r$ oziroma $\rho(x, y) < \frac{r}{1 + r}$. Potem pa velja: $$K_d(x, r) =\{y\in X;~ \rho(x, y) < \frac{r}{1 + r}\} = K_\rho(x, \frac{r}{1 + r})$$
			Iz zgornje enakosti potem sledi, da je $K_d(x, r) \in \mathcal{T}_\rho$, torej je $\mathcal{T}_d \subseteq \mathcal{T}_\rho$. Na podoben način lahko vidimo tudi, da je $K_\rho(x, r) = K_d(x, \frac{r}{1 - r})$, torej je $\mathcal{T}_\rho \subseteq \mathcal{T}_d$. Sledi, da je $\mathcal{T}_d = \mathcal{T}_\rho$.
		\end{enumerate}
	\end{proof}
	\begin{opomba}
		Naj bo $(X, \mathcal{T})$ metrizabilen prostor. Tedaj lahko brez škode za splošnost predpostavimo, da je metrika $d$ na $X$ taka, da je $\mathcal{T}_d = \mathcal{T}$ in $\forall x, y\in X: d(x, y) < 1$.
	\end{opomba}
	\begin{izr}
		\label{izr:prodmetr}
		Naj bo $\forall n\in\N (X_n, d_n)$ metrični prostor, tak, da je $\forall x, y \in X_n d_n(x, y) < 1$. Tedaj je $\map{D}{\prod_{n = 1}^{\infty}X_n\times\prod_{n = 1}^{\infty}X_n}{\R}$, ki je definirana s predpisom $D((x_1, x_2, \ldots), (y_1, y_2, \ldots)) = \max\{\frac{d_n(x_n, y_n)}{2^n;~n\in\N}\}$ za poljubne $(x_1, x_2, \ldots),(y_1, y_2, \ldots)\in \prod_{n = 1}^{\infty}X_n$ metrika na $\prod_{n = 1}^{\infty}X_n$.
	\end{izr}
	\begin{proof}
		Preverili bomo najprej, da je $D$ dobro definirana, nato pa še, da je metrika. Pri tem bomo uporabili zapis $X = \prod_{n = 1}^{\infty}X_n$ in za poljuben $(x_1, x_2, \ldots) \in X : \underline{x} = (x_1, x_2, \ldots)$.
		\begin{itemize}
			\item Naj bosta $\ul{y}$ in $\ul{y}$ poljubna elementa $X$. Obravnavamo možnosti: \begin{enumerate}[i)]
				\item Denimo, da je $\ul{x} = \ul{y}$ in označimo $A =\{\frac{d_n(x_n, y_n)}{2^n};~n\in\N\}$. Po predpostavki za vsak $n\in\N$ velja, da je $x_n = y_n$. Potem je pa $A = \{0\}$ in $D(\ul{x}, \ul{y})=\max(A) = 0$.
				\item Naj bosta sedaj $\ul{x}$ in $\ul{y}$ različna. Potem $\exists n_0\in \N$, da je $x_{n_0}\neq y_{n_0}$ in označimo $r_0 = \frac{d_{n_0}(x_{n_0}, y_{n_0})}{2^{n_0}} > 0$. Ker je $A$ neprazna množica in v $\R$ navzgor omejena z $1$, obstaja $\sup(A)$. Opazimo tudi, da je zaporedje $\frac{d_n(x_n, y_n)}{2^n}$ navzdol omejeno z $0$ in navzgor omejeno z zaporedjem $\frac{1}{2^n}$. Ker je $\lim_{n\to\infty}\frac{1}{2^n} = 0$ po izreku o sendviču potem velja, da je $\lim_{n\to\infty}\frac{d_n(x_n, y_n)}{2^n} = 0$. Naj bo $n_1\in \N$ tak, da $\forall n\in\N; n\geq n_1: \frac{d_n(x_n, y_n)}{2^n} < r_0$. Dodatno, ker je (očitno) $\sup(A) \geq r_0$, je $\sup(A) = \sup\{\frac{d_n(x_n, y_n)}{2^n};~n\in \set{n_1}\} = \max\{\frac{d_n(x_n, y_n)}{2^n};~n\in\set{n_1}\}=\max(A)$. Drugače povedano, za vsak par $\ul{x}, \ul{y}\in X$ za pripadajočo množico $A$ obstaja maksimum, torej je $D$ dobro definirana.
				\item Preverimo še, da je $D$ res metrika. \begin{enumerate}[a)]
					\item Za poljubna $\ul{x}, \ul{y}\in X$ je $D(\ul{x}, \ul{y}) \geq 0$ in $D(\ul{x}, \ul{y})\iff \ul{x} = \ul{y}$
					\item Za poljubna $\ul{x}, \ul{y}\in X$ je $D(\ul{x}, \ul{y}) = D(\ul{y}, \ul{x})$
					\item Premislimo, da velja trikotniška neenakost:
					Naj bodo $\ul{x}, \ul{y}, \ul{z}\in X$ poljubni elementi. Potem $\exists n_0 \in \N$, tak, da velja: \begin{align*}
						D(\ul{x}, \ul{y}) &\leq \max\{\frac{d_n(x_n, z_n) + d_n(z_n, y_n)}{2^n};~n\in\N\} \\ &=\frac{d_{n_0}(x_{n_0}, z_{n_0}) + d_{n_0}(z_{n_0}, y_{n_0})}{2^{n_0}} \\ 
						&= \frac{d_{n_0}(x_{n_0}, z_{n_0})}{2^{n_0}} + \frac{d_{n_0}(z_{n_0}, y_{n_0})}{2^{n_0}} \\
						&\leq \max\{\frac{d_n(x_n, z_n)}{2^n}; n\in\N\} + \max\{\frac{d_n(z_n, y_n)}{2^n}; n\in\N\} \\
						&= D(\ul{x}, \ul{z}) + D(\ul{z}, \ul{y})
					\end{align*} 
				\end{enumerate}
			\end{enumerate}
		\end{itemize}
	\end{proof}
	\begin{definicija}
		Prostor $(\prod_{n = 1}^{\infty}X_n, D)$, kjer je $D$ definirana kot v izreku \ref{izr:prodmetr}, pravimo \pojem{produkt metričnih prostorov} $(X_1, d_1), (X_2, d_2), \ldots$. Metriki $D$ pravimo \pojem{produktna metrika} na $\prod_{n = 1}^{\infty}X_n$, ki je dobljena iz metrik $d_1, d_2, \ldots$.
	\end{definicija}
	\begin{definicija}
		Naj bo $\Lambda$ poljubna indeksna množica in naj bo za $\forall\lambda\in\Lambda (X_\lambda, \mathcal{T}_\lambda)$ topološki prostor. Topologiji $\mathcal{U}$ na $X =\prod_{\lambda\in\Lambda}X_\lambda$ pravimo \pojem{produktna topologija} na $X$, dobljena iz topologij $\{\mathcal{T}_\lambda\}_{\lambda\in\Lambda}$, če je $\mathcal{P} = \{p^{-1}_\lambda(U_\lambda);~ \lambda\in\Lambda ~\&~U_\lambda \in \mathcal{T}_\lambda\}$ njena podbaza. Pri tem je $\forall\lambda\in\Lambda \map{p_\lambda}{X}{X_\lambda}$ projekcija na $\lambda$-ti faktor.
	\end{definicija}
	\begin{opomba}
		Seveda obstaja natanko ena topologija $\mathcal{U}$ na $\prod_{\lambda\in\Lambda}$, ki ima $\mathcal{P}$ za podbazo.
	\end{opomba}
	Naj bo $B\in \mathcal{B}$ bazna množica za produktno topologijo $\mathcal{U}$. Potem obstajajo taki indeksi $\lambda_1, \lambda_2, \ldots, \lambda_n \in \Lambda$, da je $B = p^{-1}_{\lambda_1}(U_{\lambda_1})\cap p^{-1}_{\lambda_2}(U_{\lambda_2})\cap \ldots \cap p^{-1}_{\lambda_n}(U_{\lambda_n})$ za neke množice $U_{\lambda_1}\in \mathcal{T}_{\lambda_1}, U_{\lambda_2}\in \mathcal{T}_{\lambda_2}, \ldots, U_{\lambda_n}\in \mathcal{T}_{\lambda_n}$. Z določeno mero zlorabe notacije potem sledi: \begin{align*} B &= \left(U_{\lambda_1}\times \prod_{\lambda\in\Lambda\setminus\{\lambda_1\}}X_\lambda\right)\cap \left(U_{\lambda_2}\times \prod_{\lambda\in\Lambda\setminus\{\lambda_2\}}X_\lambda\right)\cap \ldots \cap \left(U_{\lambda_n}\times \prod_{\lambda\in\Lambda\setminus\{\lambda_n\}}X_\lambda\right) \\
		&=\left(U_{\lambda_1}\times U_{\lambda_2}\times\ldots\times U_{\lambda_n}\right) \times \prod_{\lambda\in\Lambda\setminus\{\lambda_1, \lambda_2, \ldots, \lambda_n\}}X_\lambda
	\end{align*}
	To nam bo pomagalo pri dokazu naslednjega izreka.
	\begin{izr}
		Za poljuben $n\in\N$ naj bo $(X_n, d_n)$ metrični prostor in naj bo $D$ produktna metrika na $X = \prod_{n = 1}^\infty X_n$, dobljena iz $d_1, d_2, \ldots$. Naj bo $\mathcal{U}$ produktna topologija na $X$, ki je dobljena iz topologij $\mathcal{T}_{d_1}, \mathcal{T}_{d_2}, \ldots$. Tedaj je $\mathcal{U} = \mathcal{T}_D$.
	\end{izr}
	
	\begin{proof}
		Pokazali bomo, da obe topologiji vsebujeta drugo. \begin{itemize}
			\item[$\mathcal{U} \supseteq \mathcal{T}_D):$] Baza topologije $\mathcal{T}_D$ je $\mathcal{B}_D = \{K_D(\ul{x}, r);~ \ul{x}\in X ~\&~ r>0\}$. Dovolj bo, če pokažemo, da je $\mathcal{B}_D \subseteq \mathcal{U}$. To bomo storili tako, da za vsako kroglo $K_D(\ul{x}, r)\in \mathcal{B}_D$ in vsak element $\ul{y}\in K_D(\ul{x}, r)$ najdemo okolico $U_{\ul{y}}\in \mathcal{U}$, da bo $K_D(\ul{x}, r) = \bigcup_{\ul{y}\in K_D(\ul{x}, r)}U_{\ul{y}}$.
			
			Naj bo torej $\ul{x}$ poljubna točka iz $X$ in naj bo $r>0$. Naj bo $\ul{y}\in K_D(\ul{x}, r)$ in označimo $r_0 = \frac{r - D(\ul{x}, \ul{y})}{2} > 0$. Ker je $\lim_{n\to\infty}\frac{1}{2^n} = 0$ obstaja nek $n_0\in\N$, da bo $\forall n\geq n_0 \frac{1}{2^n} < r_0$. Označimo: $$U_{\ul{y}} = K_{d_1}(y_1, \frac{r_0}{2})\times K_{d_2}(y_2, \frac{r_0}{2})\times\ldots K_{d_{n_0}}(y_{n_0}, \frac{r_0}{2})\times\prod_{k = n_0 + 1}^\infty X_k$$
			Očitno je $\ul{y}\in U_{\ul{y}}$. Naj bo sedaj $\ul{z}\in U_{\ul{y}}$. Vemo, da je za vsak $n\in\N$ $d_n(z_n, y_n) < \frac{r_0}{2}$. Potem je pa: \begin{align*}
				D(\ul{x}, \ul{z}) &= \max\{\frac{d_n(x_n, z_n)}{2^n};~n\in\N\} \\
				&= \max\left(\{\frac{d_1(x_1, z_1)}{2}, \ldots, \frac{d_{n_0}(x_{n_0}, z_{n_0})}{2^{n_0}}\}\cup\{\frac{d_n(x_n, z_n)}{2^n};~n> n_0\}\right) \\
				&\leq \max\left(\{\frac{d_n(x_n, y_n) + d_n(y_n, z_n)}{2^n};~n\in \set{n_0}\}\cup\{\frac{1}{2^n}\}^{\infty}_{n = n_0 + 1}\right)
			\end{align*}
			Vemo, da je $d_i(x_i, z_i) < \frac{r_0}{2}$ za vse $i\in \set{n_0}$ in iz prejšnje definicije $r_0$ izrazimo $D(\ul{x}, \ul{y}) = r - 2r_0$. Potem je pa $\frac{d_i(x_i, y_i)}{2^i} \leq D(\ul{x}, \ul{y})= r - 2r_0$. Za vask $i\in \set{n_0}$ sledi ocena: \begin{align*}
				\frac{d_i(x_i, y_i)}{2^i} &\leq \frac{d_i(x_i, z_i)}{2^i} + \frac{d_i(z_i, y_i)}{2^i} < r - 2r_0 + \frac{r_0}{2^{i+1}} \\ &< r - 2r_0 + r_0 < r- r_0 < r
			\end{align*}
			Posledično je $D(\ul{x}, \ul{z}) < r$, torej je $\ul{z}\in K_D(\ul{x}, r)$. Ker to velja za vsak $\ul{z} \in U_{\ul{y}}$, je potem $U_{\ul{y}} \subseteq K_D(\ul{x}, r)$. Ker je to res za vsak $\ul{y}\in K_D(\ul{x}, r)$, je potem res $K_D(\ul{x}, r)=\bigcup_{\ul{y}\in K_D(\ul{x}, r)}U_{\ul{y}}$, torej je $\mathcal{T}_D \subseteq \mathcal{U}$.
			\item[$\mathcal{U} \subseteq \mathcal{T}_D):$] Naj bo $B = U_1\times U_2\times\ldots\times U_n \times \prod_{k = 1}^\infty X_{n+k}$ bazna množica $\mathcal{U}$ in naj bo $\ul{x}\in B$. Potem je $x_1\in U_1, x_2\in U_2,\ldots, x_n\in U_n$. Za vsak $i\in\set{n}$ je torej $U_i\in \mathcal{T}_i$, torej $\exists r_i; K_{d_i}(x_i, r_i)\subseteq U_i$. 
			
			Določimo $r = \min\{\frac{r_i}{2^i};~ i\in\set{n}\}$ in naj bo $\ul{y}\in K_D(\ul{x}, r)$. Potem je $D(\ul{y}, \ul{x}) = \max\{\frac{d_i(y_i, x_i)}{2^i};~i\in\N\} < r$. Obstaja nek $i_0\in\N$, da je $D(\ul{y}, \ul{x}) = \frac{d_{i_0}(y_{i_0}, x_{i_0})}{2^{i_0}} = r_0$. Vemo že, da je $r_0 < r = \min\{\frac{r_i}{2^i};~i\in\set{n}\}$. Potem pa za vsak $i\in \set{n}$ velja $\frac{d_i(y_i, x_i)}{2^i} \leq r_0 < r \leq \frac{r_i}{2^i}$, torej je $\forall i \in \set{n}$ $d_i(y_i, x_i) < r_i$ oz. $y_i\in K_{d_i}(x_i, r_i)$. Potem je pa $\ul{y}\in B$. Ker to velja, sledi, da je $K_D(\ul{x}, r) \subseteq B ~\forall \ul{x}\in B$ in posledično je $B = \bigcup_{\ul{x}\in B} K_D(\ul{x}, r_{\ul{x}}) \in \mathcal{T}_D$, torej je $\mathcal{U} \subseteq \mathcal{T}_D$.
		\end{itemize}
	\end{proof}
\end{document}