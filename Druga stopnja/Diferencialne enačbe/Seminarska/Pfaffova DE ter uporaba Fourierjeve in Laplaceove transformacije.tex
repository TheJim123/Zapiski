\documentclass[a4paper, 10pt]{article}
\usepackage[T1]{fontenc}
\usepackage[utf8]{inputenc}
\usepackage[slovene]{babel}
\usepackage{csquotes}
\usepackage{lmodern}
\usepackage{amsmath}
\usepackage{leftidx}
%\usepackage[backend=biber, style=numeric]{biblatex}
\usepackage{amssymb}
\usepackage{amsthm}
\usepackage{amsfonts}
\usepackage{graphicx}
\usepackage{wrapfig}
\usepackage{amsthm}
\usepackage{mathrsfs}
\usepackage{mathtools}
\usepackage{url}
\usepackage{subfigure}
\usepackage{multirow}
\usepackage{lipsum}
\usepackage{wrapfig}
\usepackage{tikz}
\usepackage[format=plain, font=small, labelfont=bf, textfont=it, justification=centerlast]{caption}
\usepackage{booktabs}
\usepackage{siunitx}
%\usepackage{cleveref}

\newtheorem{izr}{Izrek}
\newtheorem{lem}{Lema}
\newtheorem{trd}{Trditev}
\newtheorem{posl}{Posledica}[izr]

\newcounter{defcount}
\newcounter{opombe}
\newcounter{primercount}
\newcounter{zgledcount}

\newenvironment{opomba}{\begin{flushleft}\refstepcounter{opombe}\textbf{Opomba \arabic{opombe}:}}{\hfill\end{flushleft}}
\setlength{\parindent}{0mm}

\newenvironment{primer}{\begin{flushleft}\refstepcounter{primercount}\textbf{Primer \arabic{primercount}:}}{\hfill\end{flushleft}}
\setlength{\parindent}{0mm}

\newenvironment{zgled}{\begin{flushleft}\refstepcounter{zgledcount}\textbf{Zgled \arabic{zgledcount}:}}{\hfill\end{flushleft}}
\setlength{\parindent}{0mm}

\newenvironment{definicija}{\begin{flushleft}\refstepcounter{defcount}\textbf{Definicija \arabic{defcount}:}}{\hfill\end{flushleft}}
\setlength{\parindent}{0mm}

\newcommand{\naslov}[1]{\textit{#1}}
\newcommand{\abs}[1]{\ensuremath{\lvert #1 \rvert}}
\newcommand{\mth}[1]{\ensuremath{\mathbb{#1}}}
\newcommand{\R}{\mth{R}}
\newcommand{\Z}{\mth{Z}}
\newcommand{\Zp}{\mth{Z}^{+}}
\newcommand{\N}{\mth{N}}
\newcommand{\No}{\mth{N}_0}
\newcommand{\C}{\mth{C}}
\newcommand{\Q}{\mth{Q}}
\newcommand{\Qu}{\mth{Q}_u}
\newcommand{\pojem}[1]{\emph{#1}}
\newcommand{\con}{\ensuremath{\mathscr{C}}}
\newcommand{\padex}[2]{\ensuremath{{#1}^{\underline{#2}}}}
\newcommand{\rastx}[2]{\ensuremath{{#1}^{\bar{#2}}}}
\newcommand{\map}[3]{\ensuremath{{#1}: {#2} \rightarrow {#3}}}
\newcommand{\pra}[3]{{#1}{\ast}({#2}) = {#3}}

\title{Pfaffova diferencialna enačba ter uporaba Fourierjeve in Laplaceove transformacije \\ {\large Seminarska naloga pri predmetu Diferencialne enačbe}}
\date{10.~3.~2024}
\author{Jimmy Zakeršnik}
%===============================================================================
\begin{document}
	\maketitle
	\thispagestyle{empty}
	\newpage
	\tableofcontents
	\newpage
	\section{Uvod}
		Tekom študija diferencialnih enačb pogosto naletimo na posebne primere, ki od nas zahtevajo dodatno pozornost ali razvoj novega orodja in metod. Lahko se zgodi, da, kljub temu, da vemo, da rešitev obstaja, le te ne znamo eksplicitno izraziti z metodami, ki so nam na voljo. V teh situacijah pogosto posežemo po implicitnem podajanju rešitve. Prvi primer tega se skriva v iskanju prvega integrala dane linearne diferencialne enačbe prvega reda, izkaže pa se, da je to le poseben primer t.~i.~ ">Pfaffove diferencialne enačbe"<. V drugih primerih nas lahko bolj zanima, da sploh najdemo kakšno rešitev, ki zadošča določenim pogojem, in na račun tega lahko uporabimo metode, ki nam v splošnem ne bi bile na voljo. Dva primera tega - uporabo Fourierjeve in Laplaceove transformacije - bosta obravnavana v tej seminarski nalogi, skupaj z obravnavo Pfaffovih diferencialnih enačb v dveh in treh spemenljivkah. Posebej, za motivacijo, navajamo naslednje probleme:
		
		\begin{primer}
			\label{prim:Pfaff1}
			Najdi rešitev enačbe: $(5x^3 + 2y^4 + 2y^2z + 2z^3)dx + (4xy^3 + 2xyz)dy + (xy^2 + 2xz)dz = 0$
		\end{primer}
		
		\begin{primer}
			\label{prim:Fourier1}
			Naj bo $c > 0$. Poišči rešitev valovne enačbe z eno prostorsko spremenljivko $\frac{\partial^2 u}{\partial t^2} = c^2\frac{\partial^2 u}{\partial x^2}$ na $\R\times (0, \infty)$ pri začetnih pogojih: $\forall x\in \R: u(x, 0) = f(x)~\&~\frac{\partial u}{\partial t}(x, 0) = g(x)$. Pri tem predpostavi, da sta $f\in \mathcal{C}^2(\R)$ in $g\in\mathcal{C}^1(\R)$.
		\end{primer}
		
		\begin{primer}
			\label{prim:Laplace1}
			Denimo, da imamo navpično postavljeni žleb po katerem spustimo kroglico. Kakšne oblike mora biti žleb, da bo čas potovanja kroglice po njem do izbrane točke neodvisen od začetne točke, s katere smo kroglico spustili? Pri tem zanemarimo zračni upor in trenje.
		\end{primer}
		
		Pri tem se bomo pri reševanju primera \ref{prim:Pfaff1} zgledovali po \cite{bib:raman}, pri reševanju primera \ref{prim:Laplace1} pa po \cite{bib:Mag}. Primer \ref{prim:Fourier1} in njegova rešitev sta povzeta iz lastnih zapiskov.
		
	\section{Pfaffova diferencialna enačba}
		V tem poglavju si bomo ogledali t.~i.~ Pfaffove diferencialne enačbe, posebej v primeru dveh in treh spremenljivk, ter nekatere metode reševanja. Pri tem se bomo v glavni meri naslanjali na vir \cite{bib:raman}, v manjši meri pa tudi na vir \cite{bib:pfaff}.
		\subsection{Osnovne definicije in rezultati}
			Poseben primer Pfaffove enačbe smo spoznali že pri reševanju navadnih linearnih diferencialnih enačb. Rešitev tega posebnega primera smo imenovali prvi integral enačbe. Preden definiramo splošno Pfaffovo enačbo, se spomnimo tega primera.
			
			\begin{definicija}
				\label{def:1stint}
				Naj bo $y' = f(x, y)$ navadna linearna diferencialna enačba prvega reda.~\pojem{Prvi integral} te diferencialne enačbe je funkcija $F(x, y)$, ki je konstantna vzdolž vsake rešitve.
			\end{definicija}
			Torej, če je $F$ prvi integral enačbe $y' = f(x, y)$, za vsako njeno rešitev, $y = y(x)$ z definicijskim območjem $D_y$, obstaja $c\in\R$, da velja: $F(x, y(x)) = c ~\forall x\in D_y$. Ob predpostavki, da je $F$ zvezno parcialno odvedljiva po vseh spremenljivkah in da je vsaj eden od teh parcialnih odvodov različen od $0$, lahko iz enačbe $F(x, y) = c$ po izreku o implicitni funkciji na neki okolici izrazimo $y = y(x)$. Po dodatku k izreku o implicitni funkciji na prej omenjeni okolici velja: $y' = -\frac{F_x}{F_y}$, kjer sta $F_x = \frac{\partial F}{\partial x}$ in $F_y = \frac{\partial F}{\partial y}$ parcialna odvoda po $x$ in $y$. Ko upoštevamo zvezo $y' = \frac{dy}{dx}$ dobimo naslednjo enakost: $$F_x dx + F_y dy = 0$$ Če dodatno vpeljemo oznaki $P = F_x$ in $Q = F_y$, enačba zavzame obliko $P dx + Q dy = 0$. To je ravno Pfaffova enačba v dveh spremenljivkah. Če imamo podano Pfaffovo enačbo v dveh spremenljivkah $P dx + Q dy = 0$, vemo že, da je iskanje rešitve ekvivalentno iskanju vektorskega polja $F$, za katerega velja: $\nabla F = (F_x, F_y) = (P, Q)$, za katerega velja $<\nabla F, (dx, dy)> = 0$, kjer je $<, >$ standardni skalarni produkt. Če vpeljemo oznako $\vec{r} = (x, y)$, dobimo še krajši zapis: $<\nabla F, d\vec{r}> = 0$.
			
			Definirajmo sedaj Pfaffovo diferencialno enačbo v večih spremenljivkah.
			
			\begin{definicija}
				\label{def:PfaffDE}
				Naj bodo $\map{F_i}{\R^n}{\R}$ zvezne funkcije neodvisnih spremenljivk $x_1, x_2, \ldots, x_n$. ~\pojem{Pfaffova diferencialna enačba} je enačba oblike $$\sum_{i = 1}^{n}F_i(x_1, x_2, \ldots, x_n) dx_i = 0$$.
			\end{definicija}
			
			Tudi tukaj vidimo, da lahko reševanje dane enačbe prevedemo na iskanje vektorskega polja $F$ za katerega velja $grad(F) = \nabla F = [F_1, F_2, \ldots, F_n]^\top$ in $<\nabla F , d\vec{r}> = 0$, kjer je $d\vec{r} = [dx_1, dx_2, \ldots, dx_n]^\top$. Vektorska oblika Pfaffove diferencialne enačbe se torej glasi $<\nabla F, d\vec{r}> = 0$. Naravna zahteva je seveda, da je $F$ zvezno parcialno odvedljiv po vseh spremenljivkah. V nadaljevanju bomo za to uporabili krajši zapis - $F\in \mathcal{C}^1$.
			
			Posebej, v primeru treh spremenljivk: $$P(x, y, z)dx + Q(x, y, z)dy + R(x, y, z)dz = 0$$
			
			Kadar obravnavamo Pfaffove diferencialne enačbe, nas tipično zanimajo rešitve oblike $f(x_1, x_2, \ldots, x_n) = c$, kjer je $c$ neka poljubna, tipično realna, konstanta. V primeru treh spremenljivk, te rešitve tvorijo enoparametrično družino ploskev v prostoru ($\R^3$). Če je $R(x, y, z)\neq 0$ in iz enačbe $f(x, y, z) = c$ izrazimo $z = g(x, y)$ lahko preuredimo Pfaffovo diferencialno enačbo v treh spremenljivkah v obliko $dz = -\frac{P(x, y, z)}{R(x, y, z)}dx - \frac{Q(x, y, z)}{R(x, y, z)}dy$. Tako reševanje Pfaffove enačbe prevedemo na reševanje sistema parcialnih diferencialnih enačb prvega reda: $z_x = -\frac{P}{R}$ ter $z_y = -\frac{Q}{R}$. S tem se nam razkrije še ena uporabna vrednost reševanja Pfaffovih diferencialnih enačb v treh spremenljivkah.
			
			Že iz iskanja prvega integrala vemo, da za nekatere funkcije $P$ in $Q$, Pfaffova enačba $Pdx + Qdy = 0$ nima direktne rešitve. Potreben pogoj za obstoj direktne rešitve (za točno to enačbo), torej funkcije $u$, da je $\nabla u = (P, Q)$, je, da velja $P_y = Q_x$. V primeru, ko to ne velja, tipično poiščemo t.~i.~ integrirajoči množitelj $\mu(x, y)$, da velja: $(\mu\cdot P)_y = (\mu\cdot Q)_x$ in potem rešimo enačbo $(\mu\cdot P)dx + (\mu\cdot Q)dy = 0$, njeno rešitev pa razglasimo za prvi integral originalne enačbe $Pdx + Qdy = 0$. Opišimo te lastnosti še za splošno Pfaffovo diferencialno enačbo.
			
			\begin{definicija}
				\label{def:eksaktna}
				Naj bodo $\forall i\in\{1, 2, \ldots, n\}~ \map{F_i}{\R^n}{\R}$ zvezne funkcije, ki določajo Pfaffovo diferencialno enačbo $\sum_{i = 1}^{n}F_i(x_1, x_2, \ldots, x_n)dx_i = 0$. Če obstaja taka funkcija $u(x_1, x_2, \ldots, x_n)\in \mathcal{C}^1$, da za njen totalni diferencial $du$ velja $du = <\nabla u, [dx_1, dx_2, \ldots, dx_n]> = \sum_{i = 1}^{n}F_i dx_i$, potem pravimo, da je enačba \pojem{eksaktna}.
			\end{definicija}
			
			\begin{definicija}
				\label{def:integrabilna}
				Pravimo, da je Pfaffova diferencialna enačba $\sum_{i = 1}^n F_i dx_i = 0$ \pojem{integrabilna}, če obstajata taki funkciji $\mu(x_1, x_2, \ldots, x_n)\in \mathcal{C}^1$ in $u(x_1, x_2, \ldots, x_n)\in \mathcal{C}^1$, da je $<\nabla u,[dx_1, dx_2, \ldots, dx_n]^\top> = \sum_{i = 1}^n (\mu F_i) dx_i$
			\end{definicija}
			
			To, da je enačba integrabilna, je ravno potreben in zadosten pogoj, da rešitev obstaja. V primeru dveh spremenljivk, je ta pogoj ekvivalenten pogoju $P_y = Q_x$, v primeru treh spremenljivk, pa temu, da velja $$<[P, Q, R]^\top, rot([P, Q, R]^\top)> = 0$$
			
			Če je enačba integrabilna, je ena od nam že znanih metod reševanja ta, da poiščemo integrirajoči množitelj. To se lahko zakomplicira, saj je že v primeru dveh spremenljivk iskanje integrirajočega množitelja ekvivalentno reševanju parcialne diferencialne enačbe prvega reda $\mu_y\cdot P + \mu\cdot P_y - \mu_x\cdot Q + \mu\cdot Q_x = 0$ za funkcijo $\mu$. Na srečo obstajajo druge metode, ki jih bomo za primer treh spremenljivk obravnavali v naslednjem podpoglavju. Preden se lotimo le teh, navedimo še definicijo kvazi-homogenih funkcij in par rezultatov v zvezi z njimi, saj bo nam to prišlo prav kasneje.
			
			\begin{definicija}
				\label{def:kvazhomfun}
				Pravimo, da je funkcija $\map{f}{\R^n}{\R}$ \pojem{kvazi-homogena} stopnje (oz. reda) $m\in\Z$, če obstajajo taka neničelna števila $a_1, a_2, \ldots, a_n \in \Z$, da velja: $$f(x_1t^{a_1}, x_2t^{a_2}, \ldots, x_nt^{a_n}) = t^mf(x_1, x_2, \ldots, x_n)$$
				za vsak $t\in\R$.
				V tem primeru pravimo, da je število $a_i$ \pojem{dimenzija} spremenljivke $x_i$.
			\end{definicija}
			
			Seveda velja, da so vse homogene funkcije reda $m$ tudi kvazi-homogene reda $m$ ($a_1 = a_2 = \ldots = a_n = 1$). Poglejmo si primer funkcije, ki je kvazi-homogena, ni pa homogena.
			
			\begin{zgled}
				Poglejmo si polinomsko funkcijo s predpisom $f(x, y) = 4x^3y^3 -3x^2y^6 + 2xy^9 - y^{12}$. Vidimo, da velja $f(tx, ty) = 4t^6x^3y^3 -3t^8x^2y^6 + 2t^{10}xy^9 - t^{12}y^{12} \neq f(x, y)$, torej $f$ ni homogena. Če pa vzamemo $a_1 = 3$ in $a_2 = 1$ dobimo: \begin{align*}
					f(t^3x, ty) &= 4t^{12}x^3y^3 -3t^{12}x^2y^6 +2t^{12}xy^9 - t^{12}y^{12} \\
					&= t^{12}(4x^3y^3 -3x^2y^6 + 2xy^9 - y^{12}) = t^{12}f(x, y)
				\end{align*}
				
				Funkcija $f$ je torej kvazi-homogena reda $12$, ni pa homogena.
			\end{zgled}
			
			Seveda pri kvazi-homogenih funkcijah nismo omejeni samo na polinome.
			
			\begin{zgled}
				Preverimo, da je funkcija $f(x, y) = \abs{y}\sqrt{x}$ kvazi-homogena funkcija reda $2$ z dimenzijama $2$ in $1$.
				$$f(t^2x, ty) = \abs{ty}\sqrt{t^2x} = \abs{t}\abs{y}\abs{t}\sqrt{x} = \abs{t}^2\abs{y}\sqrt{x} = t^2\abs{y}\sqrt{x} = t^2f(x, y)$$
			\end{zgled}
			Da se bolje seznanimo s konceptom kvazi-homogenih funkcij, bomo navedli in dokazali par trditev.
			
			\begin{trd}
				\label{trd:kvazhom1}
				Naj bo $\map{f}{\R^n}{\R}$ kvazi-homogena funkcija reda $m$, z dimenzijami $a_1, a_2, \ldots, a_n$. Za $x_1 \neq 0$ in vse $i\in \{2, 3, \ldots, n\}$ označimo: $b_i = \frac{a_i}{a_1}$ in $y_i = \frac{x_i}{x_1^{b_i}}$. Tedaj je $$f(x_1, x_2, \ldots, x_n) = x_1^{\frac{m}{a_1}}f(1, y_2, \ldots, y_n)$$
			\end{trd}
			\begin{proof}
				Po definiciji vemo, da je $f(x_1t^{a_1}, x_2t^{a_2}, \ldots, x_nt^{a_n}) = t^mf(x_1, x_2, \ldots, x_n)$ za vsak $t\in\R$. Naj bo $x_1 \neq 0$ in $t^{a_1} = \frac{1}{x_1}$ ter s pomočjo slednje enakosti dobimo: $$t^{a_i} = t^{a_1\frac{a_i}{a_1}} = t^{a_1b_i} = (\frac{1}{x_1})^{b_i} = \frac{1}{x_1^{b_i}}$$
				
				Ko to vstavimo v začetno enakost, dobimo: $$f(1, x_2x_1^{-b_2}, \ldots, x_nx_1^{-b_n}) = x_1^{-\frac{m}{a_1}} f(x_1, x_2, \ldots, x_n)$$
				
				Ko enačbo pomnožimo z $x_1^{\frac{m}{a_1}}$ in upoštevamo oznako $y_i = x_ix_1^{-b_i}$ dobimo ravno enakost iz trditve.
			\end{proof}
			
			\begin{trd}
				\label{trd:kvazhom2}
				Naj bo $\map{f}{\R^n}{\R}$ $\mathcal{C}^1$ kvazi-homogena funkcija reda $m$ z dimenzijami $a_1, a_2, \ldots, a_n$. Tedaj velja enakost: $$mf(x_1, x_2, \ldots, x_n) = \sum_{i = 2}^{n}a_ix_i\frac{\partial f}{\partial x_i}(x_1, x_2, \ldots, x_n)$$
			\end{trd}
			
			\begin{proof}
				Po prejšnji trditvi vemo, da za $x_1 \neq 0$ velja: $f(x_1, x_2, \ldots, x_n) = x_1^{\frac{m}{a_1}}f(1, y_2, \ldots, y_n)$. Poračunajmo parcialne odvode od $f$:
				\begin{align*}
					\frac{\partial f}{\partial x_1}(x_1, \ldots, x_n) &= \frac{m}{a_1}x_1^\frac{m-a_1}{a_1}f(1, y_2, \ldots, y_n) + x_1^\frac{m}{a_1}\sum_{i = 2}^{n}x_i(-\frac{a_i}{a_1})x_1^\frac{-(a_i + a_1)}{a_1}\frac{\partial f}{\partial y_i}(1, y_2, \ldots, y_n) \\
					&= \frac{m}{a_1}x_1^\frac{m-a_1}{a_1}f(1, y_2, \ldots, y_n) + \sum_{i = 2}^{n}x_i(-\frac{a_i}{a_1})x_1^\frac{m- a_i - a_1}{a_1}\frac{\partial f}{\partial y_i}(1, y_2, \ldots, y_n)
				\end{align*}
				In za $\forall i \in \{2, \ldots, n\}$ je
				\begin{align*}
					\frac{\partial f}{\partial x_i}(x_1, \ldots, x_n) &= x_1^{\frac{m}{a_1}}x_1^\frac{-a_i}{a_1}\frac{\partial f}{\partial y_i}(1, y_2, \ldots, y_n) = x_1^\frac{m - a_i}{a_1}\frac{\partial f}{\partial y_i}(1, y_2, \ldots, y_n)
				\end{align*}
				
				Potem je \begin{align*}
					a_1x_1\frac{\partial f}{\partial x_1}(x_1, \ldots, x_n) &= m x_1^\frac{m}{a_1}f(1, y_2, \ldots, y_n) + \sum_{i = 2}^{n}(-a_i)x_ix_1^\frac{m - a_i}{a_1}\frac{\partial f}{\partial y_i}(1, y_2, \ldots, y_n) \\
					&= m f(x_1, \ldots, x_n) + \sum_{i = 2}^{n}(-1)(a_ix_i)(x_1^\frac{m - a_i}{a_1}\frac{\partial f}{\partial y_i}(1, y_2, \ldots, y_n)) \\
					&= m f(x_1, \ldots, x_n) - \sum_{i = 2}^{n}(a_ix_i)\frac{\partial f}{\partial x_i}(x_1, \ldots, x_n)
					\end{align*}
				Ko k enačbi prištejemo $\sum_{i = 2}^{n}a_ix_i\frac{\partial f}{\partial x_i}(x_1, \ldots, x_n)$, dobimo ravno enakost iz trditve.
			\end{proof}
			
			\begin{definicija}
				\label{def:pfaffKvazhom}
				Pravimo, da je Pfaffova diferencialna enačba $\sum_{i = 1}^n F_i dx_i = 0$:\begin{itemize}
					\item \pojem{homogena} reda $m$, če so za vsak $i\in\{1, 2, \ldots, n\}$ fukncije $F_i$ homogene funkcije reda $m$.
					\item \pojem{kvazi-homogena} reda $m$, z dimenzijami $a_1, a_2, \ldots, a_n$, če so za vsak $i\in\{1, 2, \ldots, n\}$ fukncije $F_i$ kvazi-homogene funkcije reda $m - a_i$ (z dimenzijami $a_i$).
				\end{itemize} 
			\end{definicija}
			
		\subsection{Metode reševanja}
			V tem podpoglavju bomo našteli in demonstrirali nekaj metod reševanja Pfaffovih enačb v treh spremenljivkah. Te metode ločimo gleda na to, če je enačba eksaktna, ali če je le integrabilna (ne pa eksaktna). Metode za eksaktne enačbe med drugimi vsebujejo:
			\begin{enumerate}
				\item Metoda ostrega pogleda
				\item Reševanje sistema parcialnih diferencialnih enačb prvega reda
				\item Integracija potencialnega polja
				\item Ločevanje spremenljivk
			\end{enumerate}
			Če enačba ni eksaktna, je pa integrabilna, potem nam pogosto koristijo naslednje metode:
			\begin{enumerate}
				\item Enačbe z ločljivo spremenljivko
				\item Homogene enačbe
				\item Natanijeva metoda
				\item Mayerjeva metoda
				\item Bertrandova metoda
				\item Kvazi-homogene enačbe
			\end{enumerate}
			Začnimo z obravnavo metod za eksaktne enačbe:
				\subsubsection{Metode za eksaktne enačbe}
					\paragraph{Metoda ostrega pogleda} 
						Najprej si poglejmo najpreprostejšo metodo - metodo ostrega pogleda. Kot ime metode naimguje, tukaj rešitev ">uganemo"<, kar lahko storimo v nekaterih redkih primerih. 
					
						\begin{zgled}
							\label{zgl:guess}
							Za Pfaffovo enačbo $xdx + ydy + zdz = 0$ lahko na podlagi simetrije in preprostosti funkcij, ki v njej nastopajo, uganemo, da je $u(x, y, z) = \frac{1}{2}(x^2 + y^2 + z^2)$ iskana funkcija, ki nam da družino rešitev u(x, y, z) = c.
						\end{zgled}
					
					V nadaljevanju predpostavimo, da so funkcije $P, Q$ in $R$ vse $\mathcal{C}^1$ funkcije.
					\paragraph{Reševanje sistema PDE prvega reda}	
						Že prej smo omenili, da je reševanje Pfaffove enačbe v treh spremenljivkah ekvivalentno reševanju sistema parcialnih diferencialnih enačb prvega reda. Posledično je naslednja metoda reševanje sistema PDE prvega reda. Denimo torej, da imamo eksaktno Pfaffovo DE v treh spremenljivkah: $P(x, y, z)dx + Q(x, y, z)dy + R(x, y, z)dz = 0$. Tedaj obstaja $\mathcal{C}^2$ funkcija $u$, za katero velja: $\frac{\partial u}{\partial x} = P, \frac{\partial u}{\partial y} = Q ~\text{in}~ \frac{\partial u}{\partial z} = R$. To funkcijo $u$ torej dobimo kot rešitev sistema: 
						\begin{align*}
							u_x(x, y, z) &= P(x, y, z) \\
							u_y(x, y, z) &= Q(x, y, z) \\
							u_z(x, y, z) &= R(x, y, z) 
						\end{align*}
					
						\begin{zgled}
							\label{zgl:PDEr1}
							Reševanje Pfaffove diferencialne enačbe $yze^{xyz}dx + xze^{xyz}dy + xye^{xyz}dz = 0$ prevedemo na reševanje sistema PDE prvega reda: \begin{align*}
								u_x(x, y, z) &= yze^{xyz} \\
								u_y(x, y, z) &= xze^{xyz} \\
								u_z(x, y, z) &= xye^{xyz} 
							\end{align*}
							V vseh treh enačbah nastopa le en odvod po eni spremenljivki, torej lahko vsako posebej tretiramo kot navadno diferencialno enačbo s parametri. Tako hitro vidimo, da funkcija $u(x, y, z) = e^{xyz} + D(y, z)$ zadošča prvi enačbi, $u(x, y, z) = e^{xyz} + C(x, z)$ pa drugi enačbi. Ko slednjo parcialno odvajamo po $x$ dobimo $yze^{xyz} + C_x(x, z) = yze^{xyz}$, torej je $C_x = 0$ oz. $C$ je odvisna le od $z$. Podobno, če prvo enakost parcialno odvajamo po $y$ dobimo enakost $D_y = 0$, torej je tudi $D$ odvisna le od $z$. Ker mora veljati enakost, sledi $C(z) = D(z)$. Sedaj $u(x, y, z) = e^{xyz} + C(z)$ parcialno odvajamo po $z$ in upoštevamo tretjo enačbo v sistemu in s tem vidimo, da je $C_z = 0$, torej je $C$ konstanta. Posledično je $u(x, y, z) = e^{xyz} + C$ iskana funkcija, ki nam da rešitev.
						\end{zgled}
					\paragraph{Integracija potencialnega polja}
						Iz vektorske analize vemo, da, če trojica $(P, Q, R)$ tvori $\mathcal{C}^1$ vektorsko polje $F$, nam eksaktnost enačbe $Pdx + Qdy + Rdz = 0$ pove, da obstaja tako $\mathcal{C}^2$ skalarno polje $u$, da je $\nabla u = F = (P, Q, R)$. Tako lahko uporabimo standardno metodo integriranja potencialnega vektorskega polja, da dobimo potencial $u$, ki določa rešitev dane Pfaffove diferencialne enačbe.
					
						\begin{zgled}
							\label{zgl:intvec}
							Vrnimo se na zgled \ref{zgl:PDEr1} in enačbo $yze^{xyz}dx + xze^{xyz}dy + xye^{xyz}dz = 0$ rešimo še s to metodo. $F(x, y, z) = (yze^{xyz}, xze^{xyz}, xye^{xyz})$ je očitno $\mathcal{C}^1$ vektorsko polje, rutinski račun pa nam pokaže tudi, da je potencialno. Integriramo $P$ po spremenljivki $x$ in dobimo $u(x, y, z) = \int yze^{xyz}dx = e^{xyz} + C(y, z)$. To sedaj parcialno odvajamo po $y$ in dobimo $xze^{xyz} + C_y(y, z) = Q(x, y, z) = xze^{xyz}$, od tod pa sklepamo, da je $C_y = 0$. Sklepamo, da je potem $C$ odvisna le od $z$ in pišemo $C(y, z) = C(z)$. Sedaj enačbo $u(x, y, z) = e^{xyz} + C(z)$ parcialno odvajamo po $z$ in dobimo enakost $xye^{xyz} + C_z(z) = R(x, y, z) = xye^{xyz}$. Od tod sklepamo, da je $C_z(z) = 0$, torej je $C$ konstanta. Dobimo funkcijo $u(x, y, z) = e^{xyz} + C$.
						\end{zgled}
					\paragraph{Ločevanje spremenljivk}
						Metodo ločevanja spremenljivk uporabimo, kadar lahko dano Pfaffovo diferencialno enačbo $P(x, y, z)dx + Q(x, y, z)dy + R(x, y, z)dz = 0$ zapišemo v obliki $\acute{P}(x)dx + \acute{Q}(y)dy + \acute{R}(z)dz = 0$. V tem primeru funkcijo $u$ dobimo kot naslednjo vsoto integralov: $$u(x, y, z) = \int \acute{P}(x)dx + \int \acute{Q}(y)dy + \int \acute{R}(z)dz$$
					
						\begin{zgled}
							\label{zgl:sepvars}
							Vrnimo se k zgledu \ref{zgl:guess}, $xdx + ydy + zdz = 0$, in denimo, da nam ne uspe uganiti rešitve. Ker je to enačba z ločenimi spremenljivkami, lahko uporabimo prejšnjo metodo in tako dobimo $u(x, y, z) = \frac{x^2}{2} + \frac{y^2}{2} + \frac{z^2}{2} + C = \frac{1}{2}(x^2 + y^2 + z^2) + C$
						\end{zgled}
					
					\begin{opomba}
						Vse do sedaj omenjene metode, razen integriranja potencialnega vektorskega polja, lahko posplošimo tudi za reševanje eksaktnih Pfaffovih diferencialnih enačb v več spremenljivkah.
					\end{opomba}
				\subsubsection{Metode za integrabilne enačbe, ki niso nujno eksaktne}
					Poglejmo si sedaj še metode za reševanje Pfaffovih diferencialnih enačb v treh spremenljivkah, ki so integrabilne, ne pa nujno tudi eksaktne.
					\paragraph{Enačbe z ločljivo spremenljivko}
						Najprej si poglejmo t.~i.~ enačbe z ločljivo spremenljivko, torej enačbe v katerih lahko eno spremenljivko ločimo od vseh drugih (na primer $f(x, y, z) = g(x, y) + h(z)$). Denimo, da je spremenljivka $z$ ločljiva spremenljivka v enačbi $P(x, y, z)dx + Q(x, y, z)dy + R(x, y, z)dz = 0$. Tedaj lahko to enačbo preoblikujemo v obliko $\acute{P}(x, y)dx + \acute{Q}(x, y)dy + \acute{R}(z)dz = 0$. Integrabilnost enačbe nam tukaj da pogoj $\frac{\partial \acute{P}}{\partial y} = \frac{\partial \acute{Q}}{\partial x}$, to pa nam pove, da je $\acute{P}(x, y)dx + \acute{Q}(x, y)dy$ totalni diferencial neke funkcije. Označimo to funkcijo z $v$. Torej, $dv = \acute{P}(x, y)dx + \acute{Q}(x, y)dy$ in naša enačba sedaj dobi obliko $dv + \acute{R}(z)dz = 0$. Funkcija $u(x, y, z)$, ki jo iščemo, je potem dobljena kot vsota funkcije $v$ in integrala $\int \acute{R}(z)dz$: $u(x, y, z) = v(x, y) + \int \acute{R}(z)dz$.
					
						\begin{zgled}
							\label{zgl:sepvar}
							Rešujemo enačbo $\frac{(x+y)}{z}dx + \frac{xy+1}{yz}dy + (z^2 + 1)dz = 0$. Enostavno je videti, da $P_z(x, y, z) = \frac{1}{z} \neq 0 = R_x(x, y, z)$, torej naša enačba ni eksaktna. Preverimo pa lahko, da je integrabilna: \begin{align*}
							P(Q_z - R_y) &+ Q(R_x - P_z) + R(P_y - Q_x) =\\ 
							&= \frac{x+y}{z}(\frac{xy+1}{yz^2}-0) + \frac{xy+1}{yz}(0-\frac{x+y}{z^2}) + (z^2 +1)(\frac{1}{z} - \frac{1}{z}) \\
							&= \frac{(x+y)(xy+1)}{yz^3} - \frac{(xy+1)(x+y)}{yz^3} + 0 = 0
							\end{align*}
							Dodatno opazimo, da lahko ločimo spremenljivko $z$ od ostalih s tem, da enačbo pomnožimo z $z$. Preoblikovana enačba se glasi:$$(x+y)dx + \frac{xy+1}{y}dy + z(z^2 +1)dz = 0$$
						
							Pri tem vemo, da je $(x+y)dx + \frac{xy+1}{y}dy$ totalni diferencial neke funkcije $v$. Iz enakosti $v_x = (x+y)$ z integriranjem dobimo $v(x, y) = \frac{x^2}{2} + xy + D(y)$, ko to odvajamo po $y$, pa dobimo $x + D_y(y) = x + \frac{1}{y}$. Od tod sklepamo, da je $D_y(y) = \frac{1}{y}$ oz. $D(y) = \ln|y|$. Funkcija $v$ se torej glasi $v(x, y) = \frac{x^2}{2} + xy + \ln|y|$. Poračunajmo še $\int z(z^2 + 1)dz$:
						
							$$	\int z(z^2 + 1)dz = \int z^3 + z dz = \frac{z^4}{4} + \frac{z^2}{2} + C $$
						
							Sledi torej, da ima iskana funkcija $u$, ki nam da rešitev Pfaffove enačbe, naslednji predpis: $$ u(x, y, z) = \frac{x^2}{2} + xy + \ln|y| + \frac{z^4}{4} + \frac{z^2}{2} + C$$
						\end{zgled}
					
					\paragraph{Homogene enačbe}
						Denimo, da je Pfaffova diferencialna enačba $P(x, y, z)dx + Q(x, y, z)dy + R(x, y, z)dz = 0$ homogena reda $m$. Sedaj vpeljemo novi spremenljivki $u$ in $v$, da velja $y = xv$ ter $z = xw$. Tedaj dobi naša enačba obliko $x^m(P(1, v, w)dx + xQ(1, v, w)dv + xR(1, v, w)dw) = 0$ oziroma $P(1, v, w)dx + xQ(1, v, w)dv + xR(1, v, w)dw = 0$. Dobljena enačba je enačba z ločljivo spremenljivko (specifično, $x$ je ločljiva), ki jo rešimo po prejšnji metodi.
						
						\begin{zgled}
							\label{zgl:homogen}
							Rešujemo Pfaffovo diferencialno enačbo $x^2yzdx + xy^2zdy + xyz^2dz = 0$. Najprej vidimo, da so funkcije $x^2yz, xy^2z$ in $xyz^2$ vse homogene funkcije reda $4$, torej je dana Pfaffova diferencialna enačba homogena reda $4$. Prepričamo se še, da je integrabilna: \begin{align*}
								P(Q_z - R_y) &+ Q(R_x - P_z) + R(P_y - Q_x) =\\ 
								&= x^2yz(xy^2 - xz^2) + xy^2z(yz^2 - x^2y) + xyz^2(x^2z - y^2z) \\
								&= x^3y^3z - x^3yz^3 + xy^3z^3 - x^3y^3z + x^3yz^3 - xy^3z^3 \\
								&= 0
							\end{align*}
							
							Sedaj v Pfaffovo diferencialno enačbo vpeljemo $y = xv$ in $z = xw$ (ter posledično $dy = xdv$ in $dz = xdw$) in tako dobimo: $$P(x, xv, xw)dx + xQ(x, xv, xw)dv +xR(x, xv, xw)dw = 0$$ oziroma $$x^4vwdx + x^5v^2wdv + x^5vw^2dw = 0$$
							Enačbo delimo z $x^4vw$ ter tako dobimo: $$dx + xvdv + xwdw = 0$$ Sedaj delimo enačbo še z $x$ in tako to spremenljivko ločimo od ostalih dveh. Končna oblika preoblikovane enačbe je torej: $$\frac{dx}{x} + vdv + wdw = 0$$ Sedaj se te enačbe lotimo po metodi za enačbe z ločljivo spremenljivko in prepoznamo, da je $vdv + wdw$ totalni diferencial funkcije $f(v, w) = \frac{1}{2}(v^2 + w^2)$. Ko to upoštevamo v naši enačbi dobimo $\frac{dx}{x} + df = 0$, kar nam pa da iskano funkcijo $u(x, v, w) = \ln\abs{x} + \frac{1}{2}(v^2 + w^2)$. Ko se vrnemo na začetne spremenljivke, ima $u$ naslednji predpis: $u(x, y, z) = \ln\abs{x} + \frac{(y^2 + z^2)}{2x^2}$.
							
						\end{zgled}
						
					\paragraph{Natanijeva metoda}
						Natanijeva metoda se Pfaffove diferencialne enačbe loti na naslednji način: Najprej se obnašamo, kot da je ena od spremenljivk konstanta in rešimo pripadajočo Pfaffovo diferencialno enačbo. Torej, če vzamemo $z$ za neko fiksno konstanto, potem rešujemo $P(x, y, z)dx + Q(x, y, z)dy = 0$. Denimo, da je $\Phi_1(x, y, z) = c_1$ rešitev te enačbe (kjer je $c_1$ neka konstanta). Tedaj je rešitev originalne enačbe oblike $\Phi(\Phi_1, z) = c_2$ (kjer je $c_2$ neka konstanta). To rešitev lahko preoblikujemo v obliko $\Phi_1(x, y, z) = \psi(z)$. To storimo tako, da najprej fiksiramo eno od spremenljivk (v tem primeru $x$ ali $y$) v enakosti $\Phi_1(x, y, z) = \psi(z)$. Denimo, da smo fiksirali $x$ in to fiksno vrednost označimo z $x_0$. Tedaj nam enakost $\Phi_1(x_0, y, z) = \psi(z)$ da rešitev Pfaffove diferencialne enačbe $Q(x_0, y, z)dy + R(x_0, y, z)dz = 0$ (saj nam enakost $\Phi_1(x, y, z) = \psi(z)$ predstavlja rešitev Pfaffove DE $P(x, y, z)dx + Q(x, y, z)dy + R(x, y, z)dz = 0$). Enačba $Q(x_0, y, z)dy + R(x_0, y, z)dz = 0$ pa je ravno Pfaffova DE v dveh spremenljivkah in njena rešitev je prvi integral neke navadne diferencialne enačbe prvega reda. Naj bo $K(y, z) = c$ prvi integral te enačbe. Tako $\Phi_1(x_0, y, z) = \psi(z)$ kot tudi $K(y, z) = c$ sta rešitvi pripadajoče navadne diferencialne enačbe prvega reda in po enoličnosti rešitve navadne diferencialne enačbe prvega reda sledi, da ti rešitvi Pfaffove diferencialne enačbe ekvivalentni. Tako lahko eliminiramo spremenljivko $y$ in končno dobimo naš iskani $\psi(z)$. Rešitev enačbe dobimo, ko pridobljeni $\psi(z)$ vstavimo v enakost $\Phi_1(x, y, z) = \psi(z)$.
						
						\begin{zgled}
							\label{zgl:natani}
							Vrnimo se k enačbi v zgledu \ref{zgl:sepvar} in jo rešimo z Natanijevo metodo. V prvem koraku obravnavamo spremenljivko $z$ kot konstanto in rešujemo enačbo $\frac{(x+y)}{z}dx + \frac{xy+1}{yz}dy = 0$. Ta enačba je eksaktna in pripadajoča funkcija $\Phi_1$ se glasi: $\Phi_1(x, y, z) = \frac{1}{z}(\frac{x^2}{2} + xy + \ln\abs{y})$. Rešitev te enačbe je torej $\Phi_1(x, y, z) = c_1$. Vrnimo se sedaj v začetno enačbo in tokrat fiksirajmo $x = 0$. Tedaj je $\Phi_1(0, y, z) = \frac{\ln\abs{y}}{z}$ rešitev enačbe $\frac{1}{yz}dy + (z^2 + 1)dz = 0$. Ta enačba je integrabilna z integrirajočim množiteljem $\mu(x, z) = z$, njena rešitev pa je podana s funkcijo $K(y, z) = \ln\abs{y} + \frac{z^2(z^2 + 2)}{4}$. Rešitev se torej glasi $K(y, z)= c$. in iz te enakosti izrazimo $y$: $y(z) = e^{\frac{4c - z^2(z^2 + 2)}{4}}$ Sedaj upoštevamo enakovrednost $K(y, z) = c$ in $\Phi_1(0, y, z) = \psi(z)$ ter s tem dobimo: $$\frac{4c - z^2(z^2 + 2)}{4z} = \psi(z)$$
							
							Sedaj, ko imamo $\psi(z)$, ga vstavimo v enakost $\Phi_1(x, y, z) = \psi(z)$ in s tem dobimo iskano rešitev, ki jo pa lahko preoblikujemo v standardno obliko $\Phi(x, y, z) = c_2$. 
							
							Storimo še to: 
							
							$$\frac{1}{z}(\frac{x^2}{2} + xy + \ln\abs{y}) = \frac{4c - z^2(z^2 + 2)}{4z}$$
							
							Vpeljemo oznako $c_2 = 4c$, enačbo pomnožimo z $z$ ter vse nekonstantne člene premaknemo na levo stran. Tako dobimo:
							
							$$\frac{x^2}{2} + xy + \ln\abs{y} + \frac{z^4 + 2z^2}{4} = c_2$$
							
						\end{zgled}
						
					\paragraph{Mayerjeva metoda}
						Z Mayerjevo metodo reduciramo iskanje rešitve Pfaffove diferencialne enačbe v treh spremenljivkah ($P(x, y, z)dx + Q(x, y, z)dy + R(x, y, z)dz = 0$) na iskanje prvega integrala neke navadne diferencialne enačbe prvega reda. Denimo, da je dana Pfaffova DE v treh spremenljivkah integrabilna. Tedaj ima rešitev oblike $f(x, y, z) = c$ za neko primerno funkcijo $f$. Ta rešitev tvori enoparametrično družino ploskev v $\R^3$, presečišča teh ploskve z ravnino $x + ky - z = 0$ pa bodo tvorile neskončno družino krivulj, ki pa bodo rešitve neke Pfaffove diferencialne enačbe v dveh spremenljivkah: $p(x, y, k)dx + q(x, y, k)dy = 0$. To enačbo tipično pridobimo tako, da vstavimo $z = x + ky$ v našo prvotno enačbo in s tem eliminiramo spremenljivko $z$. Denimo sedaj, da je $\Phi(x, y, k) = \acute{c}$ splošna rešitev Pfaffove enačbe $p(x, y, k)dx + q(x, y, k)dy = 0$. Vsaka točka na presečišču med $z$-osjo in dano družino ravnin $x + ky = z$ lahko dobimo tako, da nastavimo $y = 0$ in $x = c (=z)$ in če integralska krivulja $\Phi(x, y, k)$ vsebuje to točko, mora veljati $\Phi(x, y, k) = \Phi(c, 0, k)$ (saj je $\Phi$ vzdolž te krivulje konstantna). Če $k$ variiramo, dobimo torej enoparametrično družino krivulj, ki vsebujejo točko $(c, 0)$, če pa variiramo $c$ s tem dobimo družino krivulj, ki tečejo skozi vsako točko na preseku $z$-osi z ravnino $z = x + ky$. Ko iz slednje enačbe izrazimo $k = \frac{z - x}{y}$ in to vstavimo v $\Phi(x, y, k) = \Phi(c, 0, k)$, se znebimo parametra $k$ in tako dobimo družino ploskev $\Phi(x, y, \frac{z - x}{y}) = \Phi(c, 0, \frac{z - x}{y})$, ki je pa ravno naša iskana rešitev Pfaffove DE v treh spremenljivkah z začetka.
						
						\begin{zgled}
							Rešimo Pfaffovo diferencialno enačbo $xdx + ydy + zdz = 0$. Najprej vidimo, da je enačba res integrabilna, saj njene funkcije $P, Q$ in $R$ določajo potencialno vektorsko polje. V enačbo vstavimo enakost $z = x + ky$ in $dz = dx + kdy$ ter s tem dobimo: $$(2x+ky)dx + (kx+(k^2+1)y)dy = 0$$ Označimo $p(x, y, k) = 2x+ky$ in $q(x, y, k)=kx + (k^2+1)y$. Ker je $p_y = k q_x$, je pdx + qdy totalni diferencial neke funkcije $u$. Poračunajmo $u$: $$u(x, y, k)= x^2 + kxy + \frac{(k^2+1)y^2}{2} $$
							
							Zanima nas ravno $u(x, y, k) = u(c, 0, k) = c^2$ oz. $x^2 + kxy + \frac{(k^2+1)y^2}{2} = c^2$. V enačbo sedaj vstavimo $k = \frac{z - x}{y}$ in tako dobimo: $$ x^2 + \frac{xyz - x^2y}{y} + \frac{\frac{(x^2 - 2xz + z^2) + y^2}{y^2}y^2}{2} = c^2$$
							
							To enačbo malo uredimo in dobimo $x^2 + xz - x^2 + \frac{x^2 - 2xz + y^2 + z^2}{2} = c^2$ oziroma $\frac{1}{2}(x^2 + y^2 + z^2) = c^2$. To je ravno rešitev naše orginalne enačbe, kar nas seveda ne preseneča.
								
						\end{zgled}
						
					\paragraph{Bertrandova metoda}
						Pri Bertrandovi metodi se Pfaffove diferencialne enačbe $Pdx + Qdy + Rdz = 0$, ki ni eksaktna (torej $rot[P, Q, R]^\top \neq 0$), je pa integrabilna, lotimo tako, da najprej rešimo linearno parcialno diferencialno enačbo $(Q_z - R_y)u_x + (R_x - P_z)u_y + (P_y - Q_x)u_z = 0$. Če sta $v$ in $w$ dve neodvisna prva integrala te enačbe, potem seveda velja $(Q_z - R_y)v_x + (R_x - P_z)v_y + (P_y - Q_x)v_z = 0$ in $(Q_z - R_y)w_x + (R_x - P_z)w_y + (P_y - Q_x)w_z = 0$. Ko poleg teh dveh enakosti upoštevamo še pogoj integrabilnosti $P(Q_z - R_y) + Q(R_x - P_z) + R(P_y - Q_x) = 0$ tako dobimo sistem treh enačb. Označimo $A = \begin{bmatrix}
							v_x & v_y & v_z \\
							w_x & w_y & w_z \\
							P & Q & R
						\end{bmatrix}$. Tedaj se naš sistem enačb glasi: $$\begin{bmatrix}
							v_x & v_y & v_z \\
							w_x & w_y & w_z \\
							P & Q & R
						\end{bmatrix} \cdot \begin{bmatrix}
						Q_z - R_y \\
						R_x - P_z \\
						P_y - Q_z
						\end{bmatrix} = 0$$ Ker po predpostavki velja $rot[P, Q, R]^\top \neq 0$, sledi, da ima dana matrika netrivialno jedro, torej ni obrnljiva. Posledično velja: $\begin{vmatrix}
							v_x & v_y & v_z \\
							w_x & w_y & w_z \\
							P & Q & R
						\end{vmatrix} = 0$, in ker so vrstice linearno odvisne, lahko najdemo dve funkciji, $\lambda(v, w)$ in $\mu(v, w)$, da velja: \begin{align*}
						P &= \lambda v_x + \mu w_x \\
						Q &= \lambda v_y + \mu w_y \\
						R &= \lambda v_z + \mu w_z
						\end{align*}
						
						Ko to vstavimo v našo začetno Pfaffovo diferencialno enačbo, se ta reducira na obliko $\lambda dv + \mu dw = 0$, ki pa je rešljiva Pfaffova diferencialna enačba v dveh spremenljivkah, saj sta $\lambda$ in $\mu$ odvisni le od $v$ in $w$.
						
						\begin{zgled}
							S pomočjo Bertrandove metode rešimo Pfaffovo diferencialno enačbo iz zgleda \ref{zgl:sepvar}: $\frac{(x+y)}{z}dx + \frac{xy+1}{yz}dy + (z^2 + 1)dz = 0$.
							Zapišimo komponente od $rot[xy, xy, xy]^\top$: \begin{align*}
								a(x, y, z, u) &= Q_x - R_y = \frac{xy+1}{yz^2} \\
								b(x, y, z, u) &= R_x - P_z = -\frac{x+y}{z^2} \\
								c(x, y, z, u) &= P_y - Q_x = 0 \\
								d(x, y, z, u) &= 0
							\end{align*}
							Sedaj s pomočjo funkcij $a, b, c$ in $d$ določimo homogeno linearno PDE prvega reda: $au_x + bu_y + cu_z = d$
							Posebej: $yu_y + (x-y)u_z = 0$. Enačbo bomo rešili s pomočjo metode karakteristik. Zato najprej parametriziramo spremenljivke $x = x(t), y = y(t), z = z(t), u = u(t) = u(x(t), y(t), z(t))$.
							Karakteristični sistem za njo se glasi: \begin{align*}
								\dot{x} &= a(x, y, z, u) = \frac{xy+1}{yz^2} \\
								\dot{y} &= b(x, y, z, u) = -\frac{x+y}{z^2} \\
								\dot{z} &= c(x, y, z, u) = 0 \\
								\dot{u} &= d(x, y, z, u) = 0
							\end{align*}
							
							Ker sta $z$ in $u$ neodvisna od $t$, v resnici rešujemo sistem v prvih dveh vrsticah, kar pa lahko lahko rešimo z iskanjem prvega integrala, kar pa je enakovredno reševanju Pfaffove Diferencialne enačbe $(x+y)dx + \frac{xy+1}{y}dy = 0$. Rešitev te enačbe je pa ravno $w(x, y, z) =  \frac{x^2}{2} + xy + \ln\abs{y}$.
							
							Določimo še en, od $w$ neodvisen prvi integral: $v(x, y, z) = c_3(z)$ za neko $\mathcal{C}^1$ funkcijo $c_3$. Sledi, da je rešitev začetne parcialne diferencialne enačbe funkcija pripadajočih prvih integralov. Posebej, izkaže se, da je rešitev oblike $u(x, y, z) = c_3(z) + (\frac{x^2}{2} + xy +\ln\abs{y})$, kjer je $c_3$ lahko poljubna $\mathcal{C}^1$ funkcija. \begin{align*}
								det(A) = \begin{vmatrix}
									v_x & v_y & v_z \\
									w_x & w_y & w_z \\
									P & Q & R
								\end{vmatrix} &= \begin{vmatrix}
								0 & 0 & c_3'(z) \\
								x+y & \frac{xy+1}{y} & 0 \\
								\frac{x+y}{z} & \frac{xy+1}{yz} & (z^2+1)
								\end{vmatrix} \\
								&= c_3'(z)((x+y)\frac{xy+1}{yz} - \frac{xy+1}{y}\frac{x+y}{z}) = 0
							\end{align*}
							Sedaj poiščimo taki funkciji $\lambda$ in $\mu$, da bo: \begin{align*}
								P &= \mu(x+y) \\
								Q &= \mu \frac{xy+1}{y} \\
								R &= \lambda c_3'(z)
							\end{align*}
							
							Prva enačba nam pove, da je $\mu(x, y, z) = \frac{1}{z}$, kar nam pove tudi druga enačba. Tretja enačba nam določi $\lambda(x, y, z) = \frac{(z^2 +1)}{c_3'(z)}$. Preostane samo še, da s pomočjo $v$ in $w$ izrazimo $\lambda$ in $\mu$, nato pa rešimo enačbo $\lambda(v, w)dv + \mu(v, w)dw = 0$.
							Opazimo, da sta tako $\lambda$ kot $\mu$ obe v resnici odvisni zgolj od $z$, zato bo parametrizacija neodvisna od $w$, saj je ta neodvisna od $z$. Za parametrizacijo se torej obrnemo na $v(z) = c_3(z)$. Dodatno se omejimo na funkcije $c_3(z)$, ki so bijektivne in $\mathcal{C}^2$, ter tako lahko zapišemo $z = c_3^{-1}(v) = f(v)$. Potem je $\lambda(v) = \frac{1}{f(v)}$ in $\mu(v) = \frac{f^2(v) + 1}{c_3'(f(v))}$. S tem smo originalno enačbo prevedli na reševanje enačbe $\frac{dv}{f(v)} + \frac{f^2(v)+1}{c_3'(f(v))}dw = 0$. Ta enačba ni nujno eksaktna, je pa integrabilna za integrirajoči množitelj $\mu(v, w) = \frac{c_3'(f(v))}{(f^2(v) + 1)}$. Tedaj je rešitev enačbe podana s funkcijo $g(v, w) = w + \int \frac{c_3'(f(v))}{f(v)(f^2(v) + 1)}dv$.
							Nadaljno reševanje problema je odvisno od izbire funkcije $c_3$. Poglejmo si poseben primer, ko je $c_3(z) = \frac{z^7}{7} + \frac{2z^5}{5} + \frac{z^3}{3} = v$. Tedaj je $c_3'(z) = z^6 + 2z^4 + z^2$. Potem je $\int \frac{c_3'(f(v))}{f(v)(f^2(v) + 1)}dv = \int \frac{z^6 + 2z^4 + z^2}{z(z^2 + 1)}dz = \int \frac{(z^3 + z)^2dz}{(z^3 + z)} = \int z^3 + z dz = \frac{z^4}{4} + \frac{z^2}{2} + \acute{C}$. Ko to vstavimo v funkcijo $g$ in razpišemo $w(x, y, z)$, dobimo ravno rešitev, ki smo jo dobili v zgledih \ref{zgl:sepvar} in \ref{zgl:natani}.
						\end{zgled}
						
						\begin{opomba}
							Razlog, da je dobljena rešitev enaka kot v ostalih zgledih z isto enačbo, se skriva v tem, kako smo izbrali funkcijo $c_3(z)$. Če bi izbrali kako drugo $c_3$, bi dobili drugo rešitev. ki bi se od te razlikovala le v členih, ki so odvisni od $z$.
						\end{opomba}
					\paragraph{Kvazi-homogene enačbe}
						Za konec tega poglavja si poglejmo še reševanje kvazi-homogenih Pfaffovih diferencialnih enačb v treh spremenljivkah. Za začetek premislimo potrebne pogoje, da je Pfaffova diferencialna enačba $Pdx + Qdy + Rdz = 0$ kvazi-homogena. Denimo, da so $P, Q$ in $R$ naslednje oblike: \begin{align*}
							P(x, y, z) &= \sum_{i = 1}^{n_1}a_ix^{\alpha_i}y^{\beta_i}z^{\gamma_i} \\
							Q(x, y, z) &= \sum_{i = j}^{n_2}b_jx^{\lambda_j}y^{\mu_j}z^{\nu_j} \\
							R(x, y, z) &= \sum_{k = 1}^{n_3}c_kx^{\varepsilon_k}y^{\eta_k}z^{\zeta_k} \\
						\end{align*}
						
						
						Kjer so $a_i, b_j$ in $c_k$ koeficienti in $\alpha_i, \beta_i, \gamma_i, \lambda_j, \mu_j, \nu_j, \varepsilon_k, \eta_k, \zeta_k\in \Q,~ \forall i\in \{1,\ldots, n_1\}, \forall j\in \{1,\ldots, n_2\}, \forall k\in \{1,\ldots, n_3\}$. 
						Tedaj ima naša Pfaffova enačba obliko $$(\sum_{i = 1}^{n_1}a_ix^{\alpha_i}y^{\beta_i}z^{\gamma_i})dx + (\sum_{i = j}^{n_2}b_jx^{\lambda_j}y^{\mu_j}z^{\nu_j})dy + (\sum_{k = 1}^{n_3}c_kx^{\varepsilon_k}y^{\eta_k}z^{\zeta_k})dz = 0$$
						
						Če sedaj v enačbi $x$ zamenjamo z $xt^p$, $y$ zamenjamo z $yt^q$ ter $z$ zamenjamo z $zt^r$, potem je potenca števila $t$ v vsakem členu naše Pfaffove enačbe določena z: \begin{align*}
							p\alpha_i + q\beta_i + r\gamma_i~&;~ i\in \{1, \ldots, n_1\} \\
							p\lambda_j + q\mu_j + r\nu_j~&;~ j\in \{1, \ldots, n_2\} \\
							p\varepsilon_k + q\eta_k + r\zeta_k~&;~ k\in \{1, \ldots, n_3\}
						\end{align*}
						
						To nam da $n_1 + n_2 + n_3$ linearnih izrazov, ki so odvisni od $p, q$ in $r$. Da bi bila naša Pfaffova enačba kvazi-homogena reda $m$, bi potem morala biti $P$ kvazi-homogena reda $m-p$, $Q$ kvazi-homogena reda $m-q$ in $R$ kvazi-homogena reda $m-r$. To pa pomeni, da bi morala biti potenca števila $t$ v vsakem členu funkcije $P$ enaka $m-p$, potenca $t$ v vsakem členu $Q$ bi morala biti $m-q$, potenca $t$ v vsakem členu $R$ pa natanko $m-r$. Tako dobimo $n_1 + n_2 + n_3$ linearnih enačb v $p, q, r$ in $m$, ki se glasijo: \begin{align*}
							p(\alpha_i+1) + q\beta_i + r\gamma_i - m = 0~&;~ i\in \{1, \ldots, n_1\} \\
							p\lambda_j + q(\mu_j+1) + r\nu_j - m= 0~&;~ j\in \{1, \ldots, n_2\} \\
							p\varepsilon_k + q\eta_k + r(\zeta_k+1) - m=0~&;~ k\in \{1, \ldots, n_3\}
						\end{align*}
						
						Če je Pfaffova enačba, ki jo obravnavamo, kvazi-homogena, morajo biti zgornje enačbe medseboj konsistentne, da obstaja enolična rešitev $p, q, r, m$. Potreben pogoj za to, da je dana Pfaffova enačba kvazi-homogena je torej konsistentnost zgornjih enačb. Premislimo sedaj, kako rešujemo kvazi-homogene Pfaffove diferencialne enačbe.
						
						Naj bo $P(x, y, z)dx + Q(x, y, z)dy + R(x, y, z)dz = 0$ kvazi-linearna Pfaffova enačba  reda $m$ z dimenzijami spremenljivk $p, q$ in $r$ (po vrsti za $x, y$ in $z$). Tedaj, po definiciji \ref{def:pfaffKvazhom} velja, da so $P, Q$ in $R$ kvazihomogene funkcije redov $m-p, m-q$ in $m-r$. Če upoštevamo trditev \ref{trd:kvazhom1}, lahko te funkcije zapišemo na naslednji način: \begin{align*}
							P(x, y, z) &= x^{\frac{m-p}{p}}P(1, yx^{-\frac{q}{p}}, zx^{-\frac{r}{p}}) \\
							Q(x, y, z) &= x^{\frac{m-q}{p}}Q(1, yx^{-\frac{q}{p}}, zx^{-\frac{r}{p}}) \\
							R(x, y, z) &= x^{\frac{m-r}{p}}R(1, yx^{-\frac{q}{p}}, zx^{-\frac{r}{p}}) \\
						\end{align*}
						Če uvedemo novi spremenljivki $u = yx^{-\frac{q}{p}}$ in $v = zx^{-\frac{r}{p}}$, potem velja $dy = x^{\frac{q}{p}}du + \frac{q}{p}ux^{\frac{q-p}{p}}dx$ in $dz = x^{\frac{r}{p}}dv + \frac{r}{p}vx^{\frac{r-p}{p}}dx$. Ko to vstavimo v našo začetno enačbo dobimo: \begin{align*}
							x^{\frac{m-p}{p}}P(1, u, v) dx &+ x^{\frac{m-q}{p}}Q(1, u, v)(x^{\frac{q}{p}}du + \frac{q}{p}ux^{\frac{q-p}{p}}dx) \\ 
							&+ x^{\frac{m-r}{p}}R(1, u, v) (x^{\frac{r}{p}}dv + \frac{r}{p}vx^{\frac{r-p}{p}}dx) = 0
						\end{align*}
						Ko združimo člene, ki vsebujejo $dx$, se enačba preoblikuje v obliko: \begin{align*}
						&x^{\frac{m-p}{p}}(pP(1, u, v) + quQ(1, u, v) + rvR(1, u, v))dx \\ &+ px^{\frac{m}{p}}Q(1, u, v)du + px^{\frac{m}{p}}R(1, u, v)dv  = 0 
						\end{align*}
						Sedaj delimo enačbo s faktorjem $x^{\frac{m}{p}}(pP(1, u, v) + quQ(1, u, v) + rvR(1, u, v))$ in s tem ločimo spremenljivko $x$ od $u$ in $v$. Tako pridobljena enačba se glasi: \begin{align*}
							\frac{x^{\frac{m-p}{p}}}{x^{\frac{m}{p}}}dx &+ \frac{pQ(1, u, v)du}{pP(1, u, v) + quQ(1, u, v) + rvR(1, u, v)} \\ 
							&+ \frac{pR(1, u, v)dv}{pP(1, u, v) + quQ(1, u, v) + rvR(1, u, v)} = 0
						\end{align*}
						Ko uvedemo oznaki $$A(u, v) = \frac{pQ(1, u, v)}{pP(1, u, v) + quQ(1, u, v) + rvR(1, u, v)}$$ ter $$B(u, v) = \frac{pR(1, u, v)}{pP(1, u, v) + quQ(1, u, v) + rvR(1, u, v)}$$ se zgornja enačba glasi: $$ \frac{dx}{x} + A(u, v)du + B(u, v)dv = 0$$
						
						Tako enačbo je pa enostavno rešiti z metodo za enačbe z ločljivo spremenljivko.
						
						\begin{zgled}
							Poskusimo rešit primer \ref{prim:Pfaff1}. Imamo torej enačbo $(5x^3 + 2y^4 + 2y^2z + 2z^3)dx + (4xy^3 + 2xyz)dy + (xy^2 + 2xz)dz = 0$. Najprej preverimo, da je enačba sploh integrabilna.
							\begin{align*}
								&P(Q_z - R_y) + Q(R_x - P_z) + R(P_y - Q_x) = \\
								&= (5x^3 + 2y^4 + 2y^2z + 2z^3)(2xy - 2xy)
								+ (4xy^3 + 2xyz)(y^2 + 2z - 2y^2 - 4z) \\ 
								&+ (xy^2 + 2xz)(8y^3 + 4yz - 4y^3-2yz) =\\
								&= 0 - (4xy^5 + 2xy^3z + 8xy^3z +4xyz^2) + (4xy^5 + 8xy^3z + 2xy^3z + 4xyz^2) \\ 
								&= 0
							\end{align*}
							Enačba torej res je integrabilna. Opazimo tudi, da so naše funkcije $P, Q$ in $R$ ravno oblike iz premisleka o potrebnem pogoju za kvazi-homogenost. Preverimo torej, če je naša Pfaffova diferencialna enačba kvazi-homogena.
							
							Pripadajoči sistem enačb se glasi: \begin{align*}
								&4p -m = 0 \\
								&p +4q -m = 0 \\
								&p +2q +r -m = 0 \\
								&p +3r -m = 0 \\
								&p +4q -m = 0 \\
								&p +2q +r -m = 0 \\
								&p +2r -m = 0 \\
							\end{align*}
							
							Če upoštevamo prve $4$ enačbe, ki vse izhajajo iz potenc $t$ v $P$, pogoj konsistentnosti zahteva, da velja: $4p = p+4q = p+2q+r = p+3r$. Prvi dve enakosti nam data $3p = 4q$, druga in tretja pa $2q = r$. Če izberemo $p = 1$ potem dobimo $q = \frac{3}{4}$ in $r = \frac{3}{2}$, dodatno pa je $m=4$. Ker smo našli rešitev so enačbe konsistentne in Pfaffova diferencialna enačba je res kvazi-linearna reda $4$.
							
							Po metodi reševanja, ki smo jo prej opisali, sedaj vpeljemo novi spremenljivki $u$ in $v$, za kateri velja $y = ux^{\frac{3}{4}}$ in $z = vx^{\frac{3}{2}}$. Potem je $dy = x^{\frac{3}{4}}du + \frac{3}{4}ux^{-\frac{1}{4}}dx$ in $dz = x^{\frac{3}{2}}dv + \frac{3}{2}vx^{\frac{1}{2}}$. Naša enačba se tako spremeni v \begin{align*}
								&(5x^3 + 2y^4 + 2y^2z + 2z^3)dx + (4xy^3 + 2xyz)(x^{\frac{3}{4}}du + \frac{3}{4}ux^{-\frac{1}{4}}dx) \\ &+ (xy^2 + 2xz)(x^{\frac{3}{2}}dv + \frac{3}{2}vx^{\frac{1}{2}}) = 0
							\end{align*}
							in poenostavi v $$
								x^3(5 + 5u^4 + 5u^2v + 5v^2)dx + x^4(4u^3 + 2uv)du + x^4(u^2 + 2v)dv = 0
							$$
							
							Sedaj delimo enačbo z $x^4(5 + 5u^4 + 5u^2v + 5v^2)$ in jo tako transformiramo v obliko: $$
							\frac{dx}{x} + \frac{(4u^3 + 2uv)du}{(5 + 5u^4 + 5u^2v + 5v^2)} + \frac{(u^2 + 2v)dv}{(5 + 5u^4 + 5u^2v + 5v^2)} = 0$$
							
							Tukaj opazimo, da zadnja dva člena v resnici tvorita totalni diferencial. Naj bo $w = 1 + u^4 + u^2v + v^2$. Tedaj je $dw = (4u^3 + 2uv)du + (u^2 + 2v)dv$. Tedaj se naša enačba glasi: $$\frac{dx}{x} + \frac{dw}{5w} = 0$$
							
							Enostavno je videti, da je potem $\ln\abs{x} + \frac{1}{5}\ln\abs{w} = C$ rešitev naše Pfaffove diferencialne enačbe. To rešitev se da napisati z bolj elementarnimi funkcijami. \begin{align*}
								\ln\abs{x} &+ \frac{1}{5}\ln\abs{w} = \ln\abs{x} + \ln(\abs{1 + u^4 + u^2v + v^2}^{\frac{1}{5}}) \\ &= \ln(\abs{x(1 + u^4 + u^2v + v^2)}^{\frac{1}{5}}) = C
							\end{align*}
							
							To bo veljalo natanko tedaj, ko bo $\abs{x(1 + u^4 + u^2v + v^2)}^{\frac{1}{5}} = D$ oz. $\abs{x}^5(1 + u^4 + u^2v + v^2) = E$. Sedaj vstavimo še spremenljivki $y$ in $z$ ter tako dobimo $x^5(1 + \frac{y^4}{x^3} + \frac{y^2z}{x^3} + \frac{z^2}{x^3}) = E$ oziroma $$x^5 + x^2y^4 + x^2y^2z + x^2z^2 = E$$
							
							Le to je končna oblika rešitve naše začetne enačbe.
						\end{zgled}
					
	\section{Fourierjeva transformacija}
		Sedaj se lotimo drugega pristopa k reševanju diferencialnih enačb. V tem poglavju bomo obravnavali, kako lahko uporabimo t.~i.~ Fourierjeve transformacije, da reševanje neke diferencialne enačbe prevedemo na običajno lažji problem reševanja navadnih enačb. Za začetek spoznajmo, kaj sploh je Fourierjeva transformacija, skupaj z nekaj koristne teoretične osnove, nato pa se lotimo opisa metode in z njo rešimo nekaj zgledov. V tem poglavju se bomo pretežno sklicevali na učbenik \cite{bib:ana4}.
		\subsection{Osnovne definicije in rezultati}
			Najprej se spomnimo, kaj je Hilbertov prostor.
			\begin{definicija}
				\label{def:Banach}
				Vektorski prostor $X$, opremnljen z normo $\abs{\abs{.}}$ je \pojem{Banachov prostor}, če je za metriko, porojeno iz norme, $d(x, y) = \abs{\abs{x-y}}$ poln metrični prostor.
			\end{definicija}
			\begin{opomba}
				\label{op:hilbspac}
				\begin{itemize}
					\item Vektorski prostor $\R^n$ skupaj s standardnim skalarnim produktom, ki porodi Evklidsko normo, je Banachov prostor.
					\item Naj bo $[a, b]$ poljuben zaprti interval in naj bo $\mathcal{C}([a, b])$ prostor vseh zveznih realnih (ali kompleksnih) funkcij nad $[a, b]$. Definiramo normo s predpisom $\abs{\abs{f}} = (\int_{a}^{b}\abs{f(x)}^2dx)^{\frac{1}{2}} ~\forall f\in \mathcal{C}([a, b])$. Ta prostor ni poln metrični prostor za porojeno metriko. Da se o tem prepričamo, je dovolj obravnavati funkcijsko zaporedje $f_n(x) = \begin{cases}
						0~&;~x\in[0, \frac{1}{2}-\frac{1}{n}] \\
						\frac{n}{2}x - \frac{n-2}{4}~&;~x\in[\frac{1}{2}-\frac{1}{n}, \frac{1}{2}+\frac{1}{n}] \\
						1~&;~x\in[\frac{1}{2}+\frac{1}{n}, 1]
					\end{cases}$ na intervalu $[0, 1]$. To zaporedje je Cauchyjevo, kandidat za limito $f$ pa je funkcija $f(x)=\begin{cases}
					0~&;~x\in [0, \frac{1}{2}] \\
					1~&;~x\in [\frac{1}{2}, 1] \\
					\end{cases}$, ki pa ni element $\mathcal{C}([0, 1])$.
					\item prostor $\mathcal{C}([a, b])$ skupaj z normo s predpisom $\abs{\abs{f}} = \max\{\abs{f(x)}~|~x\in[a, b]\} ~\forall f\in \mathcal{C}([a, b])$ je Banachov prostor.
				\end{itemize}
			\end{opomba}
			
			\begin{definicija}
				\label{def:mera}
				Naj bo $K = [a_1, b_1]\times \ldots \times [a_n, b_n]$ kvader v $\R^n$. Za njegovo prostornino razglasimo $V(K) = \prod_{i=1}^{n}\abs{b_i-a_i}$. Naj bo sedaj $A$ poljubna množica v $\R^n$. Pravimo, da ima $A$ \pojem{mero $0$}, če $\forall\varepsilon>0$ obstaja kvečjemu števno pokritje množice $A$ s kvadri, $K_1, K_2, \ldots$, da velja $\sum_{i = 1}^{\infty}V(K_i) < \varepsilon$
			\end{definicija}
			
			\begin{opomba}
				Vse števne in končne množice v $\R^n$ imajo mero $0$.
			\end{opomba}
			
			\begin{definicija}
				\label{def:eqsp}
				Naj bo $A$ poljubna podmnožica v $\R$ in $f$ ter $g$ realni ali kompleksni funkciji na $A$. Pravimo, da sta $f$ in $g$ \pojem{enaki skoraj povsod} na $A$, če ima množica točk, v katerih se funkcijske vrednosti razlikujejo $\{x\in A~|~ f(x) \neq g(x)\}$ mero $0$. Tedaj pišemo $f = g~ s.p.$
			\end{definicija}
			
			\begin{opomba}
				Relacija enakosti skoraj povsod je ekvivalenčna relacija
			\end{opomba}
			
			\begin{definicija}
				Naj bo $A$ poljuben interval kar cel $\R$. Tedaj definiramo množico, ki vsebuje ekvivalenčne razrede realnih ali kompleksnih funkcij nad $A$, glede na enakost skoraj povsod, za katere velja, da je $\int_{A}\abs{f(x)}dx < \infty$. To množico imenujemo \pojem{množica absolutno integrabilnih funkcij} na $A$ in jo označimo z $L^1(A)$. Torej: $L^1(A) = \{[f]_{=~ s.p.}~|~f~\textit{funkcija na}~A~\land \int_{A}\abs{f(x)}dx ~\text{obstaja in je končen}\}$.
			\end{definicija}
			
			Nas bodo posebej zanimale absolutno integrabilne funkcije na celi realni osi: $L^1(\R) = \{[f]_{=~s.p.}~|~\map{f}{\R}{\R}~(\textit{ali}~\C) \land \int_{-\infty}^{\infty}\abs{f(x)}dx <\infty\}$. Tudi ta prostor je, za po točkah definirani operaciji $+$ in $\cdot$ vektorski prostor. Še več, če ga opremimo z normo s predpisom $\abs{\abs{f}}_1 = \int_{-\infty}^{\infty}\abs{f(x)}dx$, postane $L^1(\R)$ Banachov prostor.
			
			\begin{definicija}
				Naj bo $A$ poljuben interval kar cel $\R$. Množico zveznih in omejenih funkcij funkcij na $A$ označimo z $\mathcal{C}_b(A)$.
			\end{definicija}
			
			Posebej nas bo zanimala množica $\mathcal{C}_b(\R)$, ki skupaj s po točkah definiranima $+$ in $\cdot$, ter normo s predpisom $\abs{\abs{f}}_\infty = \sup_{x\in \R}\abs{f(x)}~\forall f\in\mathcal{C}_b(\R)$ tvori Banachov prostor.
			
			\begin{definicija}
				\label{def:FourierT}
				Preslikavo $\map{\mathcal{F}}{L^1(\R)}{\mathcal{C}_b(\R)}$, s predpisom $\mathcal{F}(f)(y) = \int_{-\infty}^{\infty}e^{iyt}f(t)dt~\forall y\in\R$, imenujemo \pojem{Fourierjeva transformacija}. Za vsak $f\in L^1(\R)$ pravimo $\mathcal{F}(f)$ \pojem{Fourierjeva transformiranka funkcije} $f$, in jo tipično označimo kar z $\hat{f}$.
			\end{definicija}
			
			Preden se lotimo obravnave lastnosti Fourierjeve transformacije, se spomnimo dveh rezultatov, ki nam bosta pri tem pomagala.
			
			\begin{definicija}
				Funkcija $\map{f}{[a, b]}{\R}$ je \pojem{odsekoma zvezno odvedljiva}, če je odsekoma zvezna na $[a, b]$ in je zvezno odvedljiva na $[a, b]$, razen v končno mnogo točkah, v katerih obstajata leva in desna limita odvoda.
			\end{definicija}
			
			\begin{lem}[Riemman-Lebesguova lema]
				\label{lem:R-L}
				Naj bo $f$ absolutno integrabilna ($L^1$) funkcija na nekem končnem ali neskončnem intervalu $I$ Potem velja: $$\lim_{\lambda\to\infty}\int_I f(t)cos(\lambda t)dt = 0, \lim_{\lambda\to\infty}\int_I f(t)sin(\lambda t)dt = 0$$ in posledično tudi $$\lim_{\lambda\to\infty}\int_I f(t)e^{i\lambda t}dt = 0$$
			\end{lem}
			
			\begin{izr}
				\label{izr:omzvlinop}
				Naj bosta $X$ in $Y$ realna ali kompleksna normirana vektorska prostora ter naj bo $\map{A}{X}{Y}$ linearni operator. Tedaj velja: $A$ je omejen $\iff~A$ je zvezen.
			\end{izr}
			Sedaj se lotimo lastnosti Fourierjeve transformacije.
			\begin{trd}
				\label{trd:propFourierT}
				Naj bosta $f, g\in L^1(\R)$ poljubni. Za fourierjevo transformacijo veljajo naslednje trditve: \begin{enumerate}
					\item $\mathcal{F}$ je zvezni linearni operator.
					\item Za vsak $a\in\R\setminus\{0\}$ je $\mathcal{F}(f(at))(y) = \frac{1}{\abs{a}}\mathcal{F}(f(t))(\frac{y}{a})~\forall y\in\R$.
					\item $\mathcal{F}(\bar{f})(y) = \overline{\mathcal{F}(f)(-y)}~\forall y\in\R$.
					\item Za vsak $a\in\R$ je $\mathcal{F}(f(t-a))(y)= e^{iay}\mathcal{F}(f(t))(y)~\forall y\in\R$.
					\item Za vsak $a\in\R$ je $\mathcal{F}(e^{iat}f(t))(y) = \mathcal{F}(f(t))(y+a)~\forall y\in\R$.
					\item $\forall f\in L^1(\R)$ je funkcija $\mathcal{F}(f)$ enakomerno zvezna na $\R$.
					\item $\lim_{y\to\infty}\mathcal{F}(f)(y) = 0$
				\end{enumerate}
			\end{trd}
			\begin{proof}
				\begin{enumerate}
					\item To, da je $\mathcal{F}$ linearni operator sledi direktno iz linearnosti določenega integrala. Dokažimo torej, da je $\mathcal{F}$ zvezen. Upoštevajoč, da sta $L^1(\R)$ in $\mathcal{C}_b(\R)$ normirana vektorska prostora nad $\R$ ali $\C$, se skličemo na izrek \ref{izr:omzvlinop} in vidimo, da je dovolj pokazati, da je $\mathcal{F}$ omejen. Naj bosta torej $f\in L^1(\R)$ in $y\in \R$ poljubna elementa ter računamo $\abs{\mathcal{F}(f)(y)} = \abs{\int_{-\infty}^\infty f(t)e^{iyt}dt}\leq \int_{-\infty}^{\infty}\abs{f(t)}\abs{e^{iyt}}dt = \int_{-\infty}^{\infty}\abs{f(t)}dt = \abs{\abs{f}}_1$. Torej, pokazali smo, da poljubna $f\in L^1(\R)$ in $y\in \R$ velja $\abs{\mathcal{F}(f)(y)} \leq \abs{\abs{f}}_1$, to pa pomeni, da neenakost velja tudi, če bi na levi strani vzeli supremum po vseh $y\in\R$. Torej: $\abs{\abs{\mathcal{F}(f)}}_\infty = \sup_{y\in\R}\abs{\mathcal{F}(f)(y)} \leq 1\cdot \abs{\abs{f}}_1$. Ker to velja za poljuben $f\in L^1(\R)$, sledi, da je $\mathcal{F}$ omejen linearni operator, torej je po izreku \ref{izr:omzvlinop} $\mathcal{F}$ zvezen.
					\item Naj bo $a$ poljubno neničelno realno število. Tedaj je $\mathcal{F}(f(at))(y) = \int_{-\infty}^\infty f(at)e^{iyt}dt$. Denimo, da je $a>0$ in vpeljemo novo spremenljivko $u = at, du = a dt$ ter tako dobimo: \begin{align*}\mathcal{F}(f(at))(y) &= \int_{-\infty}^\infty f(u)e^{iy\frac{u}{a}}\frac{1}{a}du = \frac{1}{a}\int_{-\infty}^\infty f(u)e^{i\frac{y}{a}u}du \\ &= \frac{1}{a}\mathcal{F}(f(t))(\frac{y}{a})\end{align*} Za $a<0$ je dokaz simetričen.
					\item Naj bo $y\in\R$ poljuben. \begin{align*}
						\mathcal{F}(\bar{f})(y) &= \int_{-\infty}^\infty \overline{f(t)}e^{iyt}dt = \int_{-\infty}^\infty \overline{f(t)}\cdot\overline{e^{-iyt}}dt \\ &= \int_{-\infty}^\infty \overline{f(t)e^{i(-y)t}}dt = \overline{\int_{-\infty}^\infty f(t)e^{i(-y)t}dt} = \overline{\mathcal{F}(f)(-y)}
					\end{align*}
					\item Naj bosta $a$ in $y$ poljubni realni števili. Tedaj je $$\mathcal{F}(f(t-a))(y) = \int_{-\infty}^{\infty}f(t-a)e^{ity}dt$$ Sedaj vpeljemo novo spremenljivko $u = t-a, du = dt$, ter tako dobimo:\begin{align*}
						\mathcal{F}(f(t-a))(y) &= \int_{-\infty}^{\infty}f(u)e^{i(u+a)y}du = e^{iay}\int_{-\infty}^{\infty}f(u)e^{iuy}du \\ &= e^{iay}\mathcal{F}(f(t))(y)
					\end{align*}
					\item Naj bosta $a, y\in\R$ poljubna. Tedaj velja: \begin{align*}
						\mathcal{F}(e^{iat}f(t))(y) &= \int_{-\infty}^{\infty}e^{iat}f(t)e^{ity}dt =\int_{-\infty}^{\infty}f(t)e^{it(y+a)}dt \\
						&= \mathcal{F}(f(t))(y+a)
					\end{align*}
					\item Naj bo $f\in L^1(\R)$ poljubna funkcija. Tedaj $\forall y, h\in \R$ velja \begin{align*}
					\abs{\hat{f}(y+h) - \hat{f}(y)} &= \abs{\int_{-\infty}^{\infty}f(t)(e^{it(y+h)}- e^{ity})dt} \\ &= \abs{\int_{-\infty}^{\infty}e^{ity}f(t)(e^{ith}-1)dt} = (*)
					\end{align*}
					Sedaj se spomnimo, da je $\forall x \in\R \abs{e^{ix}} = 1$ in $\abs{e^{ix} - 1} \leq 2$. Dodatno, ker je $f\in L^1(\R)$, Za vsak $\varepsilon>0$ obstaja tak $M>0$, da je $\int_{-\infty}^{-M}\abs{f(t)}dt < \frac{\varepsilon}{6}$ in da je $\int_{M}^{\infty}\abs{f(t)}dt < \frac{\varepsilon}{6}$. Naj bo torej $\varepsilon>0$ in izberemo tak $M>0$, da bosta veljali prejšnji oceni. Tedaj velja: \begin{align*}
						(*) &= \abs{\int_{-\infty}^{-M}e^{ity}f(t)(e^{ith}-1)dt + \int_{-M}^{M}e^{ity}f(t)(e^{ith}-1)dt \\ 
							&+ \int_{M}^{\infty}e^{ity}f(t)(e^{ith}-1)dt} \\
						&\leq \abs{\int_{-\infty}^{-M}e^{ity}f(t)(e^{ith}-1)dt} + \abs{\int_{-M}^{M}e^{ity}f(t)(e^{ith}-1)dt} \\
						&+ \abs{\int_{M}^{\infty}e^{ity}f(t)(e^{ith}-1)dt} \\
						&\leq \int_{-\infty}^{-M}\abs{e^{ity}f(t)(e^{ith}-1)}dt + \int_{-M}^{M}\abs{e^{ity}f(t)(e^{ith}-1)}dt \\
						&+ \int_{M}^{\infty}\abs{e^{ity}f(t)(e^{ith}-1)}dt \\
						&< \frac{2\varepsilon}{3} + \int_{-M}^{M}\abs{f(t)}\abs{(e^{ith}-1)}dt
					\end{align*}
					Ker je funkcija $th\mapsto e^{ith}$ zvezna, obstaja tak $\delta_0 > 0$, da velja sklep: 
					$$\abs{th} < \delta_0 \Rightarrow \abs{e^{ith} - 1} < \frac{\varepsilon}{3\abs{\abs{f}}_1}$$
					Vzemimo $\delta = \frac{\delta_0}{M}$. Potem velja: Če je $\abs{h} < \delta$, je $\abs{t}\abs{h}\leq M\abs{h} < \delta_0$ in posledično lahko ocenimo: \begin{align*}
						\int_{-M}^{M}\abs{f(t)}\abs{(e^{ith}-1)}dt &< \int_{-M}^{M}\abs{f(t)}\frac{\varepsilon}{3\abs{\abs{f}}_1}dt = \frac{\varepsilon}{3\abs{\abs{f}}_1}\int_{-M}^{M}\abs{f(t)}dt \\
						&< \frac{\varepsilon}{3\abs{\abs{f}}_1}\int_{-\infty}^{\infty}\abs{f(t)}dt < \frac{\varepsilon}{3\abs{\abs{f}}_1}\abs{\abs{f}}_1 = \frac{\varepsilon}{3}
					\end{align*}
					
					Povzemimo: Za dani $\varepsilon$ smo našli tak $\delta> 0$, da $\forall y\in\R$ velja: $$\abs{h} < \delta \Rightarrow \abs{\bar{f}(y+h)-\bar{f}(y)} < \varepsilon$$
					Sledi, da je $\bar{f}$ enakomerno zvezna na $\R$.
					\item To je direktna posledica (Riemman-Lebesgueve) leme \ref{lem:R-L}.
				\end{enumerate}
			\end{proof}
			
			Sedaj, ko poznamo nekaj bolj osnovnih lastnosti Fourierjeve transformacije, si oglejmo glavno lastnost, ki nas bo zanimala v zvezi z reševanjem diferencialnih enačb. Preden navedemo glavni rezultat, dokažimo še eno lemo.
			
			\begin{lem}
				\label{lem:limfx}
				Naj bo $\map{f}{[0, \infty)}{\R}$ odvedljiva povsod, razen morda v končno mnogo točkah in naj bo njen odvod, $f'$, integrabilna funkcija. Naj velja tudi $\int_0^\infty f(x)dx < \infty$ in $\int_0^\infty f'(x)dx < \infty$. Dodatno predpostavimo: $$f(x) = \int_{0}^{x} f'(t)dt + f(0) ~\forall x\in [0,\infty)$$ Tedaj je $\lim_{x\to\infty}f(x) = 0$
			\end{lem}
			\begin{proof}
				Ker je $f'$ integrabilna, je $\forall x\in [0, \infty): \abs{f(x)} = \abs{\int_{0}^{x} f'(t)dt + f(0)} \leq \int_{0}^{\infty} \abs{f'(t)}dt + \abs{f(0)} < \infty$. Posledično sklepamo, da $\lim_{x\to\infty}f(x)$ obstaja. Denimo sedaj, da je $\lim_{x\to\infty}f(x) = c > 0$. Potem obstaja tak $x_0 \in [0, \infty)$, da je za vsak $x\geq x_0 f(x) \geq \frac{c}{2}$. Sledi potem, da je $\int_{x_0}^{x_0 + 2n}f(x)dx \geq 2n\frac{c}{2} = nc$. Ko pošljemo $n$ proti $\infty$ pa gre ta izraz proti $\infty$, torej je $\int_{x_0}^{\infty}f(x)dx > \infty$, to pa je protislovno s predpostavko, da je $f$ integrabilna na $[0, \infty)$. Sledi potem, da je $\lim_{x\to\infty}f(x) = 0$.
			\end{proof}
			
			\begin{izr}
				\label{izr:FourierTodv1}
				Naj bo $f$ poljubna absolutno integrabilna funkcija na $\R$ z lastnostjo, da njen odvod, $f'$, obstaja povsod, razen morda v končno mnogo točkah, ter je tudi sama absolutno integrabilna funkcija na $\R$. Denimo še, da $\forall x\in \R$ velja $f(x) = \int_0^x f'(t)dt + f(0)$. Tedaj je $\mathcal{F}(f')(y) = (-iy)\mathcal{F}(f)(y)$.
			\end{izr}
			\begin{proof}
				 Denimo torej, da je $f, f'\in L^1(\R)$, kjer $f'$ obstaja povsod na $\R$, razen morda v končno točkah in naj velja $\forall x\in \R f(x) = \int_0^x f'(t)dt + f(0)$. Potem je: \begin{align*}
					\mathcal{F}(f')(y) &=  \int_{-\infty}^{\infty}f'(t)e^{iyt}dt = \lim_{A\to\infty}\int_{-A}^{0}f'(t)e^{iyt}dt + \lim_{B\to\infty}\int_{0}^{B}f'(t)e^{iyt}dt \\
					&= \lim_{A\to\infty}(f(t)e^{ity} |_{-A}^0 - iy\int_{-A}^{0}f(t)e^{iyt}dt) +\\ 
					&+ \lim_{B\to\infty}(f(t)e^{ity}-iy|_0^{B}\int_{0}^{B}f(t)e^{iyt}dt) \\
					&= \lim_{A\to\infty}(f(t)e^{ity} |_{-A}^0) - iy\int_{-\infty}^{0}f(t)e^{iyt}dt + \lim_{B\to\infty}(f(t)e^{ity}|_0^{B}) \\ &- iy\int_{0}^{\infty}f(t)e^{iyt}dt \\
					&=\lim_{A\to\infty}(f(t)e^{ity} |_{-A}^0) + \lim_{B\to\infty}(f(t)e^{ity}|_0^{B})- iy\int_{-\infty}^{\infty}f(t)e^{iyt}dt 
				\end{align*}
				Vidimo, da smo blizu končnega rezultata. Preostane nam le še obravnava limit.\begin{align*}
					\lim_{A\to\infty}(f(t)e^{ity} |_{-A}^0) &+ \lim_{B\to\infty}(f(t)e^{ity}|_0^{B}) \\
					&=  f(0) - \lim_{A\to\infty}(f(-A)e^{-iAy}) + \lim_{B\to\infty}(f(B)e^{iBy})- f(0) \\
				\end{align*}
				Sedaj upoštevamo prejšnjo lemo \ref{lem:limfx} in vidimo, da je potem $$\lim_{A\to\infty}(f(-A)e^{-iAy}) = 0 = \lim_{B\to\infty}(f(B)e^{iBy})$$
				
				Sledi, da je $\mathcal{F}(f')(y) = (-iy)\int_{-\infty}^{\infty}f(t)e^{ity}dt = (-iy)\mathcal{F}(f)(y)$.
			\end{proof}
			
			\begin{posl}
				\label{posl:FTodvN}
				Denimo, da je $f$ $k$-krat odvedljiva na $\R$ ter, da so $f$ in vsi njeni odvodi absolutno integrabilni na $\R$ ter da je vsak od teh (razen $k$-tega odvoda) integral svojega odvoda $\big(\forall x\in\R, \forall i\in \{1, \ldots, k\}: f^{(i-1)}(x) = \int_{0}^{x}f^{(i)}(t)dt + f^{(i-1)}(0)\big)$. Potem je $\mathcal{F}(f^{(k)})(y) = (-iy)^k\mathcal{F}(f)(y)$.
			\end{posl}
			\begin{proof}
				Zaporedna uporaba izreka \ref{izr:FourierTodv1}.
			\end{proof}
			\begin{izr}
				\label{izr:odvFourierT}
				Naj bosta funkciji $f(t)$ in $tf(t)$ obe absolutno integrabilni na $\R$. Tedaj je $\frac{d}{dy}\mathcal{F}(f)(y) = \mathcal{F}(itf)(y)~\forall y\in\R$.
			\end{izr}
			\begin{proof}
				Naj bo $y\in \R$ poljuben. Direktno izračunajmo odvod $\bar{f}$: \begin{align*}
				\lim_{h\to0}(\frac{\bar{f}(y+h) - \bar{f}(y)}{h}) &= \lim_{h\to 0}(\frac{1}{h}\mathcal{F}(e^{iht}f(t))(y) - \mathcal{F}(f(t)(y))) \\
				&= \lim_{h\to 0}(\mathcal{F}(f(t)\frac{(e^{ith}-1)}{h})(y)) = (*)
				\end{align*}
				Sedaj opazimo, da je $\lim_{h\to 0}\frac{e^{ith}-1}{h} = \frac{d}{dx}(e^{itx})(x=0) = ite^{0} = it$. To pomeni, da je $\lim_{h\to 0}f(t)\frac{e^{ith}-1}{h} = itf(t)$ in ker je po predpostavki $tf(t)$ absolutno integrabilna, integral $\int_{-\infty}^\infty (\lim_{h\to 0}f(t)\frac{(e^{ith}-1)}{h})dt$ obstaja in posledično obstaja tudi integral $\int_{-\infty}^\infty (\lim_{h\to 0}f(t)\frac{(e^{ith}-1)}{h})e^{ity}dt$. To pa pomeni, da lahko zamenjamo limito in integral. Ko to storimo, dobimo ravno $(*)$. Po drugi strani, pa je $\int_{-\infty}^\infty (\lim_{h\to 0}f(t)\frac{(e^{ith}-1)}{h})e^{ity}dt=\int_{-\infty}^\infty itf(t)e^{ity}dt = \mathcal{F}(itf(t))(y)$. Sledi, da je $\frac{d}{dy}\mathcal{F}(f)(y) = \mathcal{F}(itf)(y)$.
			\end{proof}
			
			\begin{posl}
				\label{posl:nodvFT}
				Naj bodo funkcije $f(t), tf(t), t^2f(t), \ldots, t^nf(t)$ vse absolutno integrabilne na $\R$. Tedaj je $\frac{d^n}{dy^n}\mathcal{F}(f(t))(y) = \mathcal{F}((it)^nf(t))(y) \forall y\in \R$.
			\end{posl}
			\begin{proof}
				Rezultat dobimo z zaporedno uporabo izreka \ref{izr:odvFourierT}.
			\end{proof}
			
			Sedaj, ko razmeroma dobro poznamo lastnosti Fourierjeve transformacije, se lahko vprašamo naslednje: \begin{enumerate}
				\item Ali obstaja inverzna transformacija? Če da, kdaj oz. pod katerimi pogoji?
				\item Ali obstaja kakšna zveza med produktom funkcij in fourierjevo transformacijo?
			\end{enumerate}
			
			\begin{definicija}
				\label{def:dinijev}
				Za $f\in L^1(\R)$ pravimo, da v točki $t\in\R$ zadošča \pojem{Dinijevem pogoju}, če obstaja tak $a>0$, da obstaja integral $\int_{0}^a \abs{\frac{f(t+u) + f(t-u) - 2f(t)}{u}}du$.
			\end{definicija}
			
			\begin{opomba}
				Opazimo, da če je funkcija $f$ zvezna in v točki $t$ obstajata levi ter desni odvod, je Dinijev pogoj izpolnjen. Posebej, zvezno odvedljive funkcije zadoščajo Dinijevem pogoju na celem $\R$.
			\end{opomba}
			
			\begin{izr}
				\label{izr:invFourierT}
				Če $f\in L^1(\R)$ v točki $t\in\R$ zadošča Dinijevem pogoju (za nek $a>0$), potem velja: $$f(t) = \lim_{R\to\infty}\frac{1}{2\pi}\int_{-R}^{R}\mathcal{F}(f)(x)e^{-itx} dx$$
			\end{izr}
			
			\begin{proof}
				Naj bo $f$ poljubna funkcija iz $L^1(\R)$, ki zadošča Dinijevem pogoju za izbrana fiksna $t\in\R$ in $a>0$ Označimo $F(x) = \mathcal{F}(f)(x)$ in upoštevamo, da je $F(x)$ zvezna in omejena funkcija na $\R$. Posledično, za vsak $R>0$ integral $\frac{1}{2\pi}\int_{-R}^{R}F(x)e^{-itx}dx = S(R, t)$ obstaja. Pa ga poračunajmo: $$
					S(R, t) = \frac{1}{2\pi}\int_{-R}^{R}F(x)e^{-itx}dx = \frac{1}{2\pi}\int_{-R}^{R}(\int_{-\infty}^{\infty}f(u)e^{ixu}du)e^{-itx}dx = (*)
				$$
				Zaradi enakomerne konvergence integrala $\int_{-\infty}^{\infty}f(u)(\int_{-R}^{R}e^{i(u-t)x}dx) du$ lahko vrstni uporabimo Fubinijev izrek in zamenjamo vrstni red integriranja. Sledi: $$
					(*) = \frac{1}{2\pi}\int_{-\infty}^{\infty}f(u)(\int_{-R}^{R}\cos((u-t)x) + i\sin((u-t)x) dx) du 
				$$
				tukaj upoštevamo, da je $\sin$ liha funkcija ter da jo integriramo po simetričnem območju $[-R, R]$. Njen integral bo potem enak $0$, kar pomeni, da je: \begin{align*}
					(*) &= \frac{1}{2\pi}\int_{-\infty}^{\infty}f(u)(\int_{-R}^{R}\cos((u-t)x)dx) du = \frac{1}{2\pi}\int_{-\infty}^{\infty}f(u)\big[\frac{\sin((u-t)x)}{u-t}|_{-R}^{R}\big]du \\
					&= \frac{1}{2\pi}\int_{-\infty}^{\infty}f(u)\frac{2\sin((u-t)R)}{u-t}du = \\
					&= \frac{1}{\pi}\big(\int_{-\infty}^{t}f(u)\frac{\sin((u-t)R)}{u-t}du + \int_{t}^{\infty}f(u)\frac{\sin((u-t)R)}{u-t}du\big) = (\boxdot)
				\end{align*}
				
				V levi integral sedaj uvedemo novo spremenljivko $x = t-u, dx = -du$, v desnega pa $x = u-t, dx = du$. Tako dobimo: $$
				(\boxdot) = \frac{1}{\pi}\big(\int_{0}^{\infty}f(t-x)\frac{\sin(Rx)}{x}dx + \int_{0}^{\infty}f(t+x)\frac{\sin(Rx)}{x}dx\big)
				$$
				
				Sedaj upoštevamo znan rezultat iz analize: Vemo, da je $\int_{0}^{\infty}\frac{\sin(z)}{z}dz = \frac{\pi}{2}$. Z uvedbo nove spremenljivke $z = Rx, dz = Rdx$, je potem tudi $\int_{0}^{\infty}\frac{\sin(Rx)}{x}dx = \frac{\pi}{2}$. Posledično lahko zapišemo $f(t) = \frac{2}{\pi}\int_{0}^{\infty}f(t)\frac{\sin(Rx)}{x}dx$.
				
				Sedaj, upoštevajoč prejšnjo enakost in $(\boxdot) = \frac{2}{\pi}\int_{0}^{\infty}\frac{f(t-x) + f(t+x)}{2}\frac{\sin(Rx)}{x}dx$, poračunajmo $S(R, t) - f(t)$: \begin{align*}
					S(R, t) - f(t) &= \frac{2}{\pi}\int_{0}^{\infty}\big(\frac{f(t-x) + f(t+x)}{2} - f(t)\big)\frac{\sin(Rx)}{x}dx \\
					&= \frac{2}{\pi}\int_{0}^{\infty}\big(\frac{f(t-x) + f(t+x)-2f(t)}{2}\big)\frac{\sin(Rx)}{x}dx
				\end{align*}
				Označimo $g(t, x) = \frac{f(t-x) + f(t+x)-2f(t)}{2}$ in se spomnimo, da $f$ za naša fiksirana $t$ in $a>0$ zadošča Dinijevem pogoju, torej obstaja integral $\int_{0}^{a}\abs{\frac{g(t, x)}{x}}dx$. Sledi, da je za ta izbrani $a>0$ $S(R, t) - f(t) =\int_{0}^{a}\frac{g(t, x)}{x}\sin(Rx)dx + \int_{a}^{\infty}\frac{g(t, x)}{x}\sin(Rx)dx$. Izberimo sedaj poljuben $\varepsilon>0$. Po (Riemman-Lebesguevi) lemi \ref{lem:R-L} obstaja tak $R_1 > 0$, da bo za vse $R > R_1$ veljalo $\abs{\frac{g(t, x)}{x}\sin(Rx)} < \frac{\varepsilon\pi}{4a}$. Posledično za vse $R>R_0$ velja: \begin{align*} 
					\abs{\int_{0}^{a}\frac{g(t, x)}{x}\sin(Rx)dx} &\leq \int_{0}^{a}\abs{\frac{g(t, x)}{x}\sin(Rx)}dx \\
					&< \int_{0}^{a}\frac{\varepsilon\pi}{4a}dx = \frac{\varepsilon\pi}{4}
				\end{align*} 
					Velja tudi: \begin{align*}
						\abs{\int_{a}^{\infty}\frac{g(t, x)}{x}\sin(Rx)dx} &= \abs{\int_{a}^{\infty}\frac{f(t+x) + f(t-x)}{2x}\sin(Rx)dx - \int_{a}^{\infty}f(t)\frac{\sin(Rx)}{x}dx} \\
						&\leq \abs{\int_{a}^{\infty}\frac{f(t+x) + f(t-x)}{2x}\sin(Rx)dx} + \abs{\int_{a}^{\infty}f(t)\frac{\sin(Rx)}{x}dx}
					\end{align*}
					Ker je funkcija $x \mapsto f(t+x)$ absolutno integrabilna na $\R$, je funkcija $x\mapsto\frac{f(t+x)}{x}$ absolutno integrabilna na $[a, \infty)$ in posledično enak sklep (ker je $L^1([a, \infty)$ vektorski prostor) velja tudi za funkcijo $x\mapsto\frac{f(t+x) + f(t-x)}{2x}$. Po lemi \ref{lem:R-L} je \begin{align*}
						&\lim_{R\to\infty}\big(\int_{a}^{\infty}\frac{f(t+x) + f(t-x)}{2x}\sin(Rx)dx\big) \\ &= \int_{a}^{\infty}\big(\lim_{R\to\infty}\frac{f(t+x) + f(t-x)}{2x}\sin(Rx)dx\big) = 0
					\end{align*} Posledično obstaja tak $R_2 >0$, da za vsak $R>R_2$ velja: $$\abs{\int_{a}^{\infty}\frac{f(t+x) + f(t-x)}{2x}\sin(Rx)dx} < \frac{\varepsilon\pi}{8}$$
					
					Na koncu opazimo še, da je $\abs{\int_{a}^{\infty}f(t)\frac{\sin(Rx)}{x}dx} = \abs{f(t)}\abs{\int_{a}^{\infty}\frac{\sin(Rx)}{x}dx}$. V integral vpeljemo novo spremenljivko: $y = Rx, dy = Rdx$. Tedaj velja: $$\abs{f(t)}\abs{\int_{a}^{\infty}\frac{\sin(Rx)}{x}dx} = \abs{f(t)}\abs{\int_{Ra}^{\infty}\frac{\sin(y)}{y}dy}$$
					Ker je $a>0$ fiksno število in vemo, da je funkcija $y\mapsto \frac{\sin(y)}{y}$ absolutno integrabilna na $[a, \infty)$, obstaja tak $R_3 > 0$, da za vsak $R>R_3$ velja: $$\abs{\int_{Ra}^{\infty}\frac{\sin(y)}{y}dy} < \frac{\varepsilon\pi}{8\abs{f(t)}}$$
					
					Označimo $R_0 = \max\{R_1, R_2, R_3\}$. Tedaj za vsak $R>R_0$ velja: \begin{align*}
						\abs{S(R, t) - f(t)} &= \frac{2}{\pi}\abs{\int_{0}^{a}\frac{g(t, x)}{x}\sin(Rx)dx + \int_{a}^{\infty}\frac{g(t, x)}{x}\sin(Rx)dx} \\
						&\leq \frac{2}{\pi}\big(\abs{\int_{0}^{a}\frac{g(t, x)}{x}\sin(Rx)dx} + \abs{\int_{a}^{\infty}\frac{g(t, x)}{x}\sin(Rx)dx}\big) \\
						&< \frac{2}{\pi}\big(\frac{\varepsilon\pi}{4} + \abs{\int_{a}^{\infty}\frac{f(t+x) + f(t-x)}{2x}\sin(Rx)dx}\big) \\ 
						&+ \frac{2}{\pi}\abs{\int_{a}^{\infty}f(t)\frac{\sin(Rx)}{x}dx} \\
						&< \frac{\varepsilon}{2} + \frac{2}{\pi}\frac{\varepsilon\pi}{8} + \frac{2}{\pi}\abs{f(t)}\frac{\varepsilon\pi}{8\abs{f(t)}} \\
						&= \frac{\varepsilon}{2} + \frac{\varepsilon}{4} + \frac{\varepsilon}{4} = \varepsilon
					\end{align*}
					
					Velja torej: $f(t) = \lim_{R\to\infty}S(R, t) = \lim_{R\to\infty}\frac{1}{2\pi}\int_{-R}^{R}F(x)e^{-itx}dx$.
			\end{proof}
			
			Prej smo opazili, da zvezno odvedljive funkcije zadoščajo Dinijevem pogoju. Po izreku \ref{izr:invFourierT} potem velja, da je Fourierjeva transformacija obrnljiva na $\mathcal{C}^1(\R)\subset L^1(\R)$. Še več, če je $f\in\mathcal{C}^2(\R)$ in so $f, f', f''\in L^1(\R)$, potem je $\forall t\in\R: f(t) = \frac{1}{2\pi}\int_{-\infty}^{\infty}F(x)e^{-itx}dx$. S tem smo odgovorili na prvo vprašanje. Odgovorimo sedaj še na drugo.
			
			\begin{definicija}
				\label{def:konvolucija}
				Naj bosta $f, g\in L^1(\R)$. Funkcijo $(f*g)$ s predpisom $(f*g)(x) = \int_{-\infty}^{\infty}f(t)g(x-t)dt \forall x\in\R$ imenujemo \pojem{konvolucija} $f$ in $g$.
			\end{definicija}
			
			\begin{trd}
				\label{trd:konvolucija}
				Naj bodo $f, g, h\in L^1(\R)$ poljubne. Velja: \begin{enumerate}
					\item $f*g \in L^1(\R)$
					\item $\forall a\in \R: (af)*g = a(f*g)$
					\item $(f+g)*h = f*h + g*h$
					\item $(f*g)*h = f*(g*h)$
					\item $f*g= g*f$
				\end{enumerate} 
			\end{trd}
			
			\begin{proof}
				\begin{enumerate}
					\item Velja: $\int_{-\infty}^{\infty}\abs{(f*g)(x)}dx \leq \int_{-\infty}^{\infty}\big(\int_{-\infty}^{\infty}\abs{f(t)}\abs{g(x-t)}dt\big)dx$. Sedaj upoštevamo, da je funkcija $x\mapsto g(x-t)$ absolutno integrabilna za vsak $t\in\R$. Potem lahko zamenjamo vrstni red integriranja in velja: $$\int_{-\infty}^{\infty}\big(\int_{-\infty}^{\infty}\abs{f(t)}\abs{g(x-t)}dt\big)dx = \int_{-\infty}^{\infty}\abs{f(t)}\big(\int_{-\infty}^{\infty}\abs{g(x-t)}dx\big)dt = \abs{\abs{f}}_1\abs{\abs{g}}_1$$
					Velja torej: $\int_{-\infty}^{\infty}\abs{(f*g)(x)}dx \leq \abs{\abs{f}}_1\abs{\abs{g}}_1 < \infty$, torej je $(f*g)\in L^1(\R)$.
					\item Naj bo $a\in\R$. Tedaj je $$((af)*g)(x) = \int_{-\infty}^{\infty}af(t)g(x-t)dt = a\int_{-\infty}^{\infty}f(t)g(x-t)dt = a(f*g)(x)$$
					\item Kratek račun pokaže: \begin{align*}
						((f+g)*h)(x) &= \int_{-\infty}^{\infty}(f(t)+ g(t))h(x-t)dt = \\
						&= \int_{-\infty}^{\infty}f(t)h(x-t)dt + \int_{-\infty}^{\infty}g(t)h(x-t)dt \\
						&= (f*h)(x) + (g*h)(x)
					\end{align*}
					\item 
					\begin{align*}
						(f*(g*h))(x) &= \int_{-\infty}^{\infty}f(t)(g*h)(x-t)dt \\
						&= \int_{-\infty}^{\infty}f(t)\big(\int_{-\infty}^{\infty}g(s)h(x-t-s)ds\big)dt = (*)
					\end{align*}
					Če uvedemo novo spremenljivko $y = t+s, dy = ds$, potem se integral $(*)$ spremeni v $$
					(*) =\int_{-\infty}^{\infty}\big(\int_{-\infty}^{\infty}f(t)g(y-t)h(x-y)dy\big)dt
					$$
					Ker je $f*g \in L^1(\R)$, lahko zamenjamo vrstni red integriranja. Potem je \begin{align*} 
					(*) &= \int_{-\infty}^{\infty}\big(\int_{-\infty}^{\infty}f(t)g(y-t)h(x-y)dt\big)dy \\
					&= \int_{-\infty}^{\infty}\big(\int_{-\infty}^{\infty}f(t)g(y-t)dt\big)h(x-y)dy \\
					&= \int_{-\infty}^{\infty}(f*g)(y)h(x-y)dy = ((f*g)*h)(x)
					\end{align*}
					\item V $(f*g)(x)$ vpeljemo novo spremenljivko $u = x-t, du = -dt$. Tedaj je \begin{align*}
						(f*g)(x) &= \int_{-\infty}^{\infty}f(t)g(x-t)dt = -\int_{\infty}^{-\infty}f(x-u)g(u)du \\
						&=\int_{-\infty}^{\infty}g(u)f(x-u)du = (g*f)(x)
					\end{align*}
				\end{enumerate}
			\end{proof}
			
			\begin{izr}
				\label{izr:konvFT}
				Naj bosta $f$ in $g$ absolutno integrabilni funkciji na $\R$. Tedaj velja: $$\mathcal{F}(f*g)(y) = \mathcal{F}(f)(y)\cdot\mathcal{F}(g)(y)$$
			\end{izr}
			\begin{proof}
				Po definiciji je $$\mathcal{F}(f*g)(y) = \int_{-\infty}^{\infty}e^{ixy}(f*g)(x)dx = \int_{-\infty}^{\infty}e^{ixy}\big(\int_{-\infty}^{\infty}f(t)g(x-t)dt\big)dx$$
				Ker integral $\int_{-\infty}^{\infty}e^{ixy}g(x-t)dx$ enakomerno konvergira, lahko po Fubinijevem izreku zamenjamo vrstni red integracije in tako dobimo: $$
				\mathcal{F}(f*g)(y) = \int_{-\infty}^{\infty}f(t)\big(\int_{-\infty}^{\infty}e^{ixy}g(x-t)dx\big)dt
				$$
				Sedaj vpeljemo novo spremenljivko: $s = x-t, ds = dx$ in tako dobimo: \begin{align*}
					\mathcal{F}(f*g)(y) &= \int_{-\infty}^{\infty}f(t)\big(\int_{-\infty}^{\infty}e^{i(s+t)y}g(s)ds\big)dt \\
					&= \int_{-\infty}^{\infty}e^{ity}f(t)\big(\int_{-\infty}^{\infty}e^{isy}g(s)ds\big)dt \\
					&= \int_{-\infty}^{\infty}e^{ity}f(t)dt \cdot\int_{-\infty}^{\infty}e^{isy}g(s)ds \\
					&= \mathcal{F}(f)(y)\cdot\mathcal{F}(g)(y)
				\end{align*}
				
			\end{proof}
			
			S tem smo pa tudi odgovorili na drugo zastavljeno vprašanje. Sedaj, ko smo spoznali Fourierjevo transformacijo, se lahko lotimo premisleka, kako jih lahko uporabimo pri reševanju diferencialnih enačb.
			
		\subsection{Metoda uporabe}
		
		\subsection{Primeri uporabe}
		
	\section{Laplaceova transformacija}
	
		\subsection{Osnovne definicije in rezultati}
		
		\subsection{Metoda uporabe}
		
		\subsection{Primeri uporabe}
		
	
	\begin{thebibliography}{99}
		
		\bibitem{bib:pfaff} C.~K.~Fong,~\emph{Equations involving differentials: Pfaffian equations}, [ogled 10.~3.~2024], dostopno na \url{https://people.math.carleton.ca/~ckfong/S12.pdf}.
		
		\bibitem{bib:Mag} B.~Magajna,~\emph{Uvod v diferencialne enačbe, kompleksno in Fourierjevo analizo}, DMFA -- založništvo, Ljubljana, 2018.
		
		\bibitem{bib:raman} K.~R.~Unni,~\emph{Pfaffian differential expressions and equations}, diplomsko delo, v: All graduate theses and dissertations, [ogled 10.~3.~2024], dostopno na \url{https://core.ac.uk/download/pdf/127676355.pdf}.
		
		\bibitem{bib:ana4} E.~Zakrajšek,~\emph{Analiza IV}, DMFA -- založništvo, Ljubljana, 1999.
		
	\end{thebibliography}
\end{document}