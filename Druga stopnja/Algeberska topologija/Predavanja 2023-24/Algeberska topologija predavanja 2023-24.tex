\documentclass[a4paper, 10pt]{article}
\usepackage[T1]{fontenc}
\usepackage[utf8]{inputenc}
\usepackage[slovene]{babel}
\usepackage{lmodern}
\usepackage{amsmath}
\usepackage{leftidx}
\usepackage{amssymb}
\usepackage{amsfonts}
\usepackage{graphicx}
\usepackage{wrapfig}
\usepackage{amsthm}
\usepackage{mathrsfs}
\usepackage{mathtools}
\usepackage{url}
\usepackage{subfigure}
\usepackage{multirow}
\usepackage{lipsum}
\usepackage{wrapfig}
\usepackage{tikz}
\usepackage[format=plain, font=small, labelfont=bf, textfont=it, justification=centerlast]{caption}
\usepackage{booktabs}
\usepackage{siunitx}
\usepackage{enumerate}
\usepackage{ulem}

\newtheorem{trditev}{Trditev}
\newtheorem{izr}{Izrek}
\newtheorem{nal}{Naloga}
\newtheorem{posl}{Posledica}

\newcounter{defcount}
\newcounter{opombe}
\newcounter{zgledcount}

\newenvironment{opomba}{\begin{flushleft}\stepcounter{opombe}\textbf{Opomba \arabic{opombe}:}}{\hfill\end{flushleft}}
\setlength{\parindent}{0mm}

\newenvironment{zgled}{\begin{flushleft}\stepcounter{zgledcount}\textbf{Zgled \arabic{zgledcount}:}}{\hfill\end{flushleft}}
\setlength{\parindent}{0mm}

\newenvironment{definicija}{\begin{flushleft}\stepcounter{defcount}\textbf{Definicija \arabic{defcount}:}}{\hfill\end{flushleft}}
\setlength{\parindent}{0mm}

\newenvironment{resitev}{\begin{flushleft}\textit{Rešitev:}}{\hfill\end{flushleft}}
\setlength{\parindent}{0mm}

\newcommand{\abs}[1]{\ensuremath{\lvert #1 \rvert}}
\newcommand{\mth}[1]{\ensuremath{\mathbb{#1}}}
\newcommand{\R}{\mth{R}}
\newcommand{\Z}{\mth{Z}}
\newcommand{\N}{\mth{N}}
\newcommand{\No}{\mth{N}_0}
\newcommand{\C}{\mth{C}}
\newcommand{\Cc}{\mathcal{C}}
\newcommand{\Dd}{\mathcal{D}}
\newcommand{\Mu}{\mathcal{M}}
\newcommand{\sigalg}{$\sigma$-algebra~}
\newcommand{\Qu}{\mth{Q}_u}

\newcommand{\pojem}[1]{\emph{#1}}
\newcommand{\Ob}[1]{\mathcal{O}b({#1})}
\newcommand{\Mor}[2][ ]{\mathcal{M}or_{#1}({#2})}
\newcommand{\con}{\ensuremath{\mathscr{C}}}
\newcommand{\padex}[2]{\ensuremath{{#1}^{\underline{#2}}}}
\newcommand{\rastx}[2]{\ensuremath{{#1}^{\bar{#2}}}}
\newcommand{\map}[3]{\ensuremath{{#1}: {#2} \rightarrow {#3}}}
\newcommand{\pra}[3]{{#1}{\ast}({#2}) = {#3}}

\title{Algeberska topologija\\ Zapiski predavanj}
\date{2023/24}
%===============================================================================
\begin{document}
	\maketitle
	\newpage
	\begin{abstract}
		\noindent Dokument vsebuje zapiske predavanj predmeta Algeberska topologija v okviru študija prvega letnika magistrskega študija matematike na FNM.
	\end{abstract}
	\tableofcontents
	\newpage
	\section{Uvodna motivacija}
	Tekom matematične izobrazbe se spoznamo z mnogimi t.~i.~strukturami, ki tipično zavzamejo obliko ">množica $+$ nekaj"<. Med njimi imamo tipično tudi preslikave, ki jim pogosto damo posebno ime. Naštejmo nekaj primerov.
	\begin{table}[h!]
		\centering
		\begin{tabular}{|l|l|l|}
			Ime & Oznaka & ime preslikav \\\hline
			Množice & $M$ & preslikave oz. funkcije \\\hline
			Grupe & $(G, \circ)$ & homomorfizmi grup \\\hline
			Abelove grupe & $(G, +)$ & homomorfizmi Ab.~grup \\\hline
			Polja & $(F, +, \cdot)$ & homomorfizmi polj \\\hline
			Vektorski prostori nad poljem $F$ & $(V, +, \cdot)$ & linearne preslikave \\\hline
			Delno urejene množice & $(P, \leqq)$ & naraščajoče funkcije \\\hline
			Linearno urejene množice & $(L, \leqq)$ & naraščajoče funkcije \\\hline
			Metrični prostori & $(X, d)$ & zvezne funkcije \\\hline
			Topološki prostori & $(X, \mathcal{T})$ & zvezne funkcije \\\hline
		\end{tabular}
	\end{table}
	Prej omenjene preslikave (na neki strukturi) lahko seveda tudi komponiramo. Za komponiranje velja, da obstajata leva in desna enota ter da je asociativno za preslikave, ki se ustrezno ujemajo z domenami in kodomenami (npr. za preslikave $f$, $g$ in $h$ mora, če želimo formirati $h\circ (g \circ f)$ veljati, da je kodomena $f$ hkrati domena $g$ ter da je kodomena $g$ hkrati domena $h$.)
	Posplošena obravnava lastnosti skupin določenih struktur nas privede do t.~.i~ teorije kategorij.
	
	\section{Kategorije}
	\begin{definicija}
		Razred $\mathcal{C}$ z delno binarno operacijo $\circ$ je \pojem{kategorija}, če velja: \begin{itemize}
			\item $\mathcal{C}$ je unija disjunktnih razredov $\Ob{\mathcal{C}}$ in $\Mor{\mathcal{C}}$. Elementom $\Ob{\mathcal{C}}$ pravimo \pojem{objekti}, elementom $\Mor{\mathcal{C}}$ pa \pojem{morfizmi}.
			\item Za vsak $f\in \Mor{\mathcal{C}}$ sta enolično določena ">začetek"< in ">konec"<, ki sta oba objekta kategorije $\mathcal{C}$. Pišemo $\map{f}{X}{Y}$.
			\item Za poljubna objekta $X, Y \in \Ob{\mathcal{C}}$ je $\Mor[\Cc]{(X, Y)} = \{f\in\Mor{\Cc}; \map{f}{X}{Y}\}$ množica (ne samo razred).
			\item Za poljubna morfizma $\map{f}{X}{Y}$ in $\map{g}{Y}{Z}$ je enolično definiran morfizem $\map{g\circ f}{X}{Z}$ in velja:
			\begin{enumerate}
				\item Za poljubne morfizme $\map{f}{X}{Y}, \map{g}{Y}{Z}$ in $\map{h}{Z}{W}$ je $(h\circ g)\circ f = h\circ (g\circ f)$.
				\item Za vsak $X\in \Ob{\Cc}$ obstaja enolično določen morfizem $1_X \in \Mor[\Cc]{(X, X)}$, z lastnostjo: $\forall \map{f}{X}{Y} \land \forall \map{g}{Z}{X}$ je $f\circ 1_X = f$~in~$1_X \circ g = g$ (Za poljubna $Y, Z\in\Ob{\Cc}$).
			\end{enumerate}
		\end{itemize}
	\end{definicija}
\end{document}