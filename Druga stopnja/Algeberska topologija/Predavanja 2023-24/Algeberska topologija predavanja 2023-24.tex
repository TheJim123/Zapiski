\documentclass[a4paper, 10pt]{article}
\usepackage[T1]{fontenc}
\usepackage[utf8]{inputenc}
\usepackage[slovene]{babel}
\usepackage{lmodern}
\usepackage{amsmath}
\usepackage{leftidx}
\usepackage{amssymb}
\usepackage{amsfonts}
\usepackage{graphicx}
\usepackage{wrapfig}
\usepackage{amsthm}
\usepackage{mathrsfs}
\usepackage{mathtools}
\usepackage{url}
\usepackage{subfigure}
\usepackage{multirow}
\usepackage{lipsum}
\usepackage{wrapfig}
\usepackage{tikz}
\usepackage[format=plain, font=small, labelfont=bf, textfont=it, justification=centerlast]{caption}
\usepackage{booktabs}
\usepackage{siunitx}
\usepackage{enumerate}
\usepackage{ulem}

\newtheorem{trditev}{Trditev}
\newtheorem{izr}{Izrek}
\newtheorem{nal}{Naloga}
\newtheorem{posl}{Posledica}

\newcounter{defcount}
\newcounter{opombe}
\newcounter{zgledcount}

\newenvironment{opomba}{\begin{flushleft}\stepcounter{opombe}\textbf{Opomba \arabic{opombe}:}}{\hfill\end{flushleft}}
\setlength{\parindent}{0mm}

\newenvironment{zgled}{\begin{flushleft}\stepcounter{zgledcount}\textbf{Zgled \arabic{zgledcount}:}}{\hfill\end{flushleft}}
\setlength{\parindent}{0mm}

\newenvironment{definicija}{\begin{flushleft}\stepcounter{defcount}\textbf{Definicija \arabic{defcount}:}}{\hfill\end{flushleft}}
\setlength{\parindent}{0mm}

\newenvironment{resitev}{\begin{flushleft}\textit{Rešitev:}}{\hfill\end{flushleft}}
\setlength{\parindent}{0mm}

\newcommand{\abs}[1]{\ensuremath{\lvert #1 \rvert}}
\newcommand{\mth}[1]{\ensuremath{\mathbb{#1}}}
\newcommand{\R}{\mth{R}}
\newcommand{\Z}{\mth{Z}}
\newcommand{\N}{\mth{N}}
\newcommand{\No}{\mth{N}_0}
\newcommand{\C}{\mth{C}}
\newcommand{\Cc}{\mathcal{C}}
\newcommand{\Dd}{\mathcal{D}}
\newcommand{\Mu}{\mathcal{M}}
\newcommand{\sigalg}{$\sigma$-algebra~}
\newcommand{\Qu}{\mth{Q}_u}

\newcommand{\pojem}[1]{\emph{#1}}
\newcommand{\Ob}[1]{\mathcal{O}b({#1})}
\newcommand{\Mor}[2][ ]{\mathcal{M}or_{#1}({#2})}
\newcommand{\con}{\ensuremath{\mathscr{C}}}
\newcommand{\padex}[2]{\ensuremath{{#1}^{\underline{#2}}}}
\newcommand{\rastx}[2]{\ensuremath{{#1}^{\bar{#2}}}}
\newcommand{\map}[3]{\ensuremath{{#1}: {#2} \rightarrow {#3}}}
\newcommand{\pra}[3]{{#1}{\ast}({#2}) = {#3}}

\title{Algeberska topologija\\ Zapiski predavanj}
\date{2023/24}
%===============================================================================
\begin{document}
	\maketitle
	\newpage
	\begin{abstract}
		\noindent Dokument vsebuje zapiske predavanj predmeta Algeberska topologija v okviru študija prvega letnika magistrskega študija matematike na FNM.
	\end{abstract}
	\tableofcontents
	\newpage
	\section{Uvodna motivacija}
	Tekom matematične izobrazbe se spoznamo z mnogimi t.~i.~strukturami, ki tipično zavzamejo obliko ">množica $+$ nekaj"<. Med njimi imamo tipično tudi preslikave, ki jim pogosto damo posebno ime. Naštejmo nekaj primerov.
	\begin{table}[h!]
		\centering
		\begin{tabular}{|l|l|l|}
			Ime & Oznaka & ime preslikav \\\hline
			Množice & $M$ & preslikave oz. funkcije \\\hline
			Grupe & $(G, \circ)$ & homomorfizmi grup \\\hline
			Abelove grupe & $(G, +)$ & homomorfizmi Ab.~grup \\\hline
			Polja & $(F, +, \cdot)$ & homomorfizmi polj \\\hline
			Vektorski prostori nad poljem $F$ & $(V, +, \cdot)$ & linearne preslikave \\\hline
			Delno urejene množice & $(P, \leq)$ & naraščajoče funkcije \\\hline
			Linearno urejene množice & $(L, \leq)$ & naraščajoče funkcije \\\hline
			Metrični prostori & $(X, d)$ & zvezne funkcije \\\hline
			Topološki prostori & $(X, \mathcal{T})$ & zvezne funkcije \\\hline
		\end{tabular}
	\end{table}
	Prej omenjene preslikave (na neki strukturi) lahko seveda tudi komponiramo. Za komponiranje velja, da obstajata leva in desna enota ter da je asociativno za preslikave, ki se ustrezno ujemajo z domenami in kodomenami (npr. za preslikave $f$, $g$ in $h$ mora, če želimo formirati $h\circ (g \circ f)$ veljati, da je kodomena $f$ hkrati domena $g$ ter da je kodomena $g$ hkrati domena $h$.)
	Posplošena obravnava lastnosti skupin določenih struktur nas privede do t.~.i~ teorije kategorij.
	
	\section{Kategorije}
	\begin{definicija}
		Razred $\mathcal{C}$ z delno binarno operacijo $\circ$ je \pojem{kategorija}, če velja: \begin{itemize}
			\item $\mathcal{C}$ je unija disjunktnih razredov $\Ob{\mathcal{C}}$ in $\Mor{\mathcal{C}}$. Elementom $\Ob{\mathcal{C}}$ pravimo \pojem{objekti}, elementom $\Mor{\mathcal{C}}$ pa \pojem{morfizmi}.
			\item Za vsak $f\in \Mor{\mathcal{C}}$ sta enolično določena ">začetek"< in ">konec"<, ki sta oba objekta kategorije $\mathcal{C}$. Pišemo $\map{f}{X}{Y}$.
			\item Za poljubna objekta $X, Y \in \Ob{\mathcal{C}}$ je $\Mor[\Cc]{(X, Y)} = \{f\in\Mor{\Cc}; \map{f}{X}{Y}\}$ množica (ne samo razred).
			\item Za poljubna morfizma $\map{f}{X}{Y}$ in $\map{g}{Y}{Z}$ je enolično definiran morfizem $\map{g\circ f}{X}{Z}$ in velja:
			\begin{enumerate}
				\item Za poljubne morfizme $\map{f}{X}{Y}, \map{g}{Y}{Z}$ in $\map{h}{Z}{W}$ je $(h\circ g)\circ f = h\circ (g\circ f)$.
				\item Za vsak $X\in \Ob{\Cc}$ obstaja enolično določen morfizem $1_X \in \Mor[\Cc]{(X, X)}$, z lastnostjo: $\forall \map{f}{X}{Y} \land \forall \map{g}{Z}{X}$ je $f\circ 1_X = f$~in~$1_X \circ g = g$ (Za poljubna $Y, Z\in\Ob{\Cc}$).
			\end{enumerate}
		\end{itemize}
	\end{definicija}
	\begin{zgled}
		\begin{itemize}
			\item Naj bo $(G, \circ)$ poljubna grupa in $\Ob{\Cc} = \{\ast\}, \Mor{\Cc} = G$. $\circ$ v $\Cc$ je kar $\circ$ v $G$. Očitno je (ker je to edina možnost), da vsak morfizem $a\in \Mor{\C}$ slika objekt $\ast$ v samega vase. Dodatno: $e = 1_\ast$.
			\item  Naj bo $(L, \leq)$ poljubna linearno urejena množica in naj bo $\Ob{\Cc} = L$. Morfizme določimo na naslednji način: $\forall x, y\in L: \abs{\Mor[\Cc]{(x, y)}} = \begin{cases}
			1;&~ x\leq y \\
			0;&~ \neg(x\leq y)
			\end{cases}$
			Tranzitivnost linearne urejenosti nam zagotovi enoličnost kompozituma in posledično tudi njegovo asociativnost.
		\end{itemize}
	\end{zgled}
	\begin{definicija}
		Pravimo, da je kategorija $\Cc$ \pojem{majhna}, če je $\Ob{\Cc}$ množica.
	\end{definicija}
	
	\begin{izr}
		V majhni kategoriji je $\Mor{\Cc}$ množica (in posledično tudi $\Ob{\Cc}\cup\Mor{\Cc}$)
	\end{izr}
	\begin{proof}
		Velja: $$\Mor{\Cc} = \bigcup_{(X, Y)\in \Ob{\Cc}\times\Ob{\Cc}}\Mor[\Cc]{(X, Y)}$$
		Unija družine množic je tudi sama množica.
	\end{proof}
	\begin{zgled}
		Naj bo $\Ob{\Cc} = \N$ in $F$ izbrano polje. Naj bo $\Mor{\Cc} = \bigcup_{(n, m)\in \N^2} M_{n\times m}(F) =$ množica vseh matrik z elementi iz $F$. Dodatno, omenimo, da je $M_{n\times m}(F) = \Mor[\Cc]{(m, n)}$. Za $\circ$ vzamemo množenje matrik.
	\end{zgled}
	\section{Funktorji}
	\begin{definicija}
		Naj bosta $\Cc$ in $\Dd$ poljubni kategoriji. $\map{F}{\Cc}{\Dd}$ je \pojem{kovariantni funktor}, če: \begin{itemize}
			\item $F$ slika $\Ob{\Cc}$ v $\Ob{\Dd}$ in $\Mor{\Cc}$ v $\Mor{\Dd}$.
			\item $\forall X, Y \in \Ob{\Cc},~\forall\map{f}{X}{Y}$ je $\map{F(f)}{F(X)}{F(Y)}$
			\item $\forall f, g\in \Mor{\Cc}:$ če $\exists g\circ f$, potem je $F(g\circ f) = F(g)\circ F(f)$
			\item $\forall X\in \Ob{\Cc}: F(1_X) = 1_{F(X)}$
		\end{itemize}
		
		Pravimo, da je $\map{F}{\Cc}{\Dd}$ je \pojem{kontravariantni funktor}, če: \begin{itemize}
			\item $F$ slika $\Ob{\Cc}$ v $\Ob{\Dd}$ in $\Mor{\Cc}$ v $\Mor{\Dd}$.
			\item $\forall X, Y \in \Ob{\Cc},~\forall\map{f}{X}{Y}$ je $\map{F(f)}{F(Y)}{F(X)}$
			\item $\forall f, g\in \Mor{\Cc}:$ če $\exists g\circ f$, potem je $F(g\circ f) = F(f)\circ F(g)$
			\item $\forall X\in \Ob{\Cc}: F(1_X) = 1_{F(X)}$
		\end{itemize}
	\end{definicija}
	
	\begin{definicija}
		Naj bo $\Cc$ poljubna kategorija in $\map{f}{X}{Y}$ element $\Mor{\Cc}$. Pravimo, da je $f$ \pojem{izomorfizem}, če $\exists \map{g}{Y}{X}$, da je $g\circ f = 1_X$ in $f\circ g = 1_Y$.
	\end{definicija}
	
	\begin{izr}
		Naj bo $\map{F}{\Cc}{\Dd}$ poljubni funktor poljubnih kategorij $\Cc$ in $\Dd$. Naj bosta $X, Y \in \Ob{\Cc}$ in $\map{f}{X}{Y}$ izomorfizem v $\Cc$. Potem je $F(f)$ izomorfizem v $\Dd$.
	\end{izr}
	\begin{proof}
		Denimo, da je $F$ kovariantni funktor. Naj bo $\map{g}{Y}{X}$ takšen, da je $g\circ f = 1_X$ in $f\circ g = 1_Y$. Potem je $1_{F(X)} = F(1_X) = F(g\circ f) = F(g)\circ F(f)$ in $1_{F(Y)} = F(1_Y) = F(f\circ g) = F(f)\circ F(g)$, torej je $F(f)$ izomorfizem. Še več, če označimo $g = f^{-1}$ je $F(f^{-1}) = (F(f))^{-1}$.
		Če je $F$ kontravariantni funktor poteka dokaz na enak način, zato ga opustimo.
	\end{proof}
	\begin{definicija}
		V poljubni kategoriji $\Cc$ za poljubna objekta $X, Y\in \Ob{\Cc}$ definiramo relacijo: \[
		X \approx Y \iff \exists\map{f}{X}{Y},~\textit{ki je izomorfizem}\]
	\end{definicija}
	\begin{izr}
		Relacija $\approx$ je ekvivalenčna relacija na $\Ob{\Cc}$.
	\end{izr}
	\begin{proof}
		Preverimo, ali $\approx$ zadošča vsem pogojem za ekvivalenčne relacije. \begin{itemize}
			\item[refleksivnost:] Vemo, da $\forall X\in \Ob{\Cc}$ obstaja morfizem $\map{1_X}{X}{X}$, ki je izomorfizem.
			\item[simetričnost:] Naj bosta $X, Y\in \Ob{\Cc}$ poljubna objekta in $\map{f}{X}{Y}$ izomorfizem. Po definiciji izomorfizma potem obstaja morfizem $\map{f^{-1}}{Y}{X}$, ki je tudi sam izomorfizem. Če je $X\approx Y$ je potem tudi $Y\approx X$
			\item[tranzitivnost:] Naj bodo $X, Y, Z\in \Ob{\Cc}$ za katere velja $X\approx Y$ in $Y\approx Z$. Potem obstajata izomorfizma $\map{f}{X}{Y}$ in $\map{g}{Y}{Z}$ in trdimo, da je tudi $\map{g\circ f}{X}{Z}$ je izomorfizem. Vidimo namreč, da za $f^{-1} \circ g^{-1}$ velja $(g\circ f)\circ (f^{-1} \circ g^{-1})= g\circ f\circ f^{-1} \circ g^{-1} = g\circ 1_X \circ g^{-1} = g\circ g^{-1} = 1_Y$. Podobno vidimo tudi, da velja $(f^{-1} \circ g^{-1})\circ (g\circ f) = 1_X$. Sledi, da je $(g\circ f)$ izomorfizem in $(g\circ f)^{-1} = f^{-1} \circ g^{-1}$. Posledično je $\approx$ tranzitivna.
		\end{itemize}
	\end{proof}
	\begin{posl}
		Naj bo $\map{F}{\Cc}{\Dd}$ poljuben funktor (kovariantni ali kontravariantni) poljubnih kategorij $\Cc$ in $\Dd$.
		Potem za poljubna objekta $X, Y\in \Ob{\Cc}$ velja: \begin{itemize}
			\item $X\approx Y \Rightarrow F(X) \approx F(Y)$
			\item $F(X) \not\approx F(Y) \Rightarrow X\not\approx Y$
		\end{itemize}
	\end{posl}
	
	Omenimo še nekaj oznak za znane kategorije.
	\begin{table}[h!]
		\centering
		\begin{tabular}{|l|l|}
			\hline
			Oznaka & kategorija \\\hline
			$\mathcal{S}et$ & množice \\\hline
			$\mathcal{G}r$ & grupe \\\hline
			$\mathcal{A}b$ & Ab. grupe \\\hline
			$\mathcal{M}et$ & metrični prostori \\\hline
			$\mathcal{V}ect_F$ & vektorski prostori nad poljem $F$\\\hline
			$\mathcal{T}op$ & topološki prostori \\\hline
		\end{tabular}
	\end{table}
	\begin{zgled}
		Definiramo t.~i.~ pozabljivi funktor, ki slika iz neke kategorije struktur v kategorijo množic. Na primer funktor $\map{F}{\mathcal{G}r}{\mathcal{S}et}; F((G, \circ)) = G$. Vidimo tudi, da $F$ slika homomorfizme grup v preslikave (istih množic). Funktor $F$ je kovariantni. Če tukaj uporabimo prejšnjo posledico opazimo naslednje: Za $X, Y \in \mathcal{S}et: X\approx Y \iff \abs{X} = \abs{Y}$. Drugače povedano, dve grupi različnih moči ne moreta biti izomorfni.
	\end{zgled}
	
	\begin{zgled}
		Pokažimo, da $(\R, \mathcal{T}_{evk}) \not\approx (\R^2, \acute{\mathcal{T}}_{evk})$.
		Denimo, da obstaja nek homeomorfizem $f$ med $\R$ in $\R^2$. Potem bo za $\forall a\in \R$ tudi $f_{|\R\setminus\{a\}}$ homeomorfizem iz $\R\setminus\{a\}$ v $\R^2\setminus\{(a, 0)\}$. Ti množici pa imata različno število povezanih komponent: $C(\R\setminus\{a\}) = 2, C(\R^2\setminus\{(a, 0)\}) = 1$. Prišli smo v protislovje, saj homeomorfizmi ohranjajo število povezanih komponent.
		Zgoraj povedano lahko izrazimo tudi s funktorji. Naj bo $\map{F}{\mathcal{T}op}{\mathcal{S}et}$ in $F((X, \mathcal{T}))= C((X, \mathcal{T}))$. Potem velja $\abs{C((X, \mathcal{T}))} \neq \abs{C((Y, \acute{\mathcal{T}}))} \Rightarrow (X, \mathcal{T})\not\approx (Y, \acute{\mathcal{T}})$
	\end{zgled}
\end{document}