\documentclass[a4paper, 10pt]{article}
\usepackage[T1]{fontenc}
\usepackage[utf8]{inputenc}
\usepackage[slovene]{babel}
\usepackage{lmodern}
\usepackage{amsmath}
\usepackage{leftidx}
\usepackage{amssymb}
\usepackage{amsfonts}
\usepackage{graphicx}
\usepackage{wrapfig}
\usepackage{amsthm}
\usepackage{mathrsfs}
\usepackage{mathtools}
\usepackage{url}
\usepackage{subfigure}
\usepackage{multirow}
\usepackage{lipsum}
\usepackage{wrapfig}
\usepackage{tikz}
\usepackage[format=plain, font=small, labelfont=bf, textfont=it, justification=centerlast]{caption}
\usepackage{booktabs}
\usepackage{siunitx}
\usepackage{enumerate}

\newtheorem{trditev}{Trditev}
\newtheorem{izr}{Izrek}
\newtheorem{nal}{Naloga}

\newcounter{defcount}
\newcounter{opombe}
\newcounter{zgledcount}

\newenvironment{opomba}{\begin{flushleft}\stepcounter{opombe}\textbf{Opomba \arabic{opombe}:}}{\hfill\end{flushleft}}
\setlength{\parindent}{0mm}

\newenvironment{zgled}{\begin{flushleft}\stepcounter{zgledcount}\textbf{Zgled \arabic{zgledcount}:}}{\hfill\end{flushleft}}
\setlength{\parindent}{0mm}

\newenvironment{definicija}{\begin{flushleft}\stepcounter{defcount}\textbf{Definicija \arabic{defcount}:}}{\hfill\end{flushleft}}
\setlength{\parindent}{0mm}

\newenvironment{resitev}{\begin{flushleft}\textit{Rešitev:}}{\hfill\end{flushleft}}
\setlength{\parindent}{0mm}

\newcommand{\abs}[1]{\ensuremath{\lvert #1 \rvert}}
\newcommand{\mth}[1]{\ensuremath{\mathbb{#1}}}
\newcommand{\R}{\mth{R}}
\newcommand{\Z}{\mth{Z}}
\newcommand{\N}{\mth{N}}
\newcommand{\No}{\mth{N}_0}
\newcommand{\C}{\mth{C}}

\newcommand{\Qu}{\mth{Q}_u}

\newcommand{\pojem}[1]{\emph{#1}}
\newcommand{\con}{\ensuremath{\mathscr{C}}}
\newcommand{\padex}[2]{\ensuremath{{#1}^{\underline{#2}}}}
\newcommand{\rastx}[2]{\ensuremath{{#1}^{\bar{#2}}}}
\newcommand{\map}[3]{\ensuremath{{#1}: {#2} \rightarrow {#3}}}
\newcommand{\pra}[3]{{#1}{\ast}({#2}) = {#3}}

\title{Teorija mere\\ Zapiski predavanj}
\date{2023/24}
%===============================================================================
\begin{document}
	\maketitle
	\newpage
	\begin{abstract}
		\noindent Dokument vsebuje zapiske predavanj predmeta Teorija mere v okviru študija prvega letnika magistrskega študija matematike na FNM.
	\end{abstract}
	\newpage
	\tableofcontents
	\newpage
	\section{Uvodna motivacija}
	Za motivacijo bomo obravnavali en primer, pred tem pa bomo na hitro povzeli definicijo Riemannovega integrala. Naj bo \map{f}{[a, b]}{\R} realna, zvezna in omejena funkcija ter naj bo $D=\{x_0, x_1, \ldots, x_n\}$ delitev intervala $[a, b]$. Označimo $\Delta x_i = x_i - x_{i-1}$ ter z vsakega intervala $[x_{i-1}, x_i]$ izberemo neko poljubno točko $\acute{x}_i$. Vsoto $\sigma_n = \sum_{i = 1}^{n}f(\acute{x}_i)\Delta x_i$ imenujemo Riemannova vsota. Če obstaja limita $\lim_{\abs{\Delta x_i}\to 0}\sigma_n$ in je neodvisna od izbire delitve $D$ in testnih točk na podintervalih, ki jih določa $D$, ji pravimo Riemannov integral funkcije $f$ na $[a, b]$. Riemannov integral lahko posplošimo za računanje integralov funkcij večih spremenljivk, pri tem pa uporabljamo t.~i. Jordanovo mero. Motivacijski primer bo pokazal, da ima konstrukcija s to mero nekatere pomanjkljivosti.
	
	\begin{zgled}
		Naj bo $R = \mth{Q}\cap \R = \{r_1, r_2, \ldots\}$ in definiramo funkcije \map{f_k}{[0, 1]}{[0, 1]} s predpisi \[f_1(x) = \begin{cases}
			1;& x = r_1 \\
			0;& x\neq r_1
		\end{cases}, ~f_2(x) = \begin{cases}
		1;& x = r_1 \lor x = r_2 \\
		0;& x\in R\setminus\{r_1, r_2\}
		\end{cases}\] itd. Vidimo, da zaporedje $\{f_k\}$ konvergira k Dirichletovi funkciji $f_D(x) = \begin{cases}
		1;& x\in \mth{Q} \\
		0;& x\notin \mth{Q}
	\end{cases}$ in dodatno opazimo, da je $\int_{0}^{1}f_k(x)dx = 0 \forall k$, daj gre limita Riemannovih vrst za vsako funkcijo zaporedja proti $0$. To pa ne drži za Dirichletovo funkcijo. Če vse $\acute{x}_i$ pripadajo $\mth{Q}$, bo $\sigma_n = 1$, če so izbrane točke $\acute{x}_i$ iracionalne, pa je $\sigma_n = 0$. Limita Riemannovih vsot torej ni neodvisna od izbire delitve in testnih točk, torej integral $f_D$ ne obstaja.
	\end{zgled}
	
		Francoski matematik Lebesgue se je pa problema lotil drugače: Najprej razdelimo zalogo vrednosti omejene zvezne funkcije $f$ z delitvijo $\{y_0, y_1, \ldots, y_n\}$ in sestavimo množice $E_k = \{x \in [a, b]; f(x) y_k\}$. Prepoznamo, da so $E_k \times \{y_k\}$ vodoravne daljice od $(f^{-1}(y_k), y_k)$ do $(b, y_k)$. Posledično z $\abs{E_k}$ označimo dolžino daljice $E_k$. Potem je ploščina pod grafom funkcije $f$ približno enaka vsoti $\sum_{k = 1}^{n}(y_k - y_{k-1}) \abs{E_k}$. Izkaže pa se, da so lahko v splošnem množice $E_k$ takšne, da koncept dolžine in prostornine za obravnavo več ne zadošča. Zato uvedemo koncept mere.
		
		\section{Kolobar množic}
		Začnimo ta odsek z definicijo.
		\begin{definicija}
			Naj bo $X$ poljubna neprazna množica. Množica $K$ podmnožic množice $X$ je \pojem{kolobar}, če: \begin{itemize}
				\item $\forall A, B\in K: A\cup B \in K$
				\item $\forall A, B \in K: A\setminus B \in K$
			\end{itemize}
		\end{definicija}
		\begin{trditev}
			Če je $K$ kolobar množic na $X$ velja: \begin{enumerate}
				\item $\forall A, B \in K: A \triangle B \in K$
				\item $\forall A, B \in K: A \cap B \in K$
				\item $\emptyset\in K$
			\end{enumerate}
		\end{trditev}
		\begin{proof}
			\begin{enumerate}
				\item Ta trditev sledi neposredno iz definicije kolobarja množice
				\item Upoštevamo, da lahko zapišemo $A\cap B = (A\cup B)\setminus (A\triangle B)$ in potem ta trditev sledi po prejšnji.
				\item Upoštevamo, da je $\emptyset = a\setminus A$.
			\end{enumerate}
		\end{proof}
		
		\begin{definicija}
			\begin{enumerate}
				\item Pravimo, da je množica $E$ \pojem{enota kolobarja} $K$, če za $\forall A\in K$ velja $A\cap E = A$.
				\item Kolobarju z enoto pravimo \pojem{algebra}
			\end{enumerate}
		\end{definicija}
		
		\begin{zgled}
			\begin{enumerate}
				\item Če je $K$ kolobar nad $X$ in je $X\in K$, potem je $X$ enota kolobarja $K$.
				\item Naj bo $K$ kolobar vseh končnih podmnožic iz $\R$. Kolobar $K$ nima enote. To lahko vidimo tako, da predpostavimo, da obstaja enota $E$ in vzamemo dve množici, $A\in K$ z močjo $n$ in $B\in K$ z močjo $n+1$ za nek poljuben $n\in \N$. Ker je, po definiciji enote, $ A\cap E = A$, sledi, da je $A\subseteq E$, torej je $\abs{E} \geq n$. Po drugi strani, ker je $B\cap E = B$, sklepamo, da je $\abs{E}\geq n+1$. To velja za poljuben $n$, torej moč $E$ presega kardinalnost vsake množice iz $K$, torej je $E$ neskončna. To nas pa privede v protislovje s tem, da $E\in K$.
			\end{enumerate}
		\end{zgled}
		
		\begin{definicija}
			Naj bo $\{A_k\}_{k=1}^\infty$ poljubno zaporedje množic. 
			 \begin{itemize}
			 	\item Pravimo, da je množica $\overline{A}$ \pojem{zgornja limita} zaporedja $\{A_k\}_{k=1}^\infty$, če za $\forall x\in \overline{A} \exists \{k_i\}_{i = 1}^\infty: x\in A_{k_i} \forall i\in \N$.
			 	\item Pravimo, da je $\underline{A}$ \pojem{spodnja limita} zaporedja $\{A_k\}_{k=1}^\infty$, če za $\forall x \in \underline{A} \exists k_0\in \N: x\in A_k \forall k\geq k_0$
			 \end{itemize}
			 Pišemo: $\overline{A} = \overline{\lim_{n\to \infty}}A_n$ in $\underline{A} = \underline{\lim_{n\to \infty}}A_n$
		\end{definicija}
		
\end{document}