\documentclass[a4paper, 10pt]{article}
\usepackage[T1]{fontenc}
\usepackage[utf8]{inputenc}
\usepackage[slovene]{babel}
\usepackage{lmodern}
\usepackage{amsmath}
\usepackage{leftidx}
\usepackage{amssymb}
\usepackage{amsfonts}
\usepackage{graphicx}
\usepackage{wrapfig}
\usepackage{amsthm}
\usepackage{mathrsfs}
\usepackage{mathtools}
\usepackage{url}
\usepackage{subfigure}
\usepackage{multirow}
\usepackage{lipsum}
\usepackage{wrapfig}
\usepackage{tikz}
\usepackage[format=plain, font=small, labelfont=bf, textfont=it, justification=centerlast]{caption}
\usepackage{booktabs}
\usepackage{siunitx}
\usepackage{enumerate}
\usepackage{ulem}

\newtheorem{trditev}{Trditev}
\newtheorem{izr}{Izrek}
\newtheorem{nal}{Naloga}
\newtheorem{posl}{Posledica}

\newcounter{defcount}
\newcounter{opombe}
\newcounter{zgledcount}

\newenvironment{opomba}{\begin{flushleft}\stepcounter{opombe}\textbf{Opomba \arabic{opombe}:}}{\hfill\end{flushleft}}
\setlength{\parindent}{0mm}

\newenvironment{zgled}{\begin{flushleft}\stepcounter{zgledcount}\textbf{Zgled \arabic{zgledcount}:}}{\hfill\end{flushleft}}
\setlength{\parindent}{0mm}

\newenvironment{definicija}{\begin{flushleft}\stepcounter{defcount}\textbf{Definicija \arabic{defcount}:}}{\hfill\end{flushleft}}
\setlength{\parindent}{0mm}

\newenvironment{resitev}{\begin{flushleft}\textit{Rešitev:}}{\hfill\end{flushleft}}
\setlength{\parindent}{0mm}

\newcommand{\abs}[1]{\ensuremath{\lvert #1 \rvert}}
\newcommand{\mth}[1]{\ensuremath{\mathbb{#1}}}
\newcommand{\R}{\mth{R}}
\newcommand{\Z}{\mth{Z}}
\newcommand{\N}{\mth{N}}
\newcommand{\No}{\mth{N}_0}
\newcommand{\C}{\mth{C}}
\newcommand{\Mu}{\mathcal{M}}
\newcommand{\sigalg}{$\sigma$-algebra~}
\newcommand{\Qu}{\mth{Q}_u}

\newcommand{\pojem}[1]{\emph{#1}}
\newcommand{\con}{\ensuremath{\mathscr{C}}}
\newcommand{\padex}[2]{\ensuremath{{#1}^{\underline{#2}}}}
\newcommand{\rastx}[2]{\ensuremath{{#1}^{\bar{#2}}}}
\newcommand{\map}[3]{\ensuremath{{#1}: {#2} \rightarrow {#3}}}
\newcommand{\pra}[3]{{#1}{\ast}({#2}) = {#3}}

\title{Teorija mere\\ Zapiski predavanj}
\date{2023/24}
%===============================================================================
\begin{document}
	\maketitle
	\newpage
	\begin{abstract}
		\noindent Dokument vsebuje zapiske predavanj predmeta Teorija mere v okviru študija prvega letnika magistrskega študija matematike na FNM.
	\end{abstract}
	\newpage
	\tableofcontents
	\newpage
	\section{Uvodna motivacija}
	Za motivacijo bomo obravnavali en primer, pred tem pa bomo na hitro povzeli definicijo Riemannovega integrala. Naj bo \map{f}{[a, b]}{\R} realna, zvezna in omejena funkcija ter naj bo $D=\{x_0, x_1, \ldots, x_n\}$ delitev intervala $[a, b]$. Označimo $\Delta x_i = x_i - x_{i-1}$ ter z vsakega intervala $[x_{i-1}, x_i]$ izberemo neko poljubno točko $\acute{x}_i$. Vsoto $\sigma_n = \sum_{i = 1}^{n}f(\acute{x}_i)\Delta x_i$ imenujemo Riemannova vsota. Če obstaja limita $\lim_{\abs{\Delta x_i}\to 0}\sigma_n$ in je neodvisna od izbire delitve $D$ in testnih točk na podintervalih, ki jih določa $D$, ji pravimo Riemannov integral funkcije $f$ na $[a, b]$. Riemannov integral lahko posplošimo za računanje integralov funkcij večih spremenljivk, pri tem pa uporabljamo t.~i. Jordanovo mero. Motivacijski primer bo pokazal, da ima konstrukcija s to mero nekatere pomanjkljivosti.
	
	\begin{zgled}
		Naj bo $R = \mth{Q}\cap \R = \{r_1, r_2, \ldots\}$ in definiramo funkcije \map{f_k}{[0, 1]}{[0, 1]} s predpisi \[f_1(x) = \begin{cases}
			1;& x = r_1 \\
			0;& x\neq r_1
		\end{cases}, ~f_2(x) = \begin{cases}
		1;& x = r_1 \lor x = r_2 \\
		0;& x\in R\setminus\{r_1, r_2\}
		\end{cases}\] itd. Vidimo, da zaporedje $\{f_k\}$ konvergira k Dirichletovi funkciji $f_D(x) = \begin{cases}
		1;& x\in \mth{Q} \\
		0;& x\notin \mth{Q}
	\end{cases}$ in dodatno opazimo, da je $\int_{0}^{1}f_k(x)dx = 0 \forall k$, daj gre limita Riemannovih vrst za vsako funkcijo zaporedja proti $0$. To pa ne drži za Dirichletovo funkcijo. Če vse $\acute{x}_i$ pripadajo $\mth{Q}$, bo $\sigma_n = 1$, če so izbrane točke $\acute{x}_i$ iracionalne, pa je $\sigma_n = 0$. Limita Riemannovih vsot torej ni neodvisna od izbire delitve in testnih točk, torej integral $f_D$ ne obstaja.
	\end{zgled}
	
		Francoski matematik Lebesgue se je pa problema lotil drugače: Najprej razdelimo zalogo vrednosti omejene zvezne funkcije $f$ z delitvijo $\{y_0, y_1, \ldots, y_n\}$ in sestavimo množice $E_k = \{x \in [a, b]; f(x) y_k\}$. Prepoznamo, da so $E_k \times \{y_k\}$ vodoravne daljice od $(f^{-1}(y_k), y_k)$ do $(b, y_k)$. Posledično z $\abs{E_k}$ označimo dolžino daljice $E_k$. Potem je ploščina pod grafom funkcije $f$ približno enaka vsoti $\sum_{k = 1}^{n}(y_k - y_{k-1}) \abs{E_k}$. Izkaže pa se, da so lahko v splošnem množice $E_k$ takšne, da koncept dolžine in prostornine za obravnavo več ne zadošča. Zato uvedemo koncept mere.
		\newpage
		\section{Kolobar množic}
		Začnimo ta odsek z definicijo.
		\begin{definicija}
			Naj bo $X$ poljubna neprazna množica. Množica $K$ podmnožic množice $X$ je \pojem{kolobar}, če: \begin{itemize}
				\item $\forall A, B\in K: A\cup B \in K$
				\item $\forall A, B \in K: A\setminus B \in K$
			\end{itemize}
		\end{definicija}
		\begin{trditev}
			Če je $K$ kolobar množic na $X$ velja: \begin{enumerate}
				\item $\forall A, B \in K: A \triangle B \in K$
				\item $\forall A, B \in K: A \cap B \in K$
				\item $\emptyset\in K$
			\end{enumerate}
		\end{trditev}
		\begin{proof}
			\begin{enumerate}
				\item Ta trditev sledi neposredno iz definicije kolobarja množice
				\item Upoštevamo, da lahko zapišemo $A\cap B = (A\cup B)\setminus (A\triangle B)$ in potem ta trditev sledi po prejšnji.
				\item Upoštevamo, da je $\emptyset = a\setminus A$.
			\end{enumerate}
		\end{proof}
		
		\begin{definicija}
			\begin{enumerate}
				\item Pravimo, da je množica $E$ \pojem{enota kolobarja} $K$, če za $\forall A\in K$ velja $A\cap E = A$.
				\item Kolobarju z enoto pravimo \pojem{algebra}.
			\end{enumerate}
		\end{definicija}
		
		\begin{zgled}
			\begin{enumerate}
				\item Če je $K$ kolobar nad $X$ in je $X\in K$, potem je $X$ enota kolobarja $K$.
				\item Naj bo $K$ kolobar vseh končnih podmnožic iz $\R$. Kolobar $K$ nima enote. To lahko vidimo tako, da predpostavimo, da obstaja enota $E\in K$. Po definiciji $K$ obstaja neko število $n_0\in \N$, da je $\abs{E} = n_0$. Sedaj vzamemo množico, $A\in K$ z močjo $n > n_0$. Ker je, po definiciji enote, $ A\cap E = A$, sledi, da je $A\subseteq E$, torej je $\abs{E} \geq n$. Sledi, da je $\abs{E} > n_0$, kar nas pa privede v protislovje.
			\end{enumerate}
		\end{zgled}
		
		\begin{definicija}
			Naj bo $\{A_k\}_{k=1}^\infty$ poljubno zaporedje množic. 
			 \begin{itemize}
			 	\item Pravimo, da je množica $\overline{A}$ \pojem{zgornja limita} zaporedja $\{A_k\}_{k=1}^\infty$, če za $\forall x\in \overline{A} \exists \{k_i\}_{i = 1}^\infty: x\in A_{k_i} \forall i\in \N$.
			 	\item Pravimo, da je $\underline{A}$ \pojem{spodnja limita} zaporedja $\{A_k\}_{k=1}^\infty$, če za $\forall x \in \underline{A}~\exists k_0\in \N: x\in A_k \forall k\geq k_0$
			 \end{itemize}
			 Pišemo: $\overline{A} = \overline{\lim_{n\to \infty}}A_n$ in $\underline{A} = \underline{\lim_{n\to \infty}}A_n$
		\end{definicija}
		\begin{zgled}
			Poglejmo si zaporedje množic $a_n = \begin{cases}
			[0, 1 + \frac{1}{n}];&~ n~\text{je sodo} \\
			[1 - \frac{1}{n}, 2];&~ n~\text{je liho}
			\end{cases}$. Vidimo, da je $\overline{A} = [0, 2]$ in $\underline{A} = \{1\}$.
		\end{zgled}
		
		\begin{opomba}
			Opazimo, da velja tudi $\underline{A} \subseteq \overline{A}$. Da se prepričamo, da je to res, vzamemo poljubno zaporedje množic $\{A_n\}_{n = 1}^\infty$ in nek poljuben $x\in \underline{A}$. Po definiciji te množice potem $\exists n_0\in\N$, da je $x\in A_n$ za vse $n\geq n_0$. Izmed teh $n\geq n_0$ izberemo poljubno neskončno podzaporedje $\{l_k\}_{k = 1}^\infty$ in potem velja $x\in A_{l_k} \forall k\in \N$. Potem je pa $x\in \overline{A}$.
		\end{opomba}
		\begin{trditev}
			\label{trd:karlimmn}
			Za vsako zaporedje množic $\{A_n\}_{n = 1}^\infty$ velja: \begin{enumerate}[a)]
				\item $\overline{A} = \bigcap_{k = 1}^\infty \bigcup_{m = k}^\infty A_m$
				\item $\underline{A} = \bigcup_{k = 1}^\infty \bigcap_{m = k}^\infty A_m$
			\end{enumerate}
		\end{trditev}
		\begin{proof}
			\begin{enumerate}[a)]
				\item Najprej bomo pokazali inkluzijo v desno, nato pa še v levo: \begin{itemize}
					\item[$\subseteq):$] Naj bo $x\in \overline{A}$. Potem obstaja neko zaporedje indeksov $\{k_i\}_{i = 1}^\infty$, da je $x\in A_{k_i} \forall i\in\N$. Potem bo pa $x\in \bigcup_{m = k}^\infty A_m$ za vsak $k\in\N$. Posledično, je $x\in \bigcap_{k = 1}^\infty \bigcup_{m = k}^\infty A_m$.
					\item[$\supseteq):$] Denimo, da je $x\in \bigcap_{k = 1}^\infty \bigcup_{m = k}^\infty A_m$ in izberimo poljuben indeks $k_1$. Potem je $x\in \bigcup_{m = k_1}^\infty A_m$, kar pa pomeni, da obstaja nek $l_1 \geq k_1$, da je $x\in A_{l_1}$. Sedaj izberemo nov indeks $k_2 \geq l_1$ in ponovimo prejšnji postopek ter tako pridobimo $l_2$. Postopek nadaljujemo in tako tvorimo zaporedje indeksov $\{l_i\}_{i =1}^\infty$, za katerega velja $x\in A_{l_i} \forall i\in\N$, torej je $x\in \overline{A}$.
				\end{itemize}
				\item Podobno kot pri prejšnji točki, bomo najprej bomo pokazali inkluzijo v desno, nato pa še v levo: \begin{itemize}
					\item[$\subseteq):$] Naj bo $x\in \underline{A}$. Potem obstaja nek indeks $k_0$, da je $x\in A_{k} \forall k \geq k_0$. Potem je pa $x\in \bigcap_{m = k_0}^\infty A_m$, torej je tudi $x\in \bigcup_{k = 1}^\infty \bigcap_{m = k}^\infty A_m$.
					\item[$\supseteq):$] Denimo, da je $x\in \bigcup_{k = 1}^\infty \bigcap_{m = k}^\infty A_m$. Potem obstaja indeks $k_0$, da je $x\in \bigcap_{m = k_0}^\infty A_m$, kar pa pomeni, da je $x\in A_{k} \forall k \geq k_0$, torej je $x\in \underline{A}$.
				\end{itemize}
			\end{enumerate}
		\end{proof}
		
		\begin{definicija}
			Pravimo, da je zaporedje množic $\{A_k\}_{k = 1}^\infty$ \pojem{monotono}, če velja: \begin{itemize}
				\item $A_1 \subseteq A_2 \subseteq \ldots$ (naraščajoče)
				\item $A_1 \supseteq A_2 \supseteq \ldots$ (padajoče)
			\end{itemize}
		\end{definicija}
		\begin{definicija}
			Pravimo, da zaporedje množic $\{A_k\}_{k = 1}^\infty$ \pojem{konvergira} proti množici $A$, če je $\underline{A} = A = \overline{A}$.
		\end{definicija}
		
		\begin{trditev}
			Vsako monotono zaporedje $\{A_k\}_{k = 1}^\infty$ konvergira, pri čemer velja: \begin{enumerate}[a)]
				\item Če je $\{A_k\}_{k = 1}^\infty$ naraščajoče, je $\lim_{k\to\infty}A_k = A = \bigcup_{k = 1}^\infty A_k$
				\item Če je $\{A_k\}_{k = 1}^\infty$ padajoče, je $\lim_{k\to\infty}A_k = A = \bigcap_{k = 1}^\infty A_k$
			\end{enumerate}
		\end{trditev}
		\begin{proof}
			\begin{enumerate}[a)]
				\item Naj bo $\{A_k\}_{k = 1}^\infty$ naraščajoče zaporedje množic in preverimo, da je $\underline{A} = \overline{A}$. Pri tem bomo uporabili trditev \ref{trd:karlimmn}. Po eni strani vemo, da je $\overline{A} = \bigcap_{k = 1}^\infty \bigcup_{m = k}^\infty A_m$. Ko upoštevamo, da je $\{A_k\}_{k = 1}^\infty$ naraščajoče, vidimo, da je $\overline{A} = \bigcap_{k = 1}^\infty \bigcup_{m = k}^\infty A_m = \bigcap_{k = 1}^\infty \bigcup_{m = 1}^\infty A_m = \bigcup_{m = 1}^\infty A_m$. Zadnja enakost sledi iz tega, da so množice v preseku neodvisne od indeksov, po katerih delamo presek (v $A_m$ ne nastopa indeks $k$). Ker $\{A_k\}_{k = 1}^\infty$ narašča, velja tudi $\bigcup_{m = k}^\infty A_m = A_k$, torej je $\underline{A} = \bigcup_{k = 1}^\infty \bigcap_{m = k}^\infty A_m = \bigcup_{k = 1}^\infty A_k$. Torej je $\underline{A} = \overline{A}$.
				\item Pokažemo na podoben način.
			\end{enumerate}
		\end{proof}
		
		\begin{definicija}
			Naj bo $X$ poljubna množica. Množica $\Mu$ podmnožic $X$ je \pojem{\sigalg} na $X$, če velja: \begin{enumerate}
				\item $\emptyset, X \in \Mu$
				\item $\forall A \in \Mu: A^c \in \Mu$
				\item $\forall \{A_k\}_{k=1}^{\infty}; A_k \in \Mu~ \forall i\in \N: \bigcup_{k = 1}^{\infty} A_k \in \Mu$
			\end{enumerate}
		\end{definicija}
		
		\begin{zgled}
			\begin{enumerate}
				\item Naj bo $X$ neka množica in $\Mu = \{\emptyset, X\}$. Vidimo, da je $\Mu$ \sigalg.
				\item Naj bo $X = \{a, b, c\}$. Ali je $\mathcal{P}(X)$ \sigalg? Da. V resnici je $\mathcal{P}(X)$ \sigalg za poljubno množico $X$.
				\item Naj bo $X$ poljubna množica in $\Mu = \{A \subseteq X; A~\text{je kvečjemu števna ali pa je}~A^c ~\text{kvečjemu števna}\}$. Tudi ta $\Mu$ je \sigalg. To bomo tudi na hitro premislili. \begin{itemize}
					\item Očitno $\Mu$ vsebuje $\emptyset$ in $X$. 
					\item Denimo sedaj, da je $A\in\Mu$. Če je $A$ kvečjemu števna, potem je $A^c \in \Mu$, saj je $A = (A^c)^c$ kvečjemu števna. Če pa $A$ ni kvečjemu števna avtomatsko sledi, da je $A^c$ kvečjemu števna, torej je $A^c\in \Mu$.
					\item Naj bo $\{A_k\}_{k = 1}^\infty$ nek nabor množic, pri čemer je $A_k \in \Mu~\forall k\in\N$. Če so vse množice $A_k$ kvečjemu števne, je tudi $\bigcup_{k = 1}^\infty A_k$ kvečjemu števna in torej pripada $\Mu$. Denimo torej, da $\exists k_0\in\N$, da $A_{k_0}$ ni kvečjemu števna, je pa $A_{k_0}^c$. V tem primeru pogledamo $\left(\bigcup_{k = 1}^\infty A_k\right)^c = \bigcap_{k = 1}^\infty A_k^c$. Ker vemo, da je $A_{k_0}$ kvečjemu števna, je potem tudi presek $\bigcap_{k = 1}^\infty A_k^c$ kvečjemu števna množica, torej je $\bigcap_{k = 1}^\infty A_k^c \in \Mu$ in posledično je $\bigcup_{k = 1}^\infty A_k \in \Mu$.
				\end{itemize}
			\end{enumerate}
		\end{zgled}
		\begin{trditev}
			Naj bo $\Mu$ \sigalg na $X$. Potem velja: \begin{enumerate}[a)]
				\item $\forall A, B\in \Mu: A\cup B \in \Mu$
				\item $\forall \{A_k\}_{k=1}^\infty; A_K\in\Mu~\forall i\in\N: \bigcap_{k = 1}^\infty A_k \in \Mu$
				\item $\forall A, B\in \Mu: A\cap B \in \Mu$
				\item $\forall A, B\in \Mu: A\setminus B \in \Mu$
			\end{enumerate}
		\end{trditev}
		\begin{proof}
			\begin{enumerate}[a)]
				\item Naj bodo $A_1 = A, A_2 = B$ in $A_k = \emptyset~\forall k \geq 3$. Potem je $A\cup B = \bigcup_{k = 1}^\infty A_k \in\Mu$ po definiciji $\sigma-$algeber.
				\item Uporabimo, da je \sigalg po definiciji zaprta za komplimente. Torej, za vsak $A_k \in \Mu$ je tudi $A_k^c \in \Mu$. Potem pa sestavimo zaporedje $\{A_k^c\}_{k=1}^\infty$ in po definiciji $\sigma$-algebre je potem $\bigcup_{k = 1}^\infty A_k^c \in \Mu$ ter posledično še $\left(\bigcup_{k = 1}^\infty A_k^c\right)^c = \bigcap_{k = 1}^\infty (A_k^c)^c = \bigcap_{k = 1}^\infty A_k^c \in \Mu$.
				\item Vzamemo $A_1 = A, A_2 = B$ in $A_k = X~\forall k\geq 3$. Potem je $A\cap B = \bigcap_{k = 1}^\infty A_k \in \Mu$, po prejšnji točki.
				\item $A\setminus B = A\cap B^c \in \Mu$ po prejšnji trditvi, ker sta $A, B^c \in \Mu$.
			\end{enumerate}
		\end{proof}
		\begin{posl}
			\sigalg je kolobar množic.
		\end{posl}
		\begin{posl}
			\sigalg je algebra množic.
		\end{posl}
		
		\newpage
		
		\section{Mera}
		\begin{definicija}
			Naj bo $K$ kolobar množic na $X$. Funkcija \map{m}{K}{[0, +\infty)} je \pojem{mera} na $X$, če za njo velja: 
			\[ \forall A, B \in K; A\cap B = \emptyset: m(A\cup B) = m(A) + m(B)\]
		\end{definicija}
		\begin{trditev}
			Za poljubno mero $m$ na kolobarju $K$ podmnožic množice $X$ velja:
			\begin{enumerate}
				\item $m(\emptyset) = 0$
				\item $A\subseteq B \Rightarrow m(B) = m(A) + m(B\setminus A)$
				\item $A\subseteq B \Rightarrow m(A) \leq m(B)$
				\item $m(A\cup B) = m(A) + m(B) - m(A\cap B)$
				\item Naj bodo $A_1, A_2, \ldots A_n \in K$ in $A_i \cap A_j =\emptyset$ za $i\neq j$. Potem je $m(\bigcup_{k = 1}^n A_k) = \sum_{k = 1}^n m(A_k)$.
				\item Naj bodo $A_1, A_2, \ldots A_n \in K$. Potem je $m(\bigcup_{k = 1}^n A_k) \leq \sum_{k = 1}^n m(A_k)$.
				\item Naj bodo $A, A_k \in K~\forall k\in\N$ in $A_i\cap A_j =\emptyset$ za $i\neq j$. Dodatno, naj velja $\bigcup_{k = 1}^n A_k \subseteq A$. Potem je $\sum_{k = 1}^\infty m(A_k) \leq m(A)$.
			\end{enumerate}
		\end{trditev}
		\begin{proof}
			\begin{enumerate}
				\item $m(\emptyset) = m(\emptyset\cup\emptyset) = m(\emptyset) + m(\emptyset)$, torej je $2m(\emptyset) = m(\emptyset)$. To je možno le, ko je $m(\emptyset) = 0$.
				\item Denimo, da je $A\subseteq B$. Potem je $B = A\cup(B\setminus A)$ in $A \cap (B\setminus A) = \emptyset$. Po definiciji mere je potem $m(B) = m(A\cup(B\setminus A)) = m(A) + m(B\setminus A)$.
				\item Sledi po prejšnji točki.
				\item Pišemo: $A\cup B = A \cup (B\setminus (A\cap B))$. Ker je $A \cap (B\setminus (A\cap B)) = \emptyset$ je $m(A\cup B) = m(A) + m(B\setminus (A\cap B))$. Ker je $A\cap B \subseteq B$ lahko uporabimo formulo iz druge točke: $m(B) = m(A\cap B) + m(B\setminus (A\cap B))$ oz. $m(B\setminus (A\cap B)) = m(B) - m(A\cap B)$. Sledi, da je $m(A\cup B) = m(A) + m(B) - m(A\cap B)$.
				\item Točko bomo dokazali z indukcijo po številu množic. Za $ = 2$ že vemo, saj to velja po definiciji mere. Denimo torej, da velja trditev za nek $k\in\N$ in dokažimo za $n = k+1$. Denimo, da imamo paroma disjunktne množice $A_1, A_2, \ldots, A_k, A_{k+1}$. Označimo $B = \bigcup_{i = 1}^k A_i$. Potem je $\bigcup_{i = 1}^{k+1} A_i = B\cup A_{k+1}$ in $B\cap A_{k+1} = \emptyset$. Sledi, da je $m(B\cup A_{k+1}) = m(B) + m(A_{k+1}) = \sum_{i = 1}^k m(A_i) + m(A_{k+1}) = \sum_{i = 1}^{k+1}m(A_i)$.
				\item Za primer $n = 2$ že vemo, saj je to direktna posledica četrte trditve. Denimo torej, da trditev velja za nek $k\in \N$ in dokažimo, da potem velja tudi za $k+1$: Ponovno označimo $B = \bigcup_{i = 1}^{k}A_i$ in potem je $\bigcup_{i = 1}^{k+1}A_i = B\cup A_{k+1}$. Po četrti točki sklepamo, da je $m(B\cup A_{k+1}) \leq m(B) + m(A_{k+1})$. Dodatno, upoštevamo indukcijsko predpostavko, da je $m(B) = m(\bigcup_{i = 1}^{k}A_i)\leq \sum_{i = 1}^{k}m(A_i)$. Potem je pa $m(\bigcup_{i = 1}^{k+1}A_i) \leq \sum_{i = 1}^{k+1}m(A_i)$
				\item Naj bodo $A, A_k\in K~\forall k\in\N$ in $A_i\cap A_j = \emptyset$ za $i\neq j$. Naj bo $\bigcup_{k = 1}^n A_k \subseteq A$. Potem bo za $\forall l\in\N$ tudi $\bigcap_{k = 1}^l A_k \subseteq A$ in po tretji točki potem velja $m(\bigcup_{k = 1}^l A_k) \leq m(A)$. Ker so množice $A_k$ paroma disjunktne, po peti točki sledi $m(\bigcup_{k = 1}^l A_k) = \sum_{k = 1}^l m(A_k)$. Ker to velja za $\forall l\in \N$, je $\sum_{k = 1}^\infty m(A_k) = \lim_{l\to\infty}\sum_{k = 1}^l m(A_k) \leq m(A)$.
			\end{enumerate}
		\end{proof}
		
		\begin{definicija}
			Naj bo $m$ mera na kolobarju množic $K$. Pravimo, da je mera $m$: \begin{itemize}
				\item \pojem{$\sigma$-aditivna} na K, če velja sklep: $A =\bigcup_{i = 1}^\infty A_i; A, A_i\in K~\forall i\in\N \land A_i\cap A_j =\emptyset ~\text{,če je}~i\neq j \Rightarrow m(A) = \sum_{i = 1}^\infty m(A_i)$
				\item \pojem{$\sigma$-poladitivna} na K, če velja sklep: $A \subseteq \bigcup_{i = 1}^\infty A_i; A, A_i\in K~\forall i\in\N \Rightarrow m(A) \leq \sum_{i = 1}^\infty m(A_i)$
				\item \pojem{zvezna} na K, če za vsako monotono zaporedje množic $\{A_i\}_{i=1}^\infty; A_i\in K$ z limito $\lim_{i\to\infty}A_i = A\in K$ velja: $\lim_{i\to\infty}m(A_i) = m(A)$
			\end{itemize}
		\end{definicija}
		\begin{izr}
			Naj bo $K$ kolobar množic (na neki množici $X$) in $m$ poljubna mera na njem. Lastnost $\sigma$-aditivnosti, $\sigma$-poladitivnosti in zveznosti so ekvivalentne.
		\end{izr}
		\begin{proof}
			Najprej bomo dokazali ekvivalenco $\sigma$-aditivnosti in $\sigma$-poladitivnosti, nato pa bomo dokazali ekvivalenco $\sigma$-aditivnosti in zveznosti.
			\begin{enumerate}
				\item Dokazujemo, da je mera $m$ je $\sigma$-aditivna $\iff$ mera $m$ je $\sigma$-poladitivna:\begin{itemize}
					\item[$\Rightarrow):$] Naj bodo $A, A_i \in K~\forall i\in \N$ in naj bo $A\subseteq \bigcup_{i = 1}^\infty A_i$. Označimo $B_1 = A\cap A_1, B_2 = (A\cap A_2)\setminus B_1, \ldots, B_k = (A\cap A_k)\setminus (\bigcup_{i = 1}^{k-1}B_i)$ in pokažimo, da velja \dashuline{$B_i\cap B_j =\emptyset$ za $i\neq j$}. Brez škode za splošnost predpostavimo, da je $i<j$ in denimo, da imamo nek $x\in B_i\cap B_j$. Po definiciji preseka sledi $x\in B_i~\&~ x\in B_j$. Po drugi strani, pa je $B_j = (A\cap A_j)\setminus (\bigcup_{k = 1}^{j-1}B_k)$. Ker je $i<j$ je $B_i \subseteq \bigcup_{k = 1}^{j-1}B_k$. Ker je $x\in B_i$ potem sledi, da $x\notin B_j$. Prišli smo v protislovje, torej je res $B_i\cap B_j = \emptyset$ za $i\neq j$.
					Sedaj bomo pokazali, da je \dashuline{$A = \bigcup_{k = 1}^\infty B_k$}. \begin{itemize}
						\item[$\subseteq):$] Naj bo $x\in A$. Ker je $A \subseteq \bigcup_{i = 1}^\infty A_i$ obstaja neko število $k_0 \in \N$ in neko zaporedje indeksov $\{k_i\}_{i = 1}^\infty; k_i > k_0 ~\forall i\in\N$, da je $x_\in A_{k_1}, x\in A_{k_2}, \ldots $. Potem posledično velja $x\notin B_k$ za $k< k_0$, od tod pa sledi, da je $x\in B_{k_0} = (A\cap A_{k_0})\setminus (\bigcup_{k = 1}^{k_0 - 1}B_k)$. Ker smo za poljubni $x\in A$ našli neko množico $B_k$, da je $x\in B_k$, potem očitno velja $A\subseteq \bigcup_{k = 1}^\infty B_k$.
						\item[$\supseteq):$] Naj bo sedaj $x\in \bigcup_{k = 1}^\infty B_k$ in naj bo $k_0$ najmanjši indeks, taki, da je $x\in B_{k_0} = (A\cap A_{k_0})\setminus (\bigcup_{k = 1}^{k_0 -1} B_k)$. Iz tega, da je izbrani $k_0$ najmanjši izmed vseh, ki zadoščajo prejšnjemu pogoju, sledi $x\notin \bigcup_{k = 1}^{k_0 -1} B_k$, torej je $x\in A\cap A_{k_0}$. Sledi, da je $x\in A$ Ker to velja za vsak $x\in \bigcup_{k = 1}^\infty B_k$ sledi $\bigcup_{k = 1}^\infty B_k \subseteq A$
					\end{itemize}
					Pokazali smo torej, da je $A = \bigcup_{k = 1}^\infty B_k$. Sedaj lahko uporabimo $\sigma$-aditivnost: \[m(A = m(A = \bigcup_{k = 1}^\infty B_k)) = \sum_{k = 1}^\infty m(B_k)\]
					Ker je $m(B_k) \leq m(A_k)~\forall k\in\N$ je potem $m(A) \leq \sum_{k = 1}^\infty m(A_k)$ in s tem je pokazana $\sigma$-poladitivnost.
					\item[$\Leftarrow:)$] Naj bo sedaj $A = \bigcup_{k = 1}^\infty A_k$; $A, A_k\in K~\forall k\in\N$, in $A_i\cap A_j =\emptyset$ za $i\neq j$. Ker je $A \subseteq \bigcup_{k = 1}^\infty A_k$, je po $\sigma$-poladitivnosti $m(A) \leq \sum_{k = 1}^\infty m(A_k)$. Po drugi strani je pa $A \supseteq \bigcup_{k = 1}^\infty A_k$ in posledično je $m(A) \geq \sum_{k = 1}^\infty m(A_k)$. Sledi, da je $m(A) = \sum_{k = 1}^\infty m(A_k)$, torej je $m$ $\sigma$-aditivna.
				\end{itemize}
				\item Dokazujemo, da je mera $m$ $\sigma$-aditivna $\iff$ mera $m$ je zvezna: \begin{itemize}
					\item[$\Rightarrow):$] Denimo, da imamo monotono zaporedje množic $\{A_k\}_{k = 1}^\infty$ z limito $A = \lim_{k\to\infty}A_k$. \begin{itemize}
						\item Če $\{A_k\}_{k = 1}^\infty$ narašča je $A = \bigcup_{k = 1}^\infty A_k$. Določimo $A_0 = \emptyset$ in opazimo, da je, za $i \leq j$, $m(A_i)\leq m(A_j)$, ker je $A_i \subseteq A_j$. Označimo $B_1 = A_1\setminus A_0, B_2 = A_2 \setminus A_1, \ldots B_k = A_k \setminus A_{k-1}$ in vidimo, da je $B_i\cap B_j = \emptyset$ za $i\neq j$, ter da je $A = \bigcup_{k = 1}^\infty B_k$. Potem je \begin{align*}
							m(A) &= m(\bigcup_{k = 1}^\infty B_k) = \sum_{k = 1}^\infty m(B_k) = \lim_{n\to\infty}\sum_{k = 1}^n m(B_k) \\
							&= \lim_{n\to\infty}(m(B_1) + m(B_2) + \ldots + m(B_n)) \\ 
							&= \lim_{n\to\infty}(m(A_1\setminus A_0) + m(A_2\setminus A_1) + \ldots + m(A_n\setminus A_{n-1})) \\
							&= \lim_{n\to\infty}(m(A_1) - m(A_0) + m(A_2) - m(A_1) + \ldots + m(A_n) - m(A_{n-1})) \\
							&= \lim_{n\to\infty}(m(\emptyset) + m(A_n)) = \lim_{n\to\infty}m(A_n)
						\end{align*}
						\item Če $\{A_k\}_{k = 1}^\infty$ pada je $A = \bigcap_{k = 1}^\infty A_k$. Označimo $B_1 = A_1 \setminus A_2, B_2 = A_1 \setminus A_3, \ldots, B_k = A_1 \setminus A_{k+1}$. Opazimo, da je zaporedje $\{B_k\}_{k = 1}^\infty$ naraščajoče, torej je $\lim_{k\to\infty}B_k = \bigcup_{k = 1}^\infty B_k = \bigcup_{k = 1}^\infty (A_1 \setminus A_{k+1}) = A_1 \setminus \bigcap_{k = 2}^\infty A_{k}$. Ko upoštevamo, da $\{A_k\}_{k=1}^\infty$ pada, vidimo, da je $\bigcap_{k = 2}^\infty A_{k} = \bigcap_{k = 1}^\infty A_{k} = A$, torej je $\lim_{k\to\infty} B_k = A_1\setminus A$. Sledi, da je \underline{$m(\lim_{k\to\infty}B_k) = m(A_1 \setminus A) = m(A_1) - m(A)$}. Ker zaporedje $\{B_k\}_{k = 1}^\infty$ narašča, se pa lahko skličemo na prejšnjo točko in vidimo, da je \begin{align*}
							m(\lim_{k\to\infty}B_k) &= \lim_{k\to\infty}m(B_k) = \lim_{k\to\infty}m(A_1 \setminus A_{k+1}) \\
							&= \lim_{k\to\infty}(m(A_1) - m(A_{k+1})) \\
							&= m(A_1) - \lim_{k\to\infty}m(A_{k+1})
						\end{align*}
						Posledično je $m(A) = \lim_{k\to\infty}m(A_k)$, torej je $m$ zvezna.
					\end{itemize}
					\item[$\Leftarrow):$] Denimo, da je $m$ zvezna mera in naj bo $A=\bigcup_{k = 1}^\infty A_k$, kjer so $A, A_k \in K~\forall k\in\N$ in $A_i\cap A_j = \emptyset$ za $i\neq j$. Naj bo $B_k = \bigcup_{i = 1}^{k}A_i$. Vidimo, da je zaporedje $\{B_k\}_{k=1}^\infty$ naraščajoče in potem je $\lim_{k\to\infty}B_k = \bigcup_{i = 1}^\infty A_i = A$ in posledično sklepamo, da je \begin{align*} m(A) &= m(\lim_{k\to\infty}B_k) = \lim_{k\to\infty}m(B_k) = \lim_{k\to\infty}m(\bigcup_{i = 1}^k A_i) \\
						&= \lim_{k\to\infty}\sum_{i = 1}^k m(A_i) = \sum_{i = 1}^\infty m(A_i)
					\end{align*}
					Torej je $m$ $\sigma$-aditivna.
				\end{itemize}
			\end{enumerate}
		\end{proof}
		
		\begin{trditev}
			Naj bo $\{\Mu_{\alpha}\}_{\alpha\in I}$ družina $\sigma$-algeber na poljubno množico $X$. Tedaj je $\Mu = \bigcap_{\alpha\in I}\Mu_\alpha$ tudi \sigalg na $X$.
		\end{trditev}
		\begin{proof}Preverimo, da $\Mu$ ustreza vsem aksiomom $\sigma$-algebre.
			\begin{itemize}
				\item Ker sta $\emptyset, X \in \Mu_\alpha ~\forall \alpha\in I$, očitno velja $\emptyset, X \in \Mu = \bigcap_{\alpha\in I}\Mu_\alpha$.
				\item Naj bo $A\in \Mu$. Potem je $A\in \Mu_\alpha ~\forall \alpha\in I$, torej je $A^c \in \Mu_\alpha~\forall \alpha\in I$, torej je $A^c \in \bigcap_{\alpha\in I}\Mu_\alpha = \Mu$.
				\item Naj bo $\{A_i\}_{i = 1}^\infty$ zaporedje množic vsebovano v $\Mu$. Potem je to zaporedje vsebovano tudi v $\Mu_\alpha ~\forall\alpha\in I$. Od tod sledi, da je $\bigcup_{i = 1}^\infty A_i \in \Mu_\alpha~\forall \alpha\in I$, torej je $\bigcup_{i = 1}^\infty A_i \in \Mu$.
			\end{itemize}
		\end{proof}
		\begin{definicija}
			Naj bo $\mth{A}$ družina podmnožic $X$. Najmanjšo $\sigma$-algebro, ki vsebuje $\mth{A}$, označimo z $\Mu_{\mth{A}}$ in jo imenujemo \pojem{$\sigma$-algebra generirana z \mth{A}}.
		\end{definicija}
		\begin{trditev}
			Za $\mathcal{A}$ obstaja najmanjša $\sigma$-algebra, ki vsebuje $\mathcal{A}$.
		\end{trditev}
		\begin{proof}
			Vemo, da obstaja vsaj ena $\sigma$-algebra, ki vsebuje $\mathcal{A}$. Naj bo $\{\Mu_\alpha\}_{\alpha\in I}$ družina vseh $\sigma$-algeber, ki vsebujejo $\mathcal{A}$. Tedaj je $\Mu_{\mathcal{A}} = \bigcap_{\alpha\in I}\Mu_\alpha$ najmanjša taka $\sigma$-algebra, ki vsebuje $\mathcal{A}$.
		\end{proof}
		\begin{zgled}
			Vzemimo $X = \R$ in $\mathcal{A} = \{\{1\}, \{2\}\}$. Potem je $\Mu_{\mathcal{A}} = \{\emptyset, \{1\}, \{2\}, \{1, 2\}, \R\setminus\{1\}, \R\setminus\{2\}, \R\setminus\{1, 2\}, \R\}$
		\end{zgled}
		\begin{definicija}
			Naj bo $(X, d)$ metrični prostor. Najmanjšo $\sigma$-algebro, ki vsebuje vse odprte množice $(X, d)$ imenujemo \pojem{Borelova $\sigma$-algebra} in jo označimo z $\mathcal{B}_X$. Elemente Borelove $\sigma$-algebre imenujemo \pojem{Borelove množice}.
		\end{definicija}
		\begin{zgled}
			\begin{itemize}
				\item Odprte, zaprte množice
				\item Končne množice
				\item polzaprti intervali $[a, b) = \bigcap_{n = 1}^\infty (a - \frac{1}{n}, b)$
			\end{itemize}
		\end{zgled}
\end{document}