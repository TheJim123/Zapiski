\documentclass[a4paper, 10pt]{article}
\usepackage[T1]{fontenc}
\usepackage[utf8]{inputenc}
\usepackage[slovene]{babel}
\usepackage{lmodern}
\usepackage{amsmath}
\usepackage{leftidx}
\usepackage{amssymb}
\usepackage{amsfonts}
\usepackage{graphicx}
\usepackage{wrapfig}
\usepackage{amsthm}
\usepackage{mathrsfs}
\usepackage{mathtools}
\usepackage{url}
\usepackage{subfigure}
\usepackage{multirow}
\usepackage{lipsum}
\usepackage{wrapfig}
\usepackage{tikz}
\usepackage[format=plain, font=small, labelfont=bf, textfont=it, justification=centerlast]{caption}
\usepackage{booktabs}
\usepackage{siunitx}

\newtheorem{trditev}{Trditev}
\newtheorem{izr}{Izrek}

\newcounter{defcount}
\newcounter{opombe}
\newcounter{zgledcount}

\newenvironment{opomba}{\begin{flushleft}\stepcounter{opombe}\textbf{Opomba \arabic{opombe}:}}{\hfill\end{flushleft}}
\setlength{\parindent}{0mm}

\newenvironment{zgled}{\begin{flushleft}\stepcounter{zgledcount}\textbf{Zgled \arabic{zgledcount}:}}{\hfill\end{flushleft}}
\setlength{\parindent}{0mm}

\newenvironment{definicija}{\begin{flushleft}\stepcounter{defcount}\textbf{Definicija \arabic{defcount}:}}{\hfill\end{flushleft}}
\setlength{\parindent}{0mm}

\newcommand{\abs}[1]{\ensuremath{\lvert #1 \rvert}}
\newcommand{\mth}[1]{\ensuremath{\mathbb{#1}}}
\newcommand{\R}{\mth{R}}
\newcommand{\Z}{\mth{Z}}
\newcommand{\N}{\mth{N}}
\newcommand{\No}{\mth{N}_0}
\newcommand{\C}{\mth{C}}
\newcommand{\Qu}{\mth{Q}_u}
\newcommand{\pojem}[1]{\emph{#1}}
\newcommand{\con}{\ensuremath{\mathscr{C}}}
\newcommand{\padex}[2]{\ensuremath{{#1}^{\underline{#2}}}}
\newcommand{\rastx}[2]{\ensuremath{{#1}^{\bar{#2}}}}
\newcommand{\map}[3]{\ensuremath{{#1}: {#2} \rightarrow {#3}}}
\newcommand{\pra}[3]{{#1}{\ast}({#2}) = {#3}}

\title{Diskretna matematika 1 Episode II: Attack of the binomials}
\date{19.2.2019}
\author{Nek študent FMF}
%===============================================================================
\begin{document}
\maketitle
\newpage

\begin{abstract}
\noindent Ta kreacija, ki jo trenutno berete, dragi bralec, je skupek prepisanih zapiskov študijskega predmeta Diskretna matematika 1 s perspektive nekega študenta 2. letnika matematike na FMF.
Uporaba tega dokumenta za kakršnikoli namene je na lastno odgovornost.
\end{abstract}

\newpage

\section{Uvod}
Vprašajmo se najprej, kaj sploh pomeni pridevnik >>diskreten<<. Ta beseda ima v slovenščini dva pomena, ki sta v angleščini ločeni besedi: \textit{discreet} - obziren in \textit{discrete} - jasno ločen oz. razločljiv. Za naše potrebe je (cca. očitno, dokaz prepuščen bralcu) relevanten drugi pomen. Diskretna matematika se torej ukvarja s (včasih tudi števnimi) končnimi strukturami, ki so ločljive (ne tega citirat, razen če želite v indeksu zbrati tri $6$-ke in priklicati hudiča). Zanimivost: Obe besedi izhajata iz latinskega glagola \textit{discernere}, iz katerega še najbolj očitno izhaja angleški glagol >>\textit{to discern}<<, ki ima med drugimi tudi pomen >>razločiti<<.

\begin{opomba}
Preden začnemo, si oglejmo še kak praktičen primer diskretnosti:

Množica $A$ v topološkem prostoru $X$ je diskretna, če $\forall x\in A ~\exists \mathcal{O}_x$ okolica $x$ tako, da je $x$ edina točka iz A v $\mathcal{O}_x$.
\end{opomba}
Poglejmo si še kakšen zgled
\begin{zgled}
X = \mth{R}, A = \mth{Z} 
\end{zgled}

\subsection{Področja diskretne matematike:}
\begin{itemize}

\item Kombinatorika,
\item Teorija grafov,
\item Matematična logika,
\item Teorija množic (osnove),
\item Računalniška matematika,
\item Teorija kodiranja,
\item itd.

\end{itemize}
\subsection{Vsebina predmeta diskretna matematika 1:}
\begin{enumerate}

\item Kombinatorika
\begin{itemize}

\item Osnove preštevanja
\item Načelo vključitev in izključitev
\item Rekurzivne enačbe in rodovne vrste

\end{itemize}
\item Teorija grafov
\begin{itemize}

\item Osnovno o grafih
\item Drevesa in cikli
\item Ravninski grafi
\item Barvanje grafov
\item Povezanost grafov

\end{itemize}
\end{enumerate}

\subsection{Literatura:}
\begin{itemize}


\item Primož Potočnik: Zapiski predavanj iz Diskretne Matematike 1, dostopno na povezavi: \url{https://www.fmf.uni-lj.si/~potocnik/Ucbeniki/DM-Zapiski2010.pdf}

\item M. Juvan in P. Potočnik: Teorija grafov in kombinatorika: Primeri in rešene naloge
\end{itemize}

\section{Osnove Preštevanja}
\begin{enumerate}

\item \underline{Oznake}:
\begin{itemize}

\item $\mth{N}_0 = \{0, 1, 2, \ldots \} = \omega$
\item $\mth{N} = \{1, 2, 3, \ldots \}$
\item $\abs{A} =$ št. elementov oz. moč množice A (gledamo predvsem končne množice) 

\end{itemize}

\item \underline{Načelo Enakosti}:
\[
\abs{A} = \abs{B} \iff \exists f : A \rightarrow B \text{, ki je bijektivna}
\]
\item \underline{Načelo vsote}:
\[
A \cap B = \emptyset \Rightarrow \abs{A \cup B} = \abs{A} + \abs{B}
\]
Induktivno velja:
\[
A_i \cap A_j = \emptyset; ~i\neq j \Rightarrow \abs{\bigcup^n_{i = 1}{A_i}} = \sum^n_{i = 1}{\abs{A_i}}
\]
\item \underline{Načelo produkta}:
\[
\abs{A\times B} = \abs{A} \cdot \abs{B}
\]
Analogno zapišemo formulo za poljuben produkt:
\[
\abs{\prod^n_{i = 1}{A_i}} = \prod^n_{i = 1}{\abs{A_i}}
\]
\item \underline{Računovodsko pravilo}:
Naj bo $A\in \mth{C}^{m \times n}$. Tedaj velja:
\[ \sum^m_{i = 1}{\underbrace{\sum^n_{j = 1}{a_{ij}}}_{vrstična vsota}} = \sum^n_{j = 1}{\underbrace{\sum^m_{i = 1}{a_{ij}}}_{stolpčna vsota}} \]
Povedano še z besedami: Kadar seštevamo vrednosti polj v neki tabeli, je vseeno, če najprej seštejemo po vrstah in nato seštejemo vsote vrst, ali pa najprej seštejemo po stolpcih in nato seštejemo vsote stolpcev.
\end{enumerate}

\begin{definicija}
Naj bo $x \in \mth{C}$ in $n\in \mth{N}_0$.
\begin{enumerate}

\item $x^n = \underbrace{x\cdot x\cdot\ldots x}_{n} = \prod^{n}_{i = 0}{x}$ je \pojem{$n$-ta potenca} števila $x$.
\item $x^{\underline{n}} = \underbrace{x\cdot (x - 1)\cdot\ldots (x - (n - 1))}_{n} = \prod^{n - 1}_{i = 0}{(x - i)}$ je \pojem{$n$-ta padajoča potenca} števila $x$.
\item $x^{\bar{n}} = \underbrace{x\cdot (x + 1)\cdot\ldots (x + (n - 1))}_{n} = \prod^{n - 1}_{i = 0}{(x + i)}$ je \pojem{$n$-ta rastoča potenca} števila $x$.

\end{enumerate}

\end{definicija}

Hitro opazimo, da velja:
\[
x^{\underline{n}} = (x - (n - 1))^{\bar{n}} \text{~in~} x^{\bar{n}} = (x + (n - 1))^{\underline{n}}
\]
Posledično velja:
\[
1^{\bar{n}} = 1 \cdot 2 \cdot \ldots \cdot n = n! = n^{\underline{n}}
\]

\begin{opomba}
Vrednost praznega produkta je $1$. Npr.:
\[
x^0 = x^{\underline{0}} = x^{\bar{0}} = 0! = 1
\]
V analizi izraz $0^0$ ni definiran, v diskretni matematiki pa ga enačimo z $1$.
\end{opomba}

\begin{definicija}

Naj bo $x \in \mth{C}$ in $k \in \mth{Z}$.
\[
\binom{x}{k} = \begin{cases}
 \frac{x^{\underline{k}}}{k!};& k \geq 0 \\
 0;& k < 0
\end{cases}
\text{~,se imenuje binomski koeficient.}
\]
\end{definicija}
Naslednje trditve ne bomo dokazali (dokaz prepuščen bralcu za vajo).
\begin{trditev}
Naj bo $\abs{A} = n$ in $\abs{B} = k$.
\begin{enumerate}

\item Število vseh podmnožic množice $A$ je $2^n$.
\item Število podmnožic moči $k$ množice A je $\binom{n}{k}$.
\item Število vseh preslikav $A \rightarrow B$ je $k^n$.
\item število injektivnih preslikav $A \rightarrow B$ je $k^{\underline{n}}$.
\item število bijektivnih preslikav $A \rightarrow B$ je $\begin{cases}
n!; & k = n \\
0 ; & k \neq n
\end{cases}$.
Posledično je število permutacij množice $A$ (torej število bijekcij $A \rightarrow A$) enako $n!$
\end{enumerate}
\end{trditev}

\subsection{Razdelitve množice:}

\begin{definicija}
\pojem{Razdelitev/particija} množice $A$ je družina množic $\{ B_1, B_2,  \ldots, B_k\}$ za katero velja:

\begin{enumerate}

\item $\forall i \in \{1, 2, \ldots, k\}: B_i \neq \emptyset$,
\item $\forall i, j \in \{1, 2, \ldots, k\}: (i \neq j \Rightarrow B_i \cap B_j = \emptyset)$,
\item $\bigcup^k_{i = 1}{B_i} = A$

\end{enumerate}
Če imamo podano razdelitev množice, lahko z njeno pomočjo definiramo tako ekvivalenčno relacijo, da porojeni ekvivalenčni razredi sovpadajo s particijo. Takšna ekvivalenčna relacija se (v splošnem) glasi: $a \sim b \iff \exists i \in \{1, 2, \ldots, k \}; a, b \in B_i$
\end{definicija}

\begin{definicija}
Naj bosta $k, n \in \mth{N}_0$. Število razdelitev množice z $n$ elementi na $k$ blokov imenujemo \pojem{Stirlingovo število druge vrste} in ga označimo s $S(n, k)$.
\end{definicija}

\begin{zgled}
$n = 4, k = 2; S(4, 2) = ?$
\[
A = \{ a, b, c, d \}; \abs{A} = 4 \text{(ta pogoj zagotovi, da so si a, b, c in d medseboj različni)}
\]
Poglejmo si vse razdelitve množice $A$ na $2$ bloka:
\begin{tabular}{cccc}
$a, bcd$ & $b, acd$ & $c, abd$ & $d, abc$ \\
$ab, cd$ & $ac, bd$ & $ad, bc$ & \ 
\end{tabular}

Skupaj je to 7 različnih možnosti. Torej $S(4, 2) = 7$.

V splošnem tudi velja: $S(n, 2) = 2^{n - 1} - 1; n \geq 1$
\end{zgled}

\begin{trditev}
Število surjekcij $A \rightarrow B; \abs{A} = n \land \abs{B} = k$ je enako $S(n, k) \cdot k!$ za $n \geq k$ in $0$ sicer.
\end{trditev}

\begin{proof}
Opazimo: praslike elementov množice $B$ tvorijo particijo  v $A$, kjer velja $x \sim y \iff f(x) = f(y)$

\begin{enumerate}

\item Najprej $A$ razdelimo na k nepraznih blokov na S(n, k) načinov
\item Za vsak blok izberemo slike v $B$ na $k!$ načinov

\end{enumerate}
Torej če je $n \geq k$ je število surjekcij enako $S(n, k) \cdot k!$.
Če je $n < k$ potem med $A$ in $B$ ne obstaja nobena surjekcija.
\end{proof}

\begin{izr}
Za $n, k \geq 1$ velja:
\[
S(n, k) = S(n - 1, k - 1) + k \cdot S(n - 1, k)
\]
\end{izr}

\begin{proof}
Naj bo $\abs{A} = n$ in naj bo $a \in A$ izbrana točka.
Naj bo $f: A \rightarrow A\setminus \{a\}$.

Ločimo $2$ primera:

\begin{enumerate}

\item Razdelitve, v katerih je izbrani $a$ sam v svojem bloku:

Dobljena slika množice $A$ je torej množica, ki ima en element manj in hkrati tudi en blok manj. Njeno Stirlingovo število je torej $S(n - 1, k - 1)$. Opazimo tudi (baje), da je v tem primeru $f$ bijekcija, saj $A\setminus \{a\}$ >>dodamo še en blok, ki vsebuje $a$, pa je vredu.<<

\item Razdelitve, v katerih izbrani $a$ ni sam v svojem bloku:

V tem primeru se zmanjša samo moč množice $A$ (število blokov ostane enako), torej je Stirlingovo število enako $S(n - 1, k)$.
Ker imamo $k$ blokov imamo $k$ možnosti, kam vtakniti ta $a$. Torej je takih razdelitev $k \cdot S(n - 1, k)$.
 
\end{enumerate}
Opciji $1$ in $2$ sta (očitno) disjunktni. Po načelu vsote je torej $S(n, k) = S(n - 1, k - 1) + k \cdot S(n - 1, k)$
\end{proof}
Zdaj, ko smo dokazali prejšnjo rekurzivno zvezo, moramo za njo izračunati začetne pogoje:
\begin{itemize}

\item $S(0,0) = 1$
\item $S(n,0) = 0 \text{~,če je} n > 1 $
\item $S(n, n) = 1$
\item $S(n, 1) = 1; n \geq 1$
\item $S(n, k) = 0 \text{~,če je} n < k$

\end{itemize}

\begin{definicija}
Število vseh razdelitev množice z $n$ elementi imenujemo \pojem{Bellovo število} in ga označimo z $B_n$
\end{definicija}
Tabelirajmo $S(n, k)$:
\begin{table}[!h]
\centering
\begin{tabular}{l | cccccc | c}
$n \diagdown^k$ & $0$ & $1$ & $2$ & $3$ & $4$ & $5$ & $B_n$ \\ \midrule
0 & $1$ & $0$ & $0$ & $0$ & $0$ & $0$ & $1$ \\
1 & $0$ & $1$ & $0$ & $0$ & $0$ & $0$ & $1$ \\
2 & $0$ & $1$ & $1$ & $0$ & $0$ & $0$ & $2$ \\
3 & $0$ & $1$ & $3$ & $1$ & $0$ & $0$ & $5$ \\
4 & $0$ & $1$ & $7$ & $6$ & $1$ & $0$ & $15$ \\
5 & $0$ & $1$ & $15$ & $25$ & $10$ & $1$ & $52$ \\
\end{tabular}
\end{table}

Naj bodo $a_{ij}$ elementi zgornje tabele (brez zadnjega stolpca), kjer indeks $i$ predstavlja zaporedno število vrste k kateri pripada $a_{ij}$, šteto od zgoraj navzdol, in $j$ predstavlja zaporedno število stolpca v katerem je $a_{ij}$, šteto od leve proti desni. 

Opazimo sledeče:

Za $i, j > 0$ velja, da je $a_{ij} = a_{(i - 1)(j - 1)} + j * a_{(i-1)j}$

To pa sovpada s prej dokazano trditvijo (v bistvu je tabela bila izpolnjena s pomoćjo prej izračunanih začetnih pogojev in dokazane formule).

\begin{izr}
Za $n \geq 0$ je $B_{n + 1} = \sum_{k = 0}^{n}{\binom{n}{k} B_k}$
\end{izr}

\begin{proof}
Naj bo $A$ množica moči $n + 1$ in $P$ množica vseh particij $A$. Naj bo $a\in A$ izbrani element. Velja: \[ B_{n + 1} = \abs{P} \]

Za $k = 1, 2, \ldots, n$ naj bo $P_k = \{ R \in P \mid \abs{R[a]} = n - k + 1 \}$, pri čemer je $R[a]$ blok particije $R$, ki vsebuje $a$. Torej je izven $R[a]$ natanko $k$ elementov.

~

Konstrukcija razdelitve $R$ iz $P_k$: \begin{itemize}
	\item Izberemo $k$ elementov iz $A\setminus\{a\}$ na $\binom{n}{k}$ načinov. Ta izbor poimenujemo kot množico $C$; $\abs{C} = k \land C \subseteq A\setminus\{a\}$
	\item Izberemo poljubno razdelitev $C$ na $B_k$ načinov.
	\item obe vrednosti zmnožimo po pravilu produkta
	\item Ker velja $P_i \cap P_j = \emptyset ; i \neq j$, vse produkte seštejemo po pravilu vsote (sledi iz dejstva, da je $B_{n + 1} = \abs{P} = \abs{\bigcap_{k = 0}^{n}{P_k}}$).
\end{itemize}

Torej je $B_{n + 1} = \sum_{k = 0}^{n}{\abs{P_k}} = \sum_{k = 0}^{n}{\binom{n}{k} B_k}$

\end{proof}

\begin{zgled}
	\begin{table}[!h]
		\centering
		\begin{tabular}{c | ccccc | c}
			$n$ & $0$ & $1$ & $2$ & $3$ & $4$ & $5$ \\ \midrule
			$B_n$ & $1$ & $1$ & $2$ & $5$ & $15$ & ??? 
		\end{tabular}
	\end{table}
Izračunajmo $B_{5}$:
\begin{alignat*}{1}
B_5 & = \sum_{k = 0}^4{\binom{4}{k} B_k} ~= \\
& = 1 * B_0 + 4 * B_1 + 6 * B_2 + 4 * B_3 + 1 * B_4 ~= \\
& = 1 * 1 + 4 * 1 + 6 * 2 + 4 * 5 + 1 * 15 ~= \\
& = 52
\end{alignat*}
\end{zgled}

\newpage

\subsection{Algebraičen pomen Stirlingovih števil:}

\begin{trditev}
	Za vse $n \in \No$ in vse $x \in \C$ je $x^n = \sum_{k = 0}^{n}{S(n, k) \padex{x}{k}}$.
\end{trditev}

\begin{proof}
	Trditev lahko dokažemo bodisi z indukcijo po $n$ (prepuščeno bralcu za vajo) bodisi kombinatorično. Tukaj bomo trditev dokazali kombinatorično.
	
	~
	
	Naj bo $x\in\No$ in naj bota $A$ in $X$ množici, pri čimer velja $\abs{A} = n$ in $\abs{X} = x$.
	
	~
	
	Potem velja:
	\[ x^n = \abs{X^A} = \abs{\{\map{f}{A}{X}\}} = \abs{\bigcup_{B \subseteq X} \{ \map{f}{A}{X} \mid \pra{f}{A}{B}\}} = \abs{\bigcup_{B \subseteq X} \{ \map{\bar{f}}{A}{B} \mid \pra{f}{A}{B}\}}\]
	
	~
	
	Z naslednjim korakom >>združimo<< tiste $B$, ki imajo enako število elementov.
	
	\begin{align*}
	 \abs{\bigcup_{B \subseteq X} \{ \map{\bar{f}}{A}{B} \mid \pra{f}{A}{B}\}} & = \abs{\bigcup_{k = 0}^{n}~\Bigg( \bigcup_{B \subseteq X \land \abs{B} = k} \{ \map{\bar{f}}{A}{B} \mid \bar{f}\text{~je surjekcija}\}\Bigg)} \\
	 & = \sum_{k = 0}^{n}\Bigg(\sum_{B \subseteq X \land \abs{B} = k} \abs{\{ \map{\bar{f}}{A}{B} \mid \bar{f}\text{~je surjekcija}\}} \Bigg) \\
	 & = \sum_{k = 0}^{n}\Bigg(\sum_{B \subseteq X \land \abs{B} = k} S(n, k) * k! \Bigg) \\
	 & = \sum_{k = 0}^{n} \binom{x}{k}S(n, k) * k! = \sum_{k = 0}^{n} S(n, k) * \padex{x}{k}
	\end{align*}
	
	~
	
	Dokaz bi lahko tudi argumentirali na sledeči način:
	
	~
	
	Naj bosta $x^n = p(x) \in\C[x]$ in $\sum_{k = 0}^{n} S(n, k) * \padex{x}{k} = q(x) \in\C[x]$ kompleksna polinoma. Velja:
	\begin{align*}
	p(x) = q(x) \text{za vsak} x \in \No & \Rightarrow p(x) - q(x) = 0 \text{~za vsak~} x\in\No \\
	& \Rightarrow p - q \in \C[x] \text{~ima neskončno mnogo ničel~} \\
	& \Rightarrow p - q \equiv 0 \Rightarrow p(x) = q(x) \text{~za vsak~} x\in\C
	\end{align*}
	
\end{proof}
	 
\end{document}
